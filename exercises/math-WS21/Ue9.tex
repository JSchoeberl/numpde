\documentclass[11pt,a4paper]{report}
\usepackage{exscale,times}
\usepackage{graphicx}
\usepackage{epstopdf}
\usepackage{amsmath,amsthm,amssymb}
\usepackage{epsfig}
\usepackage{latexsym}
\usepackage{amssymb}
\usepackage{german}
\setlength{\parindent}{0pt}
\setlength{\parskip}{5pt plus 2pt minus 1 pt}
\topmargin  -5mm
\evensidemargin 8mm
\oddsidemargin  2mm
\textwidth  158mm
\textheight 230mm
\frenchspacing
\sloppy
\newcommand{\xvec}{\bold{x}}
\newcommand{\vvec}{\bold{v}}
\newcommand{\wvec}{\bold{w}}
\newcommand{\evec}{\bold{e}}
\newcommand{\nuvec}{\bold{\nu}}
\newcommand{\R}[1]{\mathbb{R}^{#1}}
\newcommand{\setR}{\mathbb{R}}
\newcommand{\p}{$p$}
\newcommand{\norm}[2][2]{\|#2 \|_#1}
\newcommand{\diver}{\text{div}}
\newcommand{\refer}[1]{(\ref{#1})}
\newcommand{\ndof}{\text{ndof}}
\newcommand{\gauss}[2]{e^{-\frac{(\vvec #2)^2}{#1}}}
\newcommand{\lagrange}[2]{l_i(\frac{\vvec #2}{\sqrt{#1}})}


\begin{document}
\begin{center}
\textbf{9. \"Ubung Numerik von partiellen Differentialgleichungen - station\"are Probleme} \newline 
\textbf{9. Dezember 2021}
\end{center}

\setcounter{enumi}{4}

\begin{enumerate}

\item Wenden Sie Adaptivit\"at auf das Bsp aus \"U2.1
  ein. Vergleichen Sie unterschiedliche Fehlersch\"atzer aus der
  Vorgabe. Achten Sie darauf den W\"armeleitkoeffizienten $\lambda$
  richtig einzubauen.


 \item Goal Driven Adaptivity. Sei $\Omega = (0,3) \times (0,2)$, $u$
   L\"osung des homogenen Dirichletproblems der Poissongleichung
   mit Quelle $f = \tfrac{1}{|\Omega_S|}   \chi_{\Omega_S}$, wobei $\Omega_S =
     (1-\varepsilon,1+\varepsilon)^2$.  Gesucht ist der Punktwert
     $u(2,1)$. Dieser soll durch Mittelung \"uber eine Kreis mit
     Radius $\varepsilon$ {\em angen\"ahert} werden. W\"ahlen Sie
     z.B. $\varepsilon = 10^{-3}$.

     Stellen Sie die Auswertung der Zielgr\"o\ss{}e \"uber ein lineares
     Funktional dar. Bestimmen Sie einen Referenzwert \"uber eine sehr
     genaue Rechnung.

     Verwenden Sie nun Adaptivit\"at um effiziente Netzte zu
     generieren (plotten Sie dazu Fehler \"uber Freiheitsgrade). 
     Vergleichen Sie Fehlersch\"atzer f\"ur die Energienorm, und
     goal-driven error estimates.


   \item Auf dem Gebiet $\omega$ gelte die Poincar\'e-Ungleichung mit
     Konstante $c_P$:
     $$
     \| u \|_{L_2(\omega)} \leq c_p \, \| \nabla u \|_{L_2(\omega)} \qquad
       \forall \, u \in H^1 \text{ mit } \int_\omega u = 0.
     $$
     Wir denken an einen Vertex-Patch, mit $c_P = c h$, wobei $c$ nur
     von der Form der Dreiecke abh\"angt.

     Zeigen Sie:
     $$
     \inf_{q \in P^1} \| u - q \|_{L_2(\omega)} \leq c_p^2 \, \| \nabla^2
     u \|_{L_2}
     $$

     Hinweis: \"Uberlegen Sie sich einen passenden Interpolationsoperator
   \item
     Sei $V_h \subset H^1$  FEM - Raum erster Ordnung. 
     Der originale  Cl\'ement Operator ist wie folgt definiert:
  $\omega_X$ der Vertex-patch zum Vertex $X$, und
  $$
  V_{h, \omega_X} = \{ v_{h|\omega_X} : v_h \in V_h \}.
$$
Sei $P_{\omega_X}$ der $L_2(\omega_X)$-orthogonale Projektor auf
$V_{h,\omega_X}$, und $\phi_X$ die Hut-Basisfunktion zum Knoten $X$. Dann ist
$$
\Pi_h^{Cl} v = \sum_{X \in {\text Vertices}}  \big( P_{\omega_X} v \big)
(X) \phi_X 
$$

Zeigen Sie daf\"ur:
$$
| u - \Pi_h^{Cl} u |_{H^k(T)} \leq c h^{m-k} | u |_{H^m(\omega_T)}
$$
f\"ur $k \in \{ 0,1 \}$, $m \in \{ 0, 1, 2 \}$ und $k \leq m$.

\end{enumerate}
\end{document}