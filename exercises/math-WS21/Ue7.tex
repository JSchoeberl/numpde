\documentclass[11pt,a4paper]{report}
\usepackage{exscale,times}
\usepackage{graphicx}
\usepackage{epstopdf}
\usepackage{amsmath,amsthm,amssymb}
\usepackage{epsfig}
\usepackage{latexsym}
\usepackage{amssymb}
\usepackage{german}
\setlength{\parindent}{0pt}
\setlength{\parskip}{5pt plus 2pt minus 1 pt}
\topmargin  -5mm
\evensidemargin 8mm
\oddsidemargin  2mm
\textwidth  158mm
\textheight 230mm
\frenchspacing
\sloppy
\newcommand{\xvec}{\bold{x}}
\newcommand{\vvec}{\bold{v}}
\newcommand{\wvec}{\bold{w}}
\newcommand{\evec}{\bold{e}}
\newcommand{\nuvec}{\bold{\nu}}
\newcommand{\R}[1]{\mathbb{R}^{#1}}
\newcommand{\setR}{\mathbb{R}}
\newcommand{\p}{$p$}
\newcommand{\norm}[2][2]{\|#2 \|_#1}
\newcommand{\diver}{\text{div}}
\newcommand{\refer}[1]{(\ref{#1})}
\newcommand{\ndof}{\text{ndof}}
\newcommand{\gauss}[2]{e^{-\frac{(\vvec #2)^2}{#1}}}
\newcommand{\lagrange}[2]{l_i(\frac{\vvec #2}{\sqrt{#1}})}


\begin{document}
\begin{center}
\textbf{7. \"Ubung Numerik von partiellen Differentialgleichungen - station\"are Probleme} \newline 
\textbf{25. November 2021}
\end{center}

\setcounter{enumi}{4}

\begin{enumerate}

\item Implementieren Sie Dreieckselemente 3. Ordnung, und l\"osen Sie
  damit Differentialgleichungen.

  Hinweis: Hilfreich sind die vertex-Nummern der Entpunkte einer
  Kante: \\  {\tt  auto [v1,v2] = ma->GetEdgePNums(edgenr); }.

  
\item Implementieren Sie eine {\tt ProjectLocally} Funktion \"ahnlich
  zu {\tt GridFunction.Set}. Eine gegebene
  CoefficientFunction soll in eine skalare GridFunction interpoliert
  werden. F\"uhren Sie dazu element-weise $L_2$-Projektionen durch,
  und mitteln arithmetisch \"uber die koppelnden
  Freiheitsgrade. Implementieren Sie die Funktion in ein C++
  Erweiterungsmodul zu NGSolve. \\
  Hinweis: {\tt CalcInverse(mat); } invertiert die {\tt Matrix mat}.
  
\item Mass lumping. Wiederholen Sie \"U 3.1 zur
  Diffusions-Reaktionsgleichung. Verwenden Sie Elemente 1. Ordnung,
  und mass-lumping zur Integration vom $\int u v$ Term. Das geht in
  NGSolve so:

  \begin{verbatim}
  ir = IntegrationRule(points = [(0,0), (1,0), (0,1)], \
                       weights = [1/6, 1/6, 1/6] )
  dx_lumping = dx(intrules = { TRIG : ir }) 
  a += u*v*dx_lumping
\end{verbatim}

  Was f\"allt bei nicht aufgel\"osten Grenzschichten auf ?

  Analysieren Sie den Konsistenzfehler, d.h. weisen Sie Absch\"atzung
  (4.13) aus dem Skript nach.


\item Nicht-konforme P1 Elemente.  Wiederholen Sie \"U 2.1
  (notebook Uebung2Bsp5.ipynb). L\"osen Sie das Problem mit
  nicht-konformen P1 Elementen, und interpolieren den Fluss $\lambda
  \nabla u_h$ in einen
  HDiv(order=0) Raum (GridFunction fluxhdiv). Berechnen Sie davon die Fl\"usse durch die einzelnen
  Kanten vom Rechteck, und deren Summe.

  Plotten Sie den Interpolationsfehler
  $$
  \lambda \nabla u_h - fluxhdiv,
  $$
  wobei $\nabla u_h$ hier element-weise zu  interpretiert ist.
  
  Den nicht-konformen Raum erhalten Sie \"uber 
\begin{verbatim}
fes = FESpace(mesh, "nonconforming")
\end{verbatim}
 Plotten Sie Basisfunktionen. Alternativ k\"onnen Sie ihn auch leicht selbst implementieren.


  Zeigen Sie: Wenn $T_l$ und $T_r$ zwei Dreiecke an einer gemeinsamen
  Kante $E$ sind, $f = 0$ auf $T_1 \cup T_2$, und $\lambda$
  elemente-weise konstant, dann gilt:
  $$
  \lambda_l \frac{\partial u_h|_{T_l}}{\partial n_l} =
  -\lambda_r \frac{\partial u_h|_{T_r}}{\partial n_r},
  $$
  wobei $u_h$ die FEM - L\"osung im nicht-konformen P1 Raum ist.
  Hinweis: Setzen Sie die Basisfunktion $\varphi_E$ zur Kante $E$ in die diskrete
  Variationsformulierung ein.

\end{enumerate}
\end{document}