\documentclass[11pt,a4paper]{report}
\usepackage{exscale,times}
\usepackage{graphicx}
\usepackage{epstopdf}
\usepackage{amsmath,amsthm,amssymb}
\usepackage{epsfig}
\usepackage{latexsym}
\usepackage{amssymb}
\usepackage{german}
\setlength{\parindent}{0pt}
\setlength{\parskip}{5pt plus 2pt minus 1 pt}
\topmargin  -5mm
\evensidemargin 8mm
\oddsidemargin  2mm
\textwidth  158mm
\textheight 230mm
\frenchspacing
\sloppy
\newcommand{\xvec}{\bold{x}}
\newcommand{\vvec}{\bold{v}}
\newcommand{\wvec}{\bold{w}}
\newcommand{\evec}{\bold{e}}
\newcommand{\nuvec}{\bold{\nu}}
\newcommand{\R}[1]{\mathbb{R}^{#1}}
\newcommand{\setR}{\mathbb{R}}
\newcommand{\p}{$p$}
\newcommand{\norm}[2][2]{\|#2 \|_#1}
\newcommand{\diver}{\text{div}}
\newcommand{\refer}[1]{(\ref{#1})}
\newcommand{\ndof}{\text{ndof}}
\newcommand{\gauss}[2]{e^{-\frac{(\vvec #2)^2}{#1}}}
\newcommand{\lagrange}[2]{l_i(\frac{\vvec #2}{\sqrt{#1}})}


\begin{document}
\begin{center}
\textbf{8. \"Ubung Numerik von partiellen Differentialgleichungen - station\"are Probleme} \newline 
\textbf{2. Dezember 2021}
\end{center}

\setcounter{enumi}{4}

\begin{enumerate}

\item Implementieren Sie high-order Viereckselemente in
  NGSolve. Der HighOrderFESpace sollte Netze unterst\"utzen die
  gemischt aus Dreiecken und Vierecken bestehen.
  Testen Sie die Genauigkeit am L-shape-domain $(-1,1)^2 - [0,1]^2$ mit hp-Refinement.

\item Die $L_2(-1,1)$-orthogonalen Legendrepolynome (siehe Vorlesung)
  erf\"ullen eine Drei-Term-Rekursion
  $$
  P_{n+1}(x) = \alpha_n x P_n(x) + \beta_n P_{n-1} (x)
  $$
  Die {\em Integrierten Legendrepolynome} sind definiert als
  $$
  L_n(x) = \int_{-1}^x P_{n-1} (s) \, {ds}
  $$

  Zeigen Sie:
  \begin{itemize}
  \item
    $$
    \int_{-1}^1 L(x) q(x) \, dx = 0 \qquad \forall \, q \in \Pi_{n-3}
    $$
  \item
    Auch die Integrierten Legendrepolynome erf\"ullen eine Drei-Term-Rekursion
    $$
    L_{n+1}(x) = \gamma_n x L_n(x) + \delta_n L_{n-1}(x)
    $$
    Hinweis: Koeffizientenvergleich der f\"uhrenden Koeffizienten, und
    Orthogonalit\"at f\"ur den Rest.
  \item 
    $$
    L_n(x) = c_n (P_n(x) - P_{n-2}(x))
    $$
  \end{itemize}
  Implementieren Sie Python-Funktionen um $L_n$ und $P_n$ zu bestimmen und
  plotten die Funktionen (auf einem sinnvoll gew\"ahlten Definitionsbereich).
\item
  Bestimmen Sie Elementmatrizen f\"ur das high order 1D Element
  $[-1,1]$  f\"ur $\int_{-1}^1 uv$ und $\int_{-1}^1
  u^\prime v^\prime$ per Hand. Verwenden Sie die hierarchische Basis
  (d.h. Basisfunktionen f\"ur Ordnung 1, und Integrierte Legendrepolynome).

\item Sch\"atzen Sie folgende Terme durch die $H^1$-Seminorm von $u$
  auf dem entsprechenden Definitionsbereich ab,. d.h.  $\ldots \leq c h^? \| \nabla u \|$:
    \begin{itemize}
  \item 
    $\| u - \overline{u}^T \|_{L_2(T)}$
  \item 
    $\| u - \overline{u}^E \|_{L_2(E)}$
  \item
   $| \overline{u}^T - \overline{u}^E |$  \qquad (als Zahl) 
 \item
   $| \overline{u}^T - \overline{u}^{\tilde T} |$  \qquad (als Zahl) 
  \item 
    $\| u - \overline{u}^{T \cup \tilde T} \|_{L_2(T\cup \tilde T)}$

    \end{itemize}
  Dabei ist $T$ ein Dreieck, $E$ eine Kante von $T$, und $\tilde T$
  ein weiteres Dreick mit $T \cap \tilde T = E$. Mittelung \"uber
  Gebiete wird mit $\overline{u}^T, \overline{u}^E, \ldots $
  geschreiben, wobei das Symbol sowohl f\"ur den Mittelwert als auch
  f\"ur die konstante Funktion steht.

  Hinweis: Vieles geht mit Transformation und Bramble-Hilbert - aber
  Vorsicht !
\end{enumerate}
\end{document}