\documentclass[11pt,a4paper]{report}
\usepackage{exscale,times}
\usepackage{graphicx}
\usepackage{epstopdf}
\usepackage{amsmath,amsthm,amssymb}
\usepackage{epsfig}
\usepackage{latexsym}
\usepackage{amssymb}
\usepackage{german}
\setlength{\parindent}{0pt}
\setlength{\parskip}{5pt plus 2pt minus 1 pt}
\topmargin  -5mm
\evensidemargin 8mm
\oddsidemargin  2mm
\textwidth  158mm
\textheight 230mm
\frenchspacing
\sloppy
\newcommand{\xvec}{\bold{x}}
\newcommand{\vvec}{\bold{v}}
\newcommand{\wvec}{\bold{w}}
\newcommand{\evec}{\bold{e}}
\newcommand{\nuvec}{\bold{\nu}}
\newcommand{\R}[1]{\mathbb{R}^{#1}}
\newcommand{\setR}{\mathbb{R}}
\newcommand{\p}{$p$}
\newcommand{\norm}[2][2]{\|#2 \|_#1}
\newcommand{\diver}{\text{div}}
\newcommand{\refer}[1]{(\ref{#1})}
\newcommand{\ndof}{\text{ndof}}
\newcommand{\gauss}[2]{e^{-\frac{(\vvec #2)^2}{#1}}}
\newcommand{\lagrange}[2]{l_i(\frac{\vvec #2}{\sqrt{#1}})}

\newcommand{\opgrad}{\operatorname{grad}}
\newcommand{\opdiv}{\operatorname{div}}
\newcommand{\opcurl}{\operatorname{curl}}
\newcommand{\opdet}{\operatorname{det}}
\newcommand{\optr}{\operatorname{tr}}
\newcommand{\optrn}{\operatorname{tr}_n}
\newcommand{\sfrac}[2]{ { \textstyle \frac{#1}{#2} } }
\newcommand{\eps}{\varepsilon}




\begin{document}
\begin{center}
\textbf{11. \"Ubung Numerik von partiellen Differentialgleichungen - station\"are Probleme} \newline 
\textbf{20. J\"anner 2022}
\end{center}

\setcounter{enumi}{4}

\begin{enumerate}
\item (Theorem 132) Beweisen Sie die commuting diagram property
  $$
  I^Q \opcurl = \opcurl I^V,
  $$
  d.h. den mittleren Teil vom de Rham Komplex
  
\begin{equation}
\begin{array}{ccccccc}
H^1             &      \stackrel{\nabla}{\longrightarrow}          &
H(\opcurl)      &      \stackrel{\opcurl}{\longrightarrow}   &
H(\opdiv)       &      \stackrel{\opdiv}{\longrightarrow}    & 
L^2                                                                                    \\[8pt]
\Big\downarrow  \vcenter{ \rlap{$I^W$}}  &                  &
\Big\downarrow  \vcenter{ \rlap{$I^V$}}  &                  &
\Big\downarrow  \vcenter{ \rlap{$I^Q$}}  &                  &
\Big\downarrow  \vcenter{ \rlap{$I^S$}}                             \\[8pt]
 W_h                   &      
\stackrel{\nabla}{\longrightarrow}          &
 V_h       &     
 \stackrel{\opcurl}{\longrightarrow}   &
 Q_h          &      
\stackrel{\opdiv}{\longrightarrow}    & 
S_h  \:.                                                                               \\[8pt]
\end{array}
\label{equ_derham}
\end{equation}

  
Zeigen Sie dazu dass linke und rechte Seite im gleichen FEM - Raum
liegen, und dass die Funktionale (Freiheitsgrade) \"ubereinstimmen.
  

\item Sei $j$ eine divergenzfreie Stromdichte im Sinne von $(j, \nabla \psi)
  _{L_2}=0  \,\forall \psi \in H^1$. Es sei $A = A^0$ die L\"osung
  von (8.13), und $A^\eps$ die L\"osung des regularisierten Problems
  (mit $\eps \in \setR^+$): 
Ges.: $A^\eps \in H(\opcurl)$: 
$$
\int \mu^{-1} \opcurl \, A^\eps \cdot \opcurl v + \eps \, A^\eps \cdot v = \int
j \cdot v \qquad \forall \, v \in H(\opcurl)
$$
Zeigen Sie: $\| A - A^\eps \|_{H(\opcurl)} = O(\eps)$
 
Hinweis: Bringen Sie beide Gleichungen in gemischte Form, und
subtrahieren diese um eine Gleichung f\"ur die Differenz zu
bekommen. Sch\"atzen Sie dann diese Differenz mit Hilfe der stabilen
L\"osbarkeit durch die rechte Seite ab.

\item Betrachten Sie die (erweiterte) diskrete Sequenz aus Thm 131
  $$
  \begin{array}{ccccccccc}
    \setR &  \stackrel{\operatorname{id}}{\longrightarrow}          &
W_h             &      \stackrel{\opgrad}{\longrightarrow}          &
V_h             &      \stackrel{\opcurl}{\longrightarrow}   &
Q_h             &      \stackrel{\opdiv}{\longrightarrow}    & 
S_h                                                                                    
\end{array}
$$
wobei $W_h$, $V_h$, $Q_h$, $S_h$ FEM - R\"aume niedrigster Ordnung in
$H^1$, $H(\opcurl)$, $H(\opdiv)$ und $L_2$ sind. Bestimmen Sie die
Dimensionen von Bild $R(\cdot)$ und Kern $N(\cdot)$ der einzelnen
Differentialoperatoren (siehe jupyter-Vorlag). \"Uberpr\"ufen Sie experimentell
\begin{eqnarray*}
    \operatorname{dim} W_h & \stackrel{?}{=}   & \operatorname{dim}
                                               R(\operatorname{id}) +
                                               \operatorname{dim}
                                               R(\opgrad)  \\
  \operatorname{dim} V_h & \stackrel{?}{=}   & \operatorname{dim}
                                               R(\opgrad) +
                                               \operatorname{dim}
                                               R(\opcurl)  \\
  \operatorname{dim} Q_h & \stackrel{?}{=}   & \operatorname{dim}
                                               R(\opcurl) +
                                               \operatorname{dim}
                                               R(\opdiv)  
\end{eqnarray*}

Bestimmen Sie eine Gleichung f\"ur die Anzahl vertices \#V, edge \#E, faces \#F, und
cells \#T f\"ur Netze auf einfach zusammenh\"angenden Gebieten. 

Was \"andert sich wenn das Gebiet nicht mehr einfach zusammenh\"angend
ist, sondern L\"ocher und Tunnel hat ? 

Vgl.: Betti-numbers,  siehe  Doug Arnold, {\em Finite Element Exterior
  Calculus}, S 14-15

\item Bestimmen Sie die Induktivit\"at einer Zylinderspule.

Bei einer Zylinderspule ist ein Draht mit N Windungen in Form eines
Hohlzylinders (Radien $r_i$ und $r_o$, und L\"ange $l$)
aufgewickelt. Durch den Draht flie\ss{}e ein Strom von $I = 1$ Ampere.
Berechnen Sie dazu das Magnetfeld $B = \opcurl A$ durch die
Gleichung der Magnetostatik
$$
\int \mu^{-1} \opcurl A \, \opcurl v = \int j v \qquad \forall \, v
$$
(wobei $\mu = 1.257 \cdot 10^{-6}$ die Permeabilit\"at in Luft
ist). Setzen Sie eine Luftbox um die Spule.
Regularisieren Sie wie in Bsp 2. Um nicht alle Dr\"ahte
einzeln aufl\"osen zu m\"ussen, verwenden Sie die Stromdichte $j$ (ein
Vektorfeld, Einheit $Ampere / m^2$), das in Richtung des Drahtes zeigt,
und das Integral \"uber den Querschnitt $N I $ ergibt.

Die Energie im Magnetfeld ist
$$
\tfrac{1}{2}  \int \mu^{-1} | B |^2.
$$

Die Induktivit\"at L ergibt sich nun so dass die magnetische Energie zugleich
$$
\tfrac{1}{2} L I^2
$$
ist. Diese Kenngr\"o\ss{}e $L$ kann dann z.B. f\"ur eine
Netzwerksimulation verwendet werden.

Ist die Dicke des Hohlzylinders sehr klein ($r_o - r_i \ll r_i$), dann
kann die Stromdichte auch als Fl\"achenstromdichte 
modelliert werden, d.h. durch ein Oberfl\"achenintegral am Zylindermantel.

Dazu gibt es eine analytische L\"osung, siehe z.B.
  https://de.wikipedia.org/wiki/Zylinderspule. Vergleichen Sie Ihre
  numerisch berechnete Induktivit\"at mit dem analytischen Ergebnis.
  
\end{enumerate}

\bigskip
\end{document}