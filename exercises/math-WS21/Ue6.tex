\documentclass[11pt,a4paper]{report}
\usepackage{exscale,times}
\usepackage{graphicx}
\usepackage{epstopdf}
\usepackage{amsmath,amsthm,amssymb}
\usepackage{epsfig}
\usepackage{latexsym}
\usepackage{amssymb}
\usepackage{german}
\setlength{\parindent}{0pt}
\setlength{\parskip}{5pt plus 2pt minus 1 pt}
\topmargin  -5mm
\evensidemargin 8mm
\oddsidemargin  2mm
\textwidth  158mm
\textheight 230mm
\frenchspacing
\sloppy
\newcommand{\xvec}{\bold{x}}
\newcommand{\vvec}{\bold{v}}
\newcommand{\wvec}{\bold{w}}
\newcommand{\evec}{\bold{e}}
\newcommand{\nuvec}{\bold{\nu}}
\newcommand{\R}[1]{\mathbb{R}^{#1}}
\newcommand{\setR}{\mathbb{R}}
\newcommand{\p}{$p$}
\newcommand{\norm}[2][2]{\|#2 \|_#1}
\newcommand{\diver}{\text{div}}
\newcommand{\refer}[1]{(\ref{#1})}
\newcommand{\ndof}{\text{ndof}}
\newcommand{\gauss}[2]{e^{-\frac{(\vvec #2)^2}{#1}}}
\newcommand{\lagrange}[2]{l_i(\frac{\vvec #2}{\sqrt{#1}})}


\begin{document}
\begin{center}
\textbf{6. \"Ubung Numerik von partiellen Differentialgleichungen - station\"are Probleme} \newline 
\textbf{18. November 2021}
\end{center}

\setcounter{enumi}{4}

\begin{enumerate}

\item (2 Punkte) Implementieren Sie Finite Elemente erster und zweiter
  Ordnung in NGSolve f\"ur 
  \begin{itemize}
  \item Linienelemente
  \item Viereckselemente
  \item Tetraederelemente
  \end{itemize}

Verwenden Sie Ihre Elemente um Differentialgleichungen zu l\"osen.
  
\item Zeigen Sie f\"ur $\Omega$ Lipschitz: $H_0^1(\Omega) = [L_2(\Omega),
H_0^2(\Omega)]_{1/2}$.

Hinweis: \"Ahnlich zur Vorlesung Thm 59, jetzt ist $E$ trivial und $R$
nicht. Geeignete Fortsetzungsoperatoren d\"urfen Sie ohne Beweis verwenden. 

\item F\"ur welche $\alpha$ gilt folgende Variante der Spur-Absch\"atzung:
$$
|u(0)|^2 \leq c \, \int_0^1 x^\alpha \, u^\prime(x)^2 \, \text{dx} + \int_0^1
x^\alpha \, u(x)^2 \, \text{dx}
$$

Zur Entscheidung beim kritischen Wert von $\alpha$ hilft die Funktion
$u(x) = \log \log x$.

Welche Auswirkung hat diese Ergebnis auf m\"ogliche Randbedingungen der schwachen Formulierung
am Symmetrierand bei Verwendung von Zylinderkoordinaten?


\end{enumerate}
\end{document}