\documentclass[11pt,a4paper]{report}
\usepackage{exscale,times}
\usepackage{graphicx}
\usepackage{epstopdf}
\usepackage{amsmath,amsthm,amssymb}
\usepackage{epsfig}
\usepackage{latexsym}
\usepackage{amssymb}
\usepackage{german}
\setlength{\parindent}{0pt}
\setlength{\parskip}{5pt plus 2pt minus 1 pt}
\topmargin  -5mm
\evensidemargin 8mm
\oddsidemargin  2mm
\textwidth  158mm
\textheight 230mm
\frenchspacing
\sloppy
\newcommand{\xvec}{\bold{x}}
\newcommand{\vvec}{\bold{v}}
\newcommand{\wvec}{\bold{w}}
\newcommand{\evec}{\bold{e}}
\newcommand{\nuvec}{\bold{\nu}}
\newcommand{\R}[1]{\mathbb{R}^{#1}}
\newcommand{\setR}{\mathbb{R}}
\newcommand{\p}{$p$}
\newcommand{\norm}[2][2]{\|#2 \|_#1}
\newcommand{\diver}{\text{div}}
\newcommand{\refer}[1]{(\ref{#1})}
\newcommand{\ndof}{\text{ndof}}
\newcommand{\gauss}[2]{e^{-\frac{(\vvec #2)^2}{#1}}}
\newcommand{\lagrange}[2]{l_i(\frac{\vvec #2}{\sqrt{#1}})}


\begin{document}
\begin{center}
\textbf{2. \"Ubung Numerik von partiellen Differentialgleichungen - station\"are Probleme} \newline 
\textbf{21. Oktober 2021}
\end{center}

\setcounter{enumi}{4}

\begin{enumerate}

\item Experimente mit $H(\operatorname{div})$, siehe jupyter - notebook.

\item Sei $V$ ein Hilbertraum, $f$ eine stetige Linearform und $A$ eine stetige, symmetrische und elliptische Bilinearform. Sei $u$ die L\"osung des Variationsproblems
$$A(u,v) = f(v).$$
Zeigen Sie, dass f\"ur $u$
$$\sup\limits_{v\neq 0} \frac{f(v)}{\|v\|_A} = \|u\|_A$$
gilt. Zeigen Sie dass das Supremum f\"ur $v=u$ angenommen wird.
\vspace{15pt}



\item Sei $V$ ein Hilbertraum sowie $A: V \rightarrow V$ ein linearer,
  beschr\"ankter sowie selbstadjungierter Operator. Zeigen Sie:\\
$$ \| A\| := \sup\limits_{v \in V \atop v\neq 0} \frac{\|Av \|}{\| v
  \|} = \sup\limits_{v\in V\atop v\neq 0} \frac{|(Av,v)|}{\|v\|^2}$$

Hinweis:
\begin{enumerate}
\item
  Zeigen Sie zuerst das Ergebnis in $\setR^2$ mittels
  Eigenwerten. Hier ist $A$ eine symmetrische $2 \times 2$ Matrix.
\item
  \"Ubertragen Sie dann das Ergebnis auf allgemeine Hilbertr\"aume
  ohne weiter Spektraltheorie zu verwenden.
\end{enumerate}

\vspace{15pt}





\item Seien $X$, $Y$ Hilbertr\"aume mit gemeinsamer
  Vektorraum-Struktur (d.h. Sie k\"onnen Elemente von $X$ zu Elemente
  von $Y$ addieren). Auf  dem Summenraum $X+Y:= \{z = x+ y: x\in
  X\;y\in Y \}$ sei die Summenraumnorm
$$\| z\|_{X+Y}:= \inf\limits_{z=x+y \atop x\in X,\; y\in Y} \sqrt{\|x\|_X^2 +\|y\|_Y^2}.$$
definiert. Zeigen Sie, dass die Norm $\|\cdot \|_{X+Y}$ ein Skalarprodukt
induziert. Hinweis: Parallelogrammlemma.
\vspace{15pt}



\end{enumerate}
\end{document}

