\documentclass[11pt,a4paper]{report}
\usepackage{exscale,times}
\usepackage{graphicx}
\usepackage{epstopdf}
\usepackage{amsmath,amsthm,amssymb}
\usepackage{epsfig}
\usepackage{latexsym}
\usepackage{amssymb}
\usepackage{german}
\setlength{\parindent}{0pt}
\setlength{\parskip}{5pt plus 2pt minus 1 pt}
\topmargin  -5mm
\evensidemargin 8mm
\oddsidemargin  2mm
\textwidth  158mm
\textheight 230mm
\frenchspacing
\sloppy
\newcommand{\xvec}{\bold{x}}
\newcommand{\vvec}{\bold{v}}
\newcommand{\wvec}{\bold{w}}
\newcommand{\evec}{\bold{e}}
\newcommand{\nuvec}{\bold{\nu}}
\newcommand{\R}[1]{\mathbb{R}^{#1}}
\newcommand{\setR}{\mathbb{R}}
\newcommand{\p}{$p$}
\newcommand{\norm}[2][2]{\|#2 \|_#1}
\newcommand{\diver}{\text{div}}
\newcommand{\refer}[1]{(\ref{#1})}
\newcommand{\ndof}{\text{ndof}}
\newcommand{\gauss}[2]{e^{-\frac{(\vvec #2)^2}{#1}}}
\newcommand{\lagrange}[2]{l_i(\frac{\vvec #2}{\sqrt{#1}})}


\begin{document}
\begin{center}
\textbf{5. \"Ubung Numerik von partiellen Differentialgleichungen - station\"are Probleme} \newline 
\textbf{11. November 2021}
\end{center}

\setcounter{enumi}{4}

\begin{enumerate}


\item
Die $xy$ Ebene habe elektrisches Potential 0 Volt, und eine Metallkugel
mit Mittelpunkt $(0,0,1)$ Meter und Radius 10 cm habe Potential 1000
Volt. Welche Energie $E$ steckt im freien Raum $\R{} \times \R{} \times
\R{+} \setminus Kugel$ ? Das Potential $u$ ergibt sich dort aus der
Poissongleichung mit den vorgegebenen Dirichletrandwerten. Die Energie
ist $E = \frac12\int
\epsilon_0 | \nabla u |^2 \, dx$, mit der Permittivit\"at von Vakuum
$\epsilon_0 = 8.854 \cdot 10^{-12} \frac{As} {Vm}$. L\"osen Sie das Problem mit FEM. Schneiden
Sie dazu das Gebiet mit dem Zylinder $\{ x^2 + y^2 < R^2, z < R \}$, und
setzen am k\"unstlichen Rand a) homogene Dirichlet, oder b) homogene
Neumann Randbedingungen, und berechnen Sie $E_D(R)$ und $E_N(R)$. 
Zeigen Sie: $E_D$ ist monoton fallend, $E_N$ monoton wachsend, und $E$
liegt dazwischen.


\item Obiges Beispiel hat eine zylinder-symmetrische
  L\"osung, d.h. $u(x,y,z) = \tilde u(r,z)$.  L\"osen Sie es unter Verwendung von Zylinderkoordinaten.
Bestimmen Sie dazu eine Variationsformulierung auf zwei Arten:
\begin{enumerate}
\item starten Sie vom Laplace-Operator in Zylinderkoordinaten, und
  leiten dazu eine symmetrische VF her
\item stellen Sie in der 3D Variationsformulierung $\nabla u$ und
  $\nabla v$ in Zylinderkoordinaten dar, und setzen
  rotationssymmetrische Ansatz- und Testfunktionen ein.
\end{enumerate}
Was ist der nat\"urliche Hilbertraum ?
L\"osen Sie das Beispiel mit FEM, und vergleichen mit der 3D Simulation. Diskutieren Sie m\"ogliche Randbedingungen auf
der Symmetrieachse.


  
 \item
Seien $V_1 \subset V_0$ geeignet f\"ur Banach-Raum Interpolation. Beweisen Sie
$$
\| u \|_{s} \leq c \|u \|_0^{1-s} \|u \|_1^s \qquad \forall \, u \in V_1
$$
Hinweis: Integral \"uber K-Funktional aufspalten, und die beiden
trivialen Absch\"atzungen f\"ur $K(t,u)$ verwenden.

 \item Sei $I=(-1, 1)$ sowie $T$ das Dreieck mit Eckpunkten $(-1,0),\,(1,0),\,(0,1)$. Wir definieren den Fortsetzungsoperator $\mathcal{E}$:
$$\mathcal{E}: \left\{
							\begin{matrix}   L_2(I) \rightarrow L_2(T) \\ 
					 									u \mapsto \frac{1}{2y}\int\limits_{x-y}^{x+y}u(s) \,ds
					    \end{matrix}\right.$$
Zeigen Sie, dass f"ur $y\in (0,1)$ 
$$\int\limits_{-1+y}^{1-y} |\mathcal{E}u|^2 + |\nabla(\mathcal{E}u)|^2\,dx \leq c\, \min\{\|u\|^2_{H^1(I)}, \frac{1}{y^2}\|u\|^2_{L_2(I)}\}$$
gilt.

Zeigen Sie damit weiters
$$
\| \nabla \mathcal{E}u \|_{L_2([y-1,1-y] \times \{y\}) } \leq c \, \|u \|_{y^{-1}L_2(I) + H^1(I)}. 
$$
Rechts steht die Norm eines Summenraums, $s^{-1}L_2$ ist der Raum $L_2$
versehen mit der \newline Norm $s^{-1}\| \cdot \|_{L_2}$.

Schlie\ss{}en Sie daraus
$$
\| \nabla \mathcal{E}u \|_{L_2(T) } \leq c \, \|u\|_{H^{1/2)}(I)}
  $$
Hinweis: Integration \"uber $y$, Interpolationsraum \"uber K-Funktional.

  
\end{enumerate}
\end{document}

