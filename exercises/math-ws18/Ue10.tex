\documentclass[11pt,a4paper]{report}
\usepackage{exscale,times}
\usepackage{graphicx}
\usepackage{epstopdf}
\usepackage{amsmath,amsthm,amssymb}
\usepackage{epsfig}
\usepackage{latexsym}
\usepackage{amssymb}
\usepackage{german}
\setlength{\parindent}{0pt}
\setlength{\parskip}{5pt plus 2pt minus 1 pt}
\topmargin  -5mm
\evensidemargin 8mm
\oddsidemargin  2mm
\textwidth  158mm
\textheight 230mm
\frenchspacing
\sloppy
\newcommand{\xvec}{\bold{x}}
\newcommand{\vvec}{\bold{v}}
\newcommand{\wvec}{\bold{w}}
\newcommand{\evec}{\bold{e}}
\newcommand{\nuvec}{\bold{\nu}}
\newcommand{\R}[1]{\mathbb{R}^{#1}}
\newcommand{\p}{$p$}
\newcommand{\norm}[2][2]{\|#2 \|_#1}
\newcommand{\diver}{\text{div}}
\newcommand{\refer}[1]{(\ref{#1})}
\newcommand{\ndof}{\text{ndof}}
\newcommand{\gauss}[2]{e^{-\frac{(\vvec #2)^2}{#1}}}
\newcommand{\lagrange}[2]{l_i(\frac{\vvec #2}{\sqrt{#1}})}


% \newcommand{\det}{\text{det}}

\begin{document}
\begin{center}
\textbf{10. \"Ubung Numerik von partiellen Differentialgleichungen - station\"are Probleme} \newline 
\textbf{11. J\"anner 2019}
\end{center}


\begin{enumerate}

\item Es seinen $A$ und $M$ SPD-Matrizen. Zeigen Sie: Die Eigenpaare von $A u = \lambda M u$ sind genau die kritischen Punkte vom Rayleigh-Quotienten
  $$
  r(u) = \frac{u^T A u }{u^T M u}
  $$
Das vorkonditionierte Gradientenverfahren zur Bestimmung des kleinsten
EW ist wie folgt: Sei $C \approx A$ ein Vorkonditionierer, und $u^0$ ein zuf\"alliger Startwert.
\begin{eqnarray*}
  w^k & := & C^{-1} \nabla r (u^k) \\
  u^{k+1} & :=& \operatorname{arg}\min_{v \in \operatorname{span} \{u^k,
    w^k \}}  r(v)
\end{eqnarray*}
Geben Sie explizite (implementierbare) Formeln f\"ur $w^k$ und $u^{k+1}$ an.   

\item Implementieren Sie das Gradientenverfahren f\"ur
  Eigenwertprobleme in NGSolve. Bestimmen
  Sie den kleinsten EW von $-\Delta u = \lambda u$ auf $\Omega =
  (0,1)^2$ mit Dirichlet-Randbedingungen $u = 0$ auf $\partial
  \Omega$. Testen Sie gute und schlechte Vorkonditionierer.


\item Implementieren Sie einen low-order/high-order Vorkonditionierer
  in NGSolve: Verwenden Sie lokale Bl\"ocke mit allen Freiheitsgraden
  in den Vertex-patches, und zus\"atzlich den globalen Teilraum der $p=1$
  Basisfunktionen. Untersuchen Sie experimentell die Abh\"angigkeit
  der Konditionszahl von Netzfeinheit und Polynomgrad f\"ur die
  Laplacegleichung am Einheitsquadrat.

\item Es sei $\Omega = (0,1)^2$ und $V = H_0^1(\Omega)$. Es sei
  $A(u,v) = \int \lambda \nabla u \, \nabla v$ mit 
  $$
  \lambda(x) = \left\{ \begin{array} {cl}
                         \lambda_a  & (x,y) \in (0.2, 0.4) \times (0.3, 0.5)
                         \\
                         \lambda_b  & (x,y) \in (0.6, 0.8) \times (0.5, 0.7)
                         \\
                         1 & \text{sonst}
                             \end{array}
                             \right.
  $$
und $f(v) = \int_\Omega v$. Testen Sie den low-order/high-order
Vorkonditionierer, und einen Vorkonditionierer mit nur lokalen
Bl\"ocken. Variieren Sie die Parameter $\lambda_a$ und $\lambda_b$ im
Bereich von mindestens $10^{-6} \ldots 10^6$. Bestimmen Sie die
Eigenwerte von $C^{-1} A$, als auch die ben\"otigten
CG-Iterationen. Vergleichen Sie mit dem Beispiel vom Skript S 106.

\end{enumerate}
\end{document}

 