\documentclass[11pt,a4paper]{report}
\usepackage{exscale,times}
\usepackage{graphicx}
\usepackage{epstopdf}
\usepackage{amsmath,amsthm,amssymb}
\usepackage{epsfig}
\usepackage{latexsym}
\usepackage{amssymb}
\usepackage{german}
\setlength{\parindent}{0pt}
\setlength{\parskip}{5pt plus 2pt minus 1 pt}
\topmargin  -5mm
\evensidemargin 8mm
\oddsidemargin  2mm
\textwidth  158mm
\textheight 230mm
\frenchspacing
\sloppy
\newcommand{\xvec}{\bold{x}}
\newcommand{\vvec}{\bold{v}}
\newcommand{\wvec}{\bold{w}}
\newcommand{\evec}{\bold{e}}
\newcommand{\nuvec}{\bold{\nu}}
\newcommand{\R}[1]{\mathbb{R}^{#1}}
\newcommand{\p}{$p$}
\newcommand{\norm}[2][2]{\|#2 \|_#1}
\newcommand{\diver}{\text{div}}
\newcommand{\refer}[1]{(\ref{#1})}
\newcommand{\ndof}{\text{ndof}}
\newcommand{\gauss}[2]{e^{-\frac{(\vvec #2)^2}{#1}}}
\newcommand{\lagrange}[2]{l_i(\frac{\vvec #2}{\sqrt{#1}})}


\begin{document}
\begin{center}
\textbf{2. \"Ubung Numerik von partiellen Differentialgleichungen - station\"are Probleme} \newline 
\textbf{19. Oktober 2018}
\end{center}
\begin{enumerate}








\item Sei $V$ ein Hilbertraum, $f$ eine stetige Linearform und $A$ eine stetige, symmetrische und elliptische Bilinearform. Sei $u$ die L\"osung des Variationsproblems
$$A(u,v) = f(v).$$
Zeigen Sie, dass f\"ur $u$
$$\sup\limits_{v\neq 0} \frac{f(v)}{\|v\|_A} = \|u\|_A$$
gilt. Das Supremum wird f\"ur $v=u$ angenommen.
\vspace{15pt}


\item Sei $V$ ein Hilbertraum und $a: V \times V \rightarrow \mathbb{R}$ eine symmetrische, elliptische (mit $\alpha_1$) und stetige (mit $\alpha_2$) Bilinearform. Des Weiteren sei $X = V \times V$, sowie
$$
B: \left\{
\begin{aligned}
 & X \times X \rightarrow \mathbb{R} \\
 &((u_1,u_2),(v_1,v_2)) \mapsto a(u_1,v_1) + a(u_1,v_2) + a(u_2,v_2)
\end{aligned}
\right.
$$
Zeigen Sie, dass $B$ elliptisch ist und geben Sie eine Elliptizit\"atskonstante an.
Sei $X_N \subset X$ ein linearer Teilraum. L\"asst sich die Elliptizit\"at von $B$ auf $X_N$ \"ubertragen?
\vspace{15pt}


\item Wie Aufgabe 2. Sei $b: V \times V \rightarrow \mathbb{R}$ eine weitere stetige (mit $\beta_2$) Bilinearform. Jetzt sei
$$
B: X \times X \rightarrow \mathbb{R}:\;((u_1,u_2),(v_1,v_2)) \mapsto a(u_1,v_1) + b(u_1,v_2) + a(u_2,v_2)
$$
Zeigen Sie die $\inf-\sup$ Stabilit\"at von $B$ (Konstante = ?).
Geben Sie m\"ogliche Teilr\"aume $X_N\subset X$ an, sodass auch das
diskrete Problem $\inf-\sup$-stabil ist. Geben Sie ein Beispiel an,
f\"ur das das Variationsproblem auf $X_N \subset X$ nicht l\"osbar ist.
\vspace{15pt}


\item Seien $X$, $Y$ Hilbertr\"aume mit gemeinsamer Vektorraum-Struktur. Auf dem Summenraum $X+Y:= \{z = x+ y: x\in X\;y\in Y \}$ sei 
$$\| z\|_{X+Y}:= \inf\limits_{z=x+y \atop x\in X,\; y\in Y} \sqrt{\|x\|_X^2 +\|y\|_Y^2}.$$
Zeigen Sie, dass die Norm $\|\cdot \|_{X+Y}$ ein Skalarprodukt
induziert. Hinweis: Parallelogrammlemma.
\vspace{15pt}



\item \label{bspfluss} Sei $\Omega =(0,1)^2$, und $\Gamma_b$, $\Gamma_r$, $\Gamma_t$ und
  $\Gamma_l$ unterer, rechter, oberer und linker Rand. L\"osen Sie
  $-\Delta u = f := x$ auf $\Omega$, homogene Dirichlet Randbedingungen
  links und rechts, und homogene Neumann Randbedingungen unten und
  oben mittels FEM ($p=1,2,3$). Berechnen Sie die W\"armefl\"usse durch linken und
  rechten Rand
$$
W_l = - \int_{\Gamma_l} \frac{\partial u}{\partial n} \qquad \text{und} \qquad
W_r = - \int_{\Gamma_r} \frac{\partial u}{\partial n}
$$
und $W_l+W_r$. Plotten Sie die Fehler f\"ur eine Folge von Netzen.

Berechnen Sie die Integrale \"uber zwei Varianten:
\begin{enumerate}
\item numerische Integration, wobei Sie $\frac{\partial u}{\partial
    x}$ in den von Ihnen gew\"ahlten Integrationspunkten auswerten
\item \"uber die {\tt Integrate( ...., BND, region\_wise=True) }
  Funktion. Dazu m\"ussen Sie zuerst $\frac{\partial u}{\partial x}$ in eine weitere
  $H^1$-gridfunction {\tt dudx} interpolieren (mittels {\tt dudx.Set(grad(u)[0])}. Warum wohl ?
\end{enumerate}

\vspace{15pt}


\item Wie Bsp \ref{bspfluss}. Zeigen Sie zun\"achst f\"ur die L\"osung
  $u$ und eine beliebige $H^1$ Funktion $w$:
$$
\int_\Omega \nabla u \nabla w - \int_\Omega f w = \int_{\Gamma_l \cup
  \Gamma_r}  \frac{\partial u}{\partial n} w
$$
Bestimmen Sie die Fl\"usse \"uber diese Gleichung. Definieren Sie dazu
eine {\tt GridFunction} w, und setzen Sie diese so dass sie links
bzw. rechts die Werte 1 bzw. 0 annimmt.

Berechnen Sie dann die linke Seite
\begin{enumerate}
\item durch Berechnung der Integrale
\item durch Matrix-Vektor Operationen 
\end{enumerate}

\end{enumerate}
\end{document}

