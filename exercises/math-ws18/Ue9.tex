\documentclass[11pt,a4paper]{report}
\usepackage{exscale,times}
\usepackage{graphicx}
\usepackage{epstopdf}
\usepackage{amsmath,amsthm,amssymb}
\usepackage{epsfig}
\usepackage{latexsym}
\usepackage{amssymb}
\usepackage{german}
\setlength{\parindent}{0pt}
\setlength{\parskip}{5pt plus 2pt minus 1 pt}
\topmargin  -5mm
\evensidemargin 8mm
\oddsidemargin  2mm
\textwidth  158mm
\textheight 230mm
\frenchspacing
\sloppy
\newcommand{\xvec}{\bold{x}}
\newcommand{\vvec}{\bold{v}}
\newcommand{\wvec}{\bold{w}}
\newcommand{\evec}{\bold{e}}
\newcommand{\nuvec}{\bold{\nu}}
\newcommand{\R}[1]{\mathbb{R}^{#1}}
\newcommand{\p}{$p$}
\newcommand{\norm}[2][2]{\|#2 \|_#1}
\newcommand{\diver}{\text{div}}
\newcommand{\refer}[1]{(\ref{#1})}
\newcommand{\ndof}{\text{ndof}}
\newcommand{\gauss}[2]{e^{-\frac{(\vvec #2)^2}{#1}}}
\newcommand{\lagrange}[2]{l_i(\frac{\vvec #2}{\sqrt{#1}})}


% \newcommand{\det}{\text{det}}

\begin{document}
\begin{center}
\textbf{9. \"Ubung Numerik von partiellen Differentialgleichungen - station\"are Probleme} \newline 
\textbf{21. Dezember 2018}
\end{center}

Als Vorbereitung gehen Sie Kapitel 2.1.1 {\em Preconditioners in NGSolve} und Kap
2.1.2. {\em Building blocks for programming preconditioners} in den
i-tutorials durch (ngsolve.org $\rightarrow$ Documentation $\rightarrow$ i-Tutorials).

\begin{enumerate}

\item Implementieren Sie das Gradientenverfahren (Skript S 95) in
  NGSolve. Als Vorlage k\"onnen Sie das CG-Verfahren  aus dem File 
  {\em krylovspace.py} in Ihrer NGSolve - Installation verwenden. 
Testen Sie es f\"ur ein Standardproblem (Poissongleichung mit
Dirichlet-Rbd). Plotten Sie die ben\"otigte Iterationszahl in Abh\"angigeit der
Gitterfeinheit, und vergleichen Sie mit dem CG-Verfahren.

\item Testen Sie den Jacobi-Vorkonditionierer f\"ur die Matrix aus der
  Bilinearform
  $$
  A(u,v) = \int_\Omega \nabla u \nabla v + \int_{\partial \Omega} r u
  v 
  $$
 auf $V = H^1(\Omega)$, $\Omega = (0,1)^2$.
Bestimmen Sie experimentell den gr\"o\ss{}ten und kleinste Eigenwert
von $C^{-1}A$ in Abh\"angigkeit von Netzfeinheit $h$ und Parameter
$r$.

\item Beweisen Sie das in \"Ubung 9.2 festgestellte Verhalten von den
  kleinsten und gr\"o\ss{}ten Eigenwerten in Abh\"angigkeit von $h$
  und $r$ (vgl Skript Thm 92).

\item Es sei $V := [H_0^1(\Omega)]^2$ mit $\Omega = (0,1)^2$. Es sei
  $$
  A( (u_1,u_2), (v_1, v_2) ) := \int \nabla u_1 \nabla v_1 + \nabla
  u_2 \nabla v_2  + \frac{1}{t^2} (u_1 - u_2) (v_1 - v_2) 
  $$
Bestimmen Sie die extremalen Eigenwert von $C^{-1} A$
f\"ur den Jacobi-Vorkonditionierer, und einen Block-Jacobi
Vorkonditionierer. Die Bl\"ocke sind jeweils $2\times2$ Bl\"ocke mit
den Freiheitsgraden in einem Knoten. Testen Sie zumindest $t =
10^{-k}, k = 0, \ldots, 3$ und $h = 2^{-l}, l = 1, \ldots, 8$.

\item Analysieren Sie das in 9.4 festgestellte Verhalten.
\end{enumerate}
\end{document}
x
  
\item Es sei $\eta := t^{-2} (w^\prime - \beta)$ die sogenannte
  Scherspannung. Zeigen Sie die Regularit\"atsabsch\"atzung:
$$
\| w \|_{H^{3+m} + t^{-2} H^{1+m}} + \| \beta \|_{H^{2+m}} + \| \eta \|_{H^m}
\le c \, \| f \|_{H^{-1+m}} \qquad \forall \, m \in  \mathbb{N}_0
$$
mit $c \neq c(t)$. Hinweis: zuerst $\eta$, dann $\beta$, zuletzt $w$.

\item {\em kommutierende Interpolationsoperatoren}. Sei $I = [a,b]$,
  und
\begin{itemize}
\item
Elementraum $V_w = P^3$, Funktionale $\Psi_w = \{ v(a),
v^\prime(a), v(b), v^\prime(b) \}$, und
\item
Elementraum $V_\beta = P^2$, Funktionale $\Psi_\beta = \{ v(a), v(b),
\int_a^b v(s) ds \}$.
\end{itemize}
Zeigen Sie f\"ur die zugeh\"origen Interpolationsoperatoren 
$$
I_\beta w^\prime = (I_w w)^\prime
$$

\item Es wird nun $(w,\beta)$ mit FEM approximiert, wobei $w_h$ im $C^0$-stetigen finiten Elemente-Raum
  3. Ordnung, und $\beta_h$ im $C^0$-stetigen finite Elemente-Raum
  2. Ordnung liegt. Zeigen Sei folgende Fehlerabsch\"atzungen:
$$
\| \beta - \beta_h \|_{H^1} + \| w - w_h \|_{H^1}\leq c h^2 \| f \|_{H^1}
$$
mit $c \neq c(t)$.
Hinweise: Wenden Sie Cea's Lemma in der A-Norm an ($\|(w,\beta)\|_A^2 =
A(w,\beta; w, \beta)$). Zur Absch\"atzung
des Infimums w\"ahlen Sie die kommutierenden Interpolationsoperatoren
aus Bsp~2. 

\item Zeigen Sie f\"ur die integrierten Legendrepolynome aus \"U6.1:
$$
L_i(x) = c_i \big(  P_i(x) - P_{i-2} (x) \big)
$$
mit $c_i = \, ? \in \R{}$. Hinweis: Bestimmen Sie zuerst die beiden f\"uhrenden
nicht-null Koeffizienten aus der Rodrigues  - Formel. Testen Sie dann
mit Polynomien niedrigerer Ordnung ($\int_{-1}^1 ... q(x) \text{dx}$). 

\item Berechnen Sie die Element-Massen-Matrix $(\int_T \varphi_i
  \varphi_j )$ und die Element-Steifigkeits-Matrix $(\int_T \varphi_i^\prime
  \varphi_j^\prime )$ am Element $T = [-1,1]$ f\"ur die Basis
  aus \"U6.1. 




\end{enumerate}
\end{document}


\item Implementieren Sie ein High-Order Viereckselement in
  my-little-ngsolve. Gehen Sie analog zum High-Order Dreieckselement
  vor (files myHOElement.?pp, myHOFESpace.?pp).


\item Sei $u$ die L\"osung des Randwertproblems: Ges $u \in
  H^1(\Omega)$ mit $u = u_D$ auf $\Gamma_D$ sodass
$$
\int_\Omega \lambda(x) \nabla u \, \nabla v = \int_\Omega f v +
\int_{\Gamma_N} g v  \qquad \forall \, v \in H^1, v = 0 \text{ auf } \Gamma_D.
$$
Es sei $u_h$ eine Finite-Elemente N\"aherung mit $u_h = u$ auf
$\Gamma_D$. Weiters sei $\tau \in H(\diver)$ sodass $\diver \tau = - f$
und $\tau \cdot n = g$ auf $\Gamma_N$. Zeigen Sie:
$$
\int_\Omega \lambda | \nabla u - \nabla u_h |^2 \leq \int_\Omega
\lambda | \nabla u_h - \lambda^{-1} \tau |^2 
$$
Zusatzinfo: Kann so ein $\tau$ tats\"achlich berechnet werden, liefert dies eine
Fehlerabsch\"atzung ohne generische Konstante. \newline
Hinweis: Was ergibt $\int_\Omega (\nabla u - \nabla u_h) ( \lambda
\nabla u - \tau)$ ?
\end{enumerate}
\end{document}

