\documentclass[11pt,a4paper]{report}
\usepackage{exscale,times}
\usepackage{graphicx}
\usepackage{epstopdf}
\usepackage{amsmath,amsthm,amssymb}
\usepackage{epsfig}
\usepackage{latexsym}
\usepackage{amssymb}

\setlength{\parindent}{0pt}
\setlength{\parskip}{5pt plus 2pt minus 1 pt}
\topmargin  -5mm
\evensidemargin 8mm
\oddsidemargin  2mm
\textwidth  158mm
\textheight 230mm
\frenchspacing
\sloppy
\newcommand{\xvec}{\bold{x}}
\newcommand{\vvec}{\bold{v}}
\newcommand{\wvec}{\bold{w}}
\newcommand{\evec}{\bold{e}}
\newcommand{\nuvec}{\bold{\nu}}
\newcommand{\R}[1]{\mathbb{R}^{#1}}
\newcommand{\p}{$p$}
\newcommand{\norm}[2][2]{\|#2 \|_#1}
\newcommand{\diver}{\text{div}}
\newcommand{\refer}[1]{(\ref{#1})}
\newcommand{\ndof}{\text{ndof}}
\newcommand{\gauss}[2]{e^{-\frac{(\vvec #2)^2}{#1}}}
\newcommand{\lagrange}[2]{l_i(\frac{\vvec #2}{\sqrt{#1}})}


\begin{document}
\begin{center}
\textbf{1. \"Ubung Numerik von partiellen Differentialgleichungen - station\"are Probleme} \newline 
\textbf{12. Oktober 2018}
\end{center}
\begin{enumerate}



\item Sei $V$ ein Hilbertraum, 
$$A: V \times V \rightarrow \mathbb{R} $$
eine stetige, elliptische und symmetrische Bilinearform, 
$$ f: V \rightarrow \mathbb{R}$$ eine stetige Linearform auf $V$ sowie $J(v):= \frac{1}{2}A(v,v)-f(v) $. Weiters sei $V_0 \subset V$ ein linearer Teilraum, $g \in V$ und $V_g = g + V_0$. 
\\
Zeigen Sie: $J$ nimmt sein Minimum \"uber $V_g$ genau dann bei $u \in V_g$ an, wenn f\"ur $u\in V_g\,$ $A(u,v)=f(v) \quad \forall v\in V_0$ gilt. Muss dazu $V_0$ abgeschlossen sein?
\vspace{15pt}



\item Sei $V$ ein Hilbertraum sowie $A: V \rightarrow V$ ein linearer, beschr\"ankter sowie selbstadjungierter Operator. \\ Zeigen Sie ohne allgemeine Spektraltheorie zu verwenden:\\
$$ \| A\| := \sup\limits_{v \in V \atop v\neq 0} \frac{\|Av \|}{\| v \|} = \sup\limits_{v\in V\atop v\neq 0} \frac{|(Av,v)|}{\|v\|^2}$$
\vspace{15pt}



\item Sei $T_\sigma:l_2 \rightarrow l_2$ ein linearer Operator mit $(T_\sigma x)_n = \sigma_n x_n$, wobei $\sigma\in \mathbb{R}^{\mathbb{N}}$ mit $\sigma_n\rightarrow 0$.
\\ Zeigen Sie, dass $T$ kompakt ist.
\vspace{15pt}


\item L\"osen Sie die Poissongleichung $-\Delta u = 1$ mit
  Dirichlet-Randbedingungen $u=0$ auf dem ``L-shape'' Gebiet $\Omega =
  (0,2)^2 \setminus [1,2]^2$ mit NGSolve. Plotten Sie die partiellen
  Ableitungen. Sch\"atzen Sie f\"ur $p=1 \ldots 8$ den $L_2$-Fehler und den
  $H^1$-Fehler, indem Sie mit einer FEM-L\"osung h\"oherer Ordnung
  vergleichen (z.B. $p+2$). 

\end{enumerate}
\end{document}

