\documentclass[11pt,a4paper]{report}
\usepackage{exscale,times}
\usepackage{graphicx}
\usepackage{epstopdf}
\usepackage{amsmath,amsthm,amssymb}
\usepackage{epsfig}
\usepackage{latexsym}
\usepackage{amssymb}
\usepackage{german}
\setlength{\parindent}{0pt}
\setlength{\parskip}{5pt plus 2pt minus 1 pt}
\topmargin  -5mm
\evensidemargin 8mm
\oddsidemargin  2mm
\textwidth  158mm
\textheight 230mm
\frenchspacing
\sloppy
\newcommand{\xvec}{\bold{x}}
\newcommand{\vvec}{\bold{v}}
\newcommand{\wvec}{\bold{w}}
\newcommand{\evec}{\bold{e}}
\newcommand{\nuvec}{\bold{\nu}}
\newcommand{\R}[1]{\mathbb{R}^{#1}}
\newcommand{\p}{$p$}
\newcommand{\norm}[2][2]{\|#2 \|_#1}
\newcommand{\diver}{\text{div}}
\newcommand{\refer}[1]{(\ref{#1})}
\newcommand{\ndof}{\text{ndof}}
\newcommand{\gauss}[2]{e^{-\frac{(\vvec #2)^2}{#1}}}
\newcommand{\lagrange}[2]{l_i(\frac{\vvec #2}{\sqrt{#1}})}


% \newcommand{\det}{\text{det}}

\begin{document}
\begin{center}
\textbf{6. \"Ubung Numerik von partiellen Differentialgleichungen - station\"are Probleme} \newline 
\textbf{30. November 2018}
\end{center}
\begin{enumerate}
% \item

% Sei $\widehat T \subset \R{d}$ und $F : \widehat T \rightarrow T$ ein
% Diffeomorphismus. Die Piola - Transformation ${\mathcal P}$ ist wie
% folgt definiert: F\"ur $\hat v \in [L_2(\widehat T)]^d$ ist die Piola
% - Transformierte $v \in [L_2(T)]^d$
% $$
% v := {\mathcal P} (\hat v) := (\det F^\prime)^{-1} F^\prime \hat v  \circ F^{-1}
% $$
% Zeigen Sie daf\"ur (die Divergenz existiere im passenden Sinne):
% $$
% \text{div} \, v =   (\det F^\prime)^{-1}  \text{div} \hat v \circ F^{-1}
% $$
% Hinweis: a) Sie zeigen dies f\"ur glatte Funktionen und
% benutzen klassische Ableitungen, oder b) Sie verwenden die Divergenz im
% schwachen Sinn. Wir sind mit $d = 2$ zufrieden.

\item
Finden Sie die kanonische Basis  f\"ur folgendes 1D Element: 
\begin{itemize}
\item $T = [-1,1]$
\item $V_T = P^k$
\item $\Psi = \{ \psi_0 : v \mapsto v(-1), \psi_1 : v \mapsto v(1),  \psi_j : v \mapsto \int_{-1}^1 v^\prime(x) P_{j-1}(x)  dx,  \; j = 2, \ldots, k \}$ 
\end {itemize}
Verwenden Sie dazu die {\em Integrierten Legendrepolynome} $L_i(x) :=
\int_{-1}^x P_{i-1}(s) \, \text{ds}$ f\"ur $i \geq 1$.

Interpretieren Sie den kanonischen Interpolationsoperator.


\item Das Beispiel aus \"Ubung 5.5 hat eine zylinder-symmetrische
  L\"osung, d.h. $u(x,y,z) = \tilde u(r,z)$.  L\"osen Sie es unter Verwendung von Zylinderkoordinaten.
Bestimmen Sie dazu eine Variationsformulierung auf zwei Arten:
\begin{enumerate}
\item starten Sie vom Laplace-Operator in Zylinderkoordinaten, und
  leiten dazu eine symmetrische VF her
\item setzen Sie in die 3D VF rotationssymmetrische Ansatz- und
  Testfunktionen ein.
\end{enumerate}
Was ist der nat\"urliche Hilbertraum ?
L\"osen Sie das Beispiel mit FEM, und vergleichen Sie die Ergebnisse mit den Rechnungen von \"U 5.5.

\item Zeigen oder widerlegen Sie folgende Spur-Absch\"atzung:
$$
|u(0)|^2 \leq c \, \int_0^1 x \, u^\prime(x)^2 \, \text{dx} + \int_0^1
x \, u(x)^2 \, \text{dx}
$$
Diskutieren Sie m\"ogliche Randbedingungen der schwachen Formulierung
am Symmetrierand bei Verwendung von Zylinderkoordinaten.


\item 
Implementieren Sie ein Dreieckselement 3. Ordnung in
MyLittle-NGSolve. Verwenden Sie die Knotenbasis zu Punktauswertung in
$\{ (i/3, j/3) : i \geq 0, j \geq 0, i+j \leq 3 \}$. Implementieren
Sie auch das dazupassende Linienelement.
L\"osen Sie damit eine Poissongleichung.
Visualisieren Sie die Basisfunktionen.  
Hinweis: Achten Sie auf die Nummerierung in GetDofNrs.

  
\end{enumerate}
\end{document}

