\documentclass[11pt,a4paper]{report}
\usepackage{exscale,times}
\usepackage{graphicx}
\usepackage{epstopdf}
\usepackage{amsmath,amsthm,amssymb}
\usepackage{epsfig}
\usepackage{latexsym}
\usepackage{amssymb}
\usepackage{german}
\setlength{\parindent}{0pt}
\setlength{\parskip}{5pt plus 2pt minus 1 pt}
\topmargin  -5mm
\evensidemargin 8mm
\oddsidemargin  2mm
\textwidth  158mm
\textheight 230mm
\frenchspacing
\sloppy
\newcommand{\xvec}{\bold{x}}
\newcommand{\vvec}{\bold{v}}
\newcommand{\wvec}{\bold{w}}
\newcommand{\evec}{\bold{e}}
\newcommand{\nuvec}{\bold{\nu}}
\newcommand{\R}[1]{\mathbb{R}^{#1}}
\newcommand{\p}{$p$}
\newcommand{\norm}[2][2]{\|#2 \|_#1}
\newcommand{\diver}{\text{div}}
\newcommand{\refer}[1]{(\ref{#1})}
\newcommand{\ndof}{\text{ndof}}
\newcommand{\gauss}[2]{e^{-\frac{(\vvec #2)^2}{#1}}}
\newcommand{\lagrange}[2]{l_i(\frac{\vvec #2}{\sqrt{#1}})}


% \newcommand{\det}{\text{det}}

\begin{document}
\begin{center}
\textbf{7. \"Ubung Numerik von partiellen Differentialgleichungen - station\"are Probleme} \newline 
\textbf{7. Dezember 2015}
\end{center}
\begin{enumerate}

\item
Sch\"atzen Sie den $L_2$-Fehler bei der nicht-konformen $P^1$ Methode
ab. \newline
Hinweis: Verbinden Sie dazu Thm 71 (Aubin-Nitsche) und Lemma 79
(2. Lemma von Strang). 

\item
  Sei $V_{h0}^{nc}$ der nicht-konforme $P^1$ Raum (Skript Seite 76 ff)
 mit $0$ Randwerten in den Kantenmittelpunkten auf $\partial \Omega$.
Beweisen Sie die diskrete Friedrichs-Ungleichung
 $$
 \| v_h \|_{L_2} \leq  c \, \| v_h \|_h  \quad \forall v_h \in V_{h0}^{nc}.
 $$
Hinweis: Definieren Sie einen Mittelungsoperator $S : V_h^{nc}
\rightarrow V_h$ in den konformen $P^1$ Raum $V_h$ per Mittelung in
den Eckpunkten.
Zeigen Sie daf\"ur
$$
\tfrac{1}{h} \| v_h - S v_h \|_{L_2} + \| \nabla S v_h \| \leq c \, \|
v_h \|_h
\qquad \forall \, v_h \in V_h^{nc}
$$


\item {\em Goal-driven adaptivity}. Implementieren Sie einen
  Fehlersch\"atzer  f\"ur lineare Funktionale (Skript S67). Verwenden
  Sie als Vorlage das Tutorial {\em adaptivity.py}. 

  Testen Sie folgende Beispiele: $\Omega = (0,1)^2$, $A(u,v) = \int
  \nabla u \nabla v$, $f(v) = \int_{[0.2,0.3] \times [0.45,0.55]}  100
  \, v$ und die Funktionale
  \begin{itemize}
  \item $b_1(u) = 100 \int_{[0.7,0.8] \times [0.45,0.55]}  u$
  \item $b_2(u) = 10 \int_{\{0.75\} \times [0.45,0.55]} u$
    \item $b_3(u) = u(0.75, 0.5)$
  \end{itemize}

Erstellen Sie Konvergenzplots f\"ur den Fehler im
Zielfunktional. Vergleichen Sie mit adaptiver Verfeinerung f\"ur die Energienorm.
  
\end{enumerate}
\end{document}