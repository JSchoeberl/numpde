\documentclass[11pt,a4paper]{report}
\usepackage{exscale,times}
\usepackage{graphicx}
\usepackage{epstopdf}
\usepackage{amsmath,amsthm,amssymb}
\usepackage{epsfig}
\usepackage{latexsym}
\usepackage{amssymb}
\usepackage{german}
\setlength{\parindent}{0pt}
\setlength{\parskip}{5pt plus 2pt minus 1 pt}
\topmargin  -5mm
\evensidemargin 8mm
\oddsidemargin  2mm
\textwidth  158mm
\textheight 230mm
\frenchspacing
\sloppy
\newcommand{\xvec}{\bold{x}}
\newcommand{\vvec}{\bold{v}}
\newcommand{\wvec}{\bold{w}}
\newcommand{\evec}{\bold{e}}
\newcommand{\nuvec}{\bold{\nu}}
\newcommand{\R}[1]{\mathbb{R}^{#1}}
\newcommand{\p}{$p$}
\newcommand{\norm}[2][2]{\|#2 \|_#1}
\newcommand{\diver}{\text{div}}
\newcommand{\refer}[1]{(\ref{#1})}
\newcommand{\ndof}{\text{ndof}}
\newcommand{\gauss}[2]{e^{-\frac{(\vvec #2)^2}{#1}}}
\newcommand{\lagrange}[2]{l_i(\frac{\vvec #2}{\sqrt{#1}})}


% \newcommand{\det}{\text{det}}

\begin{document}
\begin{center}
\textbf{7. \"Ubung Numerik von partiellen Differentialgleichungen - station\"are Probleme} \newline 
\textbf{11. Dezember 2015}
\end{center}
\begin{enumerate}

\item
Bestimmen Sie die Basis f\"ur das Raviart Thomas Element (niedrigster
Ordnung) am Einheitsdreieck $[(0,0), (1,0), (0,1)]$. Es ist $V_T = \{
(a_x, a_y) + b\, (x,y) : a_x, a_y, b \in \R{} \}$, und $\Psi = \{ \psi_E
: v \mapsto \int_E v \cdot n \, \text{ds}  \; \forall \, \text{Kanten
} E \}$. Visualisieren Sie die Basisfunktionen geeignet.


\item Zeigen Sie f\"ur die integrierten Legendrepolynome aus \"U6.2:
$$
L_i(x) = c_i \big(  P_i(x) - P_{i-2} (x) \big)
$$
mit $c_i = \, ? \in \R{}$. Hinweis: Bestimmen Sie zuerst die beiden f\"uhrenden
nicht-null Koeffizienten aus der Rodrigues  - Formel. Testen Sie dann
mit Polynomien niedrigerer Ordnung ($\int_{-1}^1 ... q(x) \text{dx}$). 

\item Berechnen Sie die Element-Massen-Matrix $(\int_T \varphi_i
  \varphi_j )$ und die Element-Steifigkeits-Matrix $(\int_T \varphi_i^\prime
  \varphi_j^\prime )$ am Element $T = [-1,1]$ f\"ur die Basis
  aus \"U6.2. 


\item 
Geben Sie eine Basis f\"ur ein allgemeines $H^1$-Dreieck mit Elementraum $P^k$ an.
Die Randwerte auf einer Kante $E$ sollen mit der Basis aus \"U6.2
  \"ubereinstimmen (Sie k\"onnen die Skalierung anpassen). 
Berechnen sie $\text{dim} V_T$.
Wie viele Basisfunktionen sind den Eckpunkten,
den einzelnen  Kanten, und dem Inneren des Dreiecks zugeordnet ?
Zusatzinfo: Das sollte eine hierarchische Basis ergeben.

Hinweis: vgl. Dubiner-Basis. Hilfreich sind jetzt skalierte
integrierte Legendrepolynome $L_n\big(
\frac{\lambda_\alpha-\lambda_\beta}{\lambda_\alpha+\lambda_\beta}\big)
(\lambda_\alpha+\lambda_\beta)^n$. Wie sehen diese auf den 3 Kanten
$E_{\alpha \beta}$, 
$E_{\alpha \gamma}$, 
$E_{\beta \gamma}$ aus ?


\item 
Implementieren Sie ein Dreieckselement 3. Ordnung in
MyLittle-NGSolve. Verwenden Sie die Knotenbasis zu Punktauswertung in
$\{ (i/3, j/3) : i \geq 0, j \geq 0, i+j \leq 3 \}$. Implementieren
Sie auch das dazupassende Linienelement.
L\"osen Sie damit eine Poissongleichung.
Visualisieren Sie die Basisfunktionen.  
Hinweis: Achten Sie auf die Nummerierung in GetDofNrs.


\end{enumerate}
\end{document}