\documentclass[11pt,a4paper]{report}
\usepackage{exscale,times}
\usepackage{graphicx}
\usepackage{epstopdf}
\usepackage{amsmath,amsthm,amssymb}
\usepackage{epsfig}
\usepackage{latexsym}
\usepackage{amssymb}
\usepackage{german}
\setlength{\parindent}{0pt}
\setlength{\parskip}{5pt plus 2pt minus 1 pt}
\topmargin  -5mm
\evensidemargin 8mm
\oddsidemargin  2mm
\textwidth  158mm
\textheight 230mm
\frenchspacing
\sloppy
\newcommand{\xvec}{\bold{x}}
\newcommand{\vvec}{\bold{v}}
\newcommand{\wvec}{\bold{w}}
\newcommand{\evec}{\bold{e}}
\newcommand{\nuvec}{\bold{\nu}}
\newcommand{\R}[1]{\mathbb{R}^{#1}}
\newcommand{\p}{$p$}
\newcommand{\norm}[2][2]{\|#2 \|_#1}
\newcommand{\diver}{\text{div}}
\newcommand{\refer}[1]{(\ref{#1})}
\newcommand{\ndof}{\text{ndof}}
\newcommand{\gauss}[2]{e^{-\frac{(\vvec #2)^2}{#1}}}
\newcommand{\lagrange}[2]{l_i(\frac{\vvec #2}{\sqrt{#1}})}
\newcommand\restr[2]{{% we make the whole thing an ordinary symbol
  \left.\kern-\nulldelimiterspace % automatically resize the bar with \right
  #1 % the function
  \vphantom{\big|} % pretend it's a little taller at normal size
  \right|_{#2} % this is the delimiter
  }}


% \newcommand{\det}{\text{det}}

\begin{document}
\begin{center}
\textbf{9. \"Ubung Numerik von partiellen Differentialgleichungen - station\"are Probleme} \newline 
\textbf{15. J\"anner 2016}
\end{center}

\begin{enumerate}


\item Auf dem Gebiet $\omega \subset \mathbb{R}^d$ gelte 
  $\|u-\overline{u}^\omega\|_{L_2(\omega)} \leq C_P \|\nabla u \|
  _{L_2(\omega)}\;\;\forall\, u \in H^1$.  (P nach Poincar\'e)
Zeigen Sie, dass dann auch 
$$\inf\limits_{q\in P^1} \|u-q \|_{L_2(\omega)} \leq C_P^2 \|\nabla^2
u \|_{L_2(\omega)} \;\quad \forall\, u \in H^2(\omega)$$
Hinweis: W"ahle $q$ so, dass $\int_\omega q = \int_\omega u$ und
$\int_\omega \nabla q = \int_\omega \nabla u$.

Zusatzinfo: Auf einem Vertex Patch gilt $C_P \simeq h$.
\vspace{10pt}

\item Mit den Voraussetzungen aus Beispiel 1 zeigen Sie, dass
$$\inf\limits_{q\in P^k} \|u-q \| \leq C_P^{k+1} | u |_{H^{k+1}(\omega)}\;\quad \forall\, u\in H^{k+1}(\omega), $$
wobei jetzt f"ur $| .  |_{H^k}$ gemischte Ableitungen mehrfach gez"ahlt werden, d.h. $|u|^2_{H^k}:=\sum\limits_{i=1}^d |\frac{\partial u}{\partial x_i}|^2_{H^{k-1}}$ \\
Hinweis: Definiere $\Pi_k: H^k\rightarrow P^k$ so, dass f"ur alle Multiindizes $\alpha$ mit $|\alpha|\leq k$ gilt: $\int_\omega \frac{\partial^{|\alpha|}}{\partial x^{\alpha}}\Pi_ku$=$\int_\omega \frac{\partial^{|\alpha|}}{\partial x^{\alpha}}u$. Daraus folgert man, dass $\frac{\partial}{\partial x_i}(\Pi_ku) = \Pi_{k-1}(\frac{\partial}{\partial x_i}u)$ gilt.
\vspace{10pt}

\item Sei $V_h$ der Finite Elemente Teilraum 1. Ordnung von
  $H^1(\Omega)$. Wir betrachten die folgenden Clement-artigen
  Quasi-Interpolationsoperatoren ($\omega_V\ldots$ Vertex Patch, $\omega_T\ldots$ Element Patch):
$$\Pi_hu = \sum\limits_{V \atop \text{Vertices}}\tilde{\Psi}_V(u) \varphi_V $$
\begin{itemize}
\item[a)] $\tilde{\Psi}_V(u) = \frac{1}{|\omega_V|}\int_{\omega_V}u$					
\item[b)] $\tilde{\Psi}_V(u) = \frac{1}{\int_{\omega_V}\varphi_V}\int_{\omega_V}\varphi_V u$
\item[c)] $\tilde{\Psi}_V(u) = (Pu)(V)$, wobei $P$ die $L_2(\omega_V)$- orthogonale Projektion von $L_2(\omega_V)\rightarrow P^1(\omega_V)$
\item[d)] $\tilde{\Psi}_V(u) = (Pu)(V)$, wobei $P$ die $L_2(\omega_V)$- orthogonale Projektion von $L_2(\omega_V)\rightarrow \restr{V_h}{\omega_V}$
\end{itemize}
Zeigen Sie, dass $\|\Pi_h u \|_{L_2(T)} \leq c\|u\|_{L_2(\omega_T)}$
gilt, wobei $c$ nur von der Form der Dreiecke abh"angt.
Zeigen Sie au"serdem, dass $\Pi_hu = u\;\;\forall\,u\in P^0(\omega_T)$ f"ur die Operatoren aus a) bzw. b). Analog dazu zeigen Sie f"ur die Operatoren aus c) und d), dass $\Pi_hu = u\;\;\forall\,u\in P^1(\omega_T)$.

\item Zeigen Sie, dass $\Pi_h$ aus dem vorigen Beispiel b)
  selbstadjungiert ist, sowie dass $\Pi_h$ aus Aufgabe d) ein Projektor auf $V_h$ ist.



\item Wiederholen Sie Bsp 3.6. Berechnen Sie zuerst einen
  \"aquilibrierten Fluss (d.h. $\sigma_{eq}$ sodass $\operatorname{div}
  \sigma_{eq} = -f$), und berechnen daraus die Fl\"usse durch die
  einzelnen Randst\"ucke, und deren Summe.

Verwenden Sie dazu den Code aus MyLittleNGSolve/equilibrate. 
Eingespielt am 4. J\"anner. Programm equilibrate\_simle.cpp
funktioniert mit \"alterem ngsolve. Alternativ ist equilibrate.cpp,
das ben\"otigt aber ein ngsolve update.

\item Sei $\Omega = (0,1)^2$, $u$ die Greensche Funktion zum Punkt
  $(0.3, 0.5)$, d.h. $-\Delta u = \delta_{(0.3,0.5)}$ mit homogenen Dirichlet Randbedingungen. Berechnen Sie
  $u(0.7, 0.5)$ m\"oglichst genau mittels FEM. 

Stellen Sie dazu das Punktauswertefunktional als Linearform $b(\cdot)$ dar, und
berechnen Sie $b(u)$ \"uber das Skalarprodukt der beiden Vektoren.

Da  Punktauswertung (in den beiden Linearformen $f(\cdot)$ und $b(\cdot)$) nicht in $H^{-1}$
liegt, ersetzen sie diese durch Mittelung \"uber Kreise mit $r = 0.01$.

Verwenden Sie adaptive Netzverfeinerung (siehe Python-Beispiel {\em
  adaptive.py}). Vergleichen Sie die generierten Netze und die
erreichte Genauigkeit f\"ur einen Energie-Fehlersch\"atzer und
einen Ziel-basierten Fehlersch\"atzer.
\end{enumerate}
\end{document}

