\documentclass[11pt,a4paper]{report}
\usepackage{exscale,times}
\usepackage{graphicx}
\usepackage{epstopdf}
\usepackage{amsmath,amsthm,amssymb}
\usepackage{epsfig}
\usepackage{latexsym}
\usepackage{amssymb}
\usepackage{german}
\setlength{\parindent}{0pt}
\setlength{\parskip}{5pt plus 2pt minus 1 pt}
\topmargin  -5mm
\evensidemargin 8mm
\oddsidemargin  2mm
\textwidth  158mm
\textheight 230mm
\frenchspacing
\sloppy
\newcommand{\xvec}{\bold{x}}
\newcommand{\vvec}{\bold{v}}
\newcommand{\wvec}{\bold{w}}
\newcommand{\evec}{\bold{e}}
\newcommand{\nuvec}{\bold{\nu}}
\newcommand{\R}[1]{\mathbb{R}^{#1}}
\newcommand{\p}{$p$}
\newcommand{\norm}[2][2]{\|#2 \|_#1}
\newcommand{\diver}{\text{div}}
\newcommand{\refer}[1]{(\ref{#1})}
\newcommand{\ndof}{\text{ndof}}
\newcommand{\gauss}[2]{e^{-\frac{(\vvec #2)^2}{#1}}}
\newcommand{\lagrange}[2]{l_i(\frac{\vvec #2}{\sqrt{#1}})}


\begin{document}
\begin{center}
\textbf{5. \"Ubung Numerik von partiellen Differentialgleichungen - station\"are Probleme} \newline 
\textbf{27. November 2015}
\end{center}
\begin{enumerate}
\item
Seien $V_1 \subset V_0$ geeignet f\"ur Banach-Raum Interpolation. Beweisen Sie
$$
\| u \|_{s} \leq c \|u \|_0^{1-s} \|u \|_1^s \qquad \forall \, u \in V_1
$$
Hinweis: Integral \"uber K-Funktional aufspalten, und die beiden
trivialen Absch\"atzungen f\"ur $K(t,u)$ verwenden.

\item
Zeigen Sie f\"ur $\Omega$ Lipschitz: $H_0^1(\Omega) = [L_2(\Omega),
H_0^2(\Omega)]_{1/2}$.

Hinweis: \"Ahnlich zur Vorlesung, jetzt ist $E$ trivial und $R$
nicht. Geeignete Fortsetzungsoperatoren d\"urfen Sie ohne Beweis verwenden. 

\item
Finden Sie die kanonische \glqq Knotenbasis \grqq  f\"ur folgendes 1D Element: 
\begin{itemize}
\item $T = [-1,1]$
\item $V_T = P^k$
\item $\Psi = \{ \psi_j : v \mapsto \int_{-1}^1 v(x) P_j(x)  dx,  \; j = 0, \ldots, k \}$ 
\end {itemize}
Hier sind $P_j$ Legendrepolynome.


\item Finden Sie die Knotenbasis f\"ur folgendes 2D Element:
\begin{itemize}
\item Dreieck $T = [V_a, V_b, V_c]$
\item $V_T = P^2$
\item $\Psi = \{ \psi_j : v \mapsto v(V_j), j \in \{ a, b, c \} \}
  \cup \{ \psi_E : v \mapsto \int_E \frac{\partial v}{\partial \tau} s
  ds
  , E \in \{ E_{ab}, E_{ac}, E_{bc} \}\}$
\end{itemize}
Die Kante $E_{ij}$ habe Anfangspunkt $V_i$ und Endpunkt $V_j$, und
sei mit $s \in [-1,1]$ linear parametrisiert, $\tau$ sei der
Einheits-Tangentialvektor.

Hinweis: Kurvenintegral einsetzen und ausrechnen. Berechnen Sie
$\nabla \lambda_i \cdot (V_i - V_j)$

\item
Die $xy$ Ebene habe elektrisches Potential 0 Volt, und eine Metallkugel
mit Mittelpunkt $(0,0,1)$ Meter und Radius 10 cm habe Potential 1000
Volt. Welche Energie $E$ steckt im freien Raum $\R{} \times \R{} \times
\R{+} \setminus Kugel$ ? Das Potential $u$ ergibt sich dort aus der
Poissongleichung mit den vorgegebenen Dirichletrandwerten. Die Energie
ist $E = \frac12\int
\epsilon_0 | \nabla u |^2 \, dx$, mit der Permittivit\"at von Vakuum
$\epsilon_0 = 8.854 \cdot 10^{-12} \frac{As} {Vm}$. L\"osen Sie das Problem mit FEM. Schneiden
Sie dazu das Gebiet mit dem Zylinder $\{ x^2 + y^2 < R^2, z < R \}$, und
setzen am k\"unstlichen Rand a) homogene Dirichlet, oder b) homogene
Neumann Randbedingungen, und berechnen Sie $E_D(R)$ und $E_N(R)$. 
Zeigen Sie: $E_D$ ist monoton fallend, $E_N$ monoton wachsend, und $E$
liegt dazwischen.

Modellierung einer 3D Geometrie entnehmen Sie dem Beispiel 
{\it Symbolic definition of bilinear/linear forms: magnetic field},
und der Dokumentation {\it Define and mesh 3D geometries}.


\item (bis 4. Dez.) Erweitern Sie MyLittleNGSolve um bilineare
  Viereckselemente. 
Teil 1 bis 27. Nov: Compilieren Sie MyLittleNGSolve, und f\"uhren Sie
simple.py aus.  Informationen dazu finden Sie auf der ngsolve-docu Seite.

\end{enumerate}
\end{document}

