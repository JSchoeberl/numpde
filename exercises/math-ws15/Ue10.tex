\documentclass[11pt,a4paper]{report}
\usepackage{exscale,times}
\usepackage{graphicx}
\usepackage{epstopdf}
\usepackage{amsmath,amsthm,amssymb}
\usepackage{epsfig}
\usepackage{latexsym}
\usepackage{amssymb}
\usepackage{german}
\setlength{\parindent}{0pt}
\setlength{\parskip}{5pt plus 2pt minus 1 pt}
\topmargin  -5mm
\evensidemargin 8mm
\oddsidemargin  2mm
\textwidth  158mm
\textheight 230mm
\frenchspacing
\sloppy
\newcommand{\xvec}{\bold{x}}
\newcommand{\vvec}{\bold{v}}
\newcommand{\wvec}{\bold{w}}
\newcommand{\evec}{\bold{e}}
\newcommand{\nuvec}{\bold{\nu}}
\newcommand{\R}[1]{\mathbb{R}^{#1}}
\newcommand{\p}{$p$}
\newcommand{\norm}[2][2]{\|#2 \|_#1}
\newcommand{\diver}{\text{div}}
\newcommand{\refer}[1]{(\ref{#1})}
\newcommand{\ndof}{\text{ndof}}
\newcommand{\gauss}[2]{e^{-\frac{(\vvec #2)^2}{#1}}}
\newcommand{\lagrange}[2]{l_i(\frac{\vvec #2}{\sqrt{#1}})}
\newcommand\restr[2]{{% we make the whole thing an ordinary symbol
  \left.\kern-\nulldelimiterspace % automatically resize the bar with \right
  #1 % the function
  \vphantom{\big|} % pretend it's a little taller at normal size
  \right|_{#2} % this is the delimiter
  }}


% \newcommand{\det}{\text{det}}

\begin{document}
\begin{center}
\textbf{10. \"Ubung Numerik von partiellen Differentialgleichungen - station\"are Probleme} \newline 
\textbf{22. J\"anner 2016}
\end{center}

\begin{enumerate}



\item Zeigen Sie f\"ur die Operatoren aus \"U 9.3 die
  Fehlerabsch\"atzungen:
$$
\| u - \Pi_h u \|_{L_2(T)} \leq c h^k | u |_{H^k(\omega_T)}
$$
mit $k = 1$ f\"ur a) und b), bwz. $k \in \{1,2\}$ f\"ur c) und d).

Hinweis: Stetigkeit in $H^1$, \"U 9.1

\item Beweisen Sie f\"ur die nicht-konforme $P^1$ Methode eine $L_2$
  Fehlerabsch\"atzung analog zu Aubin-Nitsche:
$$
\| u - u_h \|_{L_2} \leq c h^2 \| u \|_{H^2}
$$
Dazu m\"ussen wir $H^2$-Regularit\"at f\"ur das duale Problem annehmen. 

Hinweis: Ein konformer $P^1$-Raum ist Teilraum sowohl von $V$ als auch
von $V_h$.

\item \"Aquilibrierte F\"usse f\"ur die nicht-konforme $P^1$-Methode
  sind einfach zu bestimmen ($\lambda = 1$, $f$ element-weise konstant):
$$
\sigma_{eq} := \nabla u_h - \frac{1}{2}(x-\overline x) f \qquad \text{
auf } T,
$$
mit $\overline x$ dem Schwerpunkt von $T$. Zeigen Sie
$\operatorname{div} \sigma_{eq} = -f$ auf $\Omega$. 

Hinweis: Die Normalstetigkeit von $\sigma_{eq}$ \"uber die Kante $E$ erhalten Sie aus
der FEM-Gleichung zur Basisfunktion auf $E$.

\item L\"osen Sie das Beispiel aus \"U 3.6 mit einer gemischen Methode
  f\"ur den Fluss (Skript Kap 7.2). 

Einen $H(\operatorname{div})$ FEM - Raum erhalten Sie in ngspy mit {\tt V =
  HDiv(mesh,order=xx)}.  W\"ahlen Sie die $H(\operatorname{div})$ Ordnung 1 h\"oher
als $L_2$ Ordnung. 

Der 0-Block rechts unten macht manchen direkten L\"osern Probleme
(Linux ohne MKL). F\"ugen Sie dann $-10^{-8} \int u v$ zur
Bilinearform hinzu.
\end{enumerate}
\end{document}

