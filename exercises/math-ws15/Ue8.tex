\documentclass[11pt,a4paper]{report}
\usepackage{exscale,times}
\usepackage{graphicx}
\usepackage{epstopdf}
\usepackage{amsmath,amsthm,amssymb}
\usepackage{epsfig}
\usepackage{latexsym}
\usepackage{amssymb}
\usepackage{german}
\setlength{\parindent}{0pt}
\setlength{\parskip}{5pt plus 2pt minus 1 pt}
\topmargin  -5mm
\evensidemargin 8mm
\oddsidemargin  2mm
\textwidth  158mm
\textheight 230mm
\frenchspacing
\sloppy
\newcommand{\xvec}{\bold{x}}
\newcommand{\vvec}{\bold{v}}
\newcommand{\wvec}{\bold{w}}
\newcommand{\evec}{\bold{e}}
\newcommand{\nuvec}{\bold{\nu}}
\newcommand{\R}[1]{\mathbb{R}^{#1}}
\newcommand{\p}{$p$}
\newcommand{\norm}[2][2]{\|#2 \|_#1}
\newcommand{\diver}{\text{div}}
\newcommand{\refer}[1]{(\ref{#1})}
\newcommand{\ndof}{\text{ndof}}
\newcommand{\gauss}[2]{e^{-\frac{(\vvec #2)^2}{#1}}}
\newcommand{\lagrange}[2]{l_i(\frac{\vvec #2}{\sqrt{#1}})}


% \newcommand{\det}{\text{det}}

\begin{document}
\begin{center}
\textbf{8. \"Ubung Numerik von partiellen Differentialgleichungen - station\"are Probleme} \newline 
\textbf{18. Dezember 2015}
\end{center}
Betrachten Sie den Timoshenko Balken aus \"U4.3. Wir zeigen in Bsp 1-3
$t$-robuste Fehlerabsch\"atzungen f\"ur finite Elemente h\"oherer Ordnung.

\begin{enumerate}

\item Es sei $\eta := t^{-2} (w^\prime - \beta)$ die sogenannte
  Scherspannung. Zeigen Sie die Regularit\"atsabsch\"atzung:
$$
\| w \|_{H^{3+m} + t^{-2} H^{1+m}} + \| \beta \|_{H^{2+m}} + \| \eta \|_{H^m}
\le c \, \| f \|_{H^{-1+m}} \qquad \forall \, m \in  \mathbb{N}_0
$$
mit $c \neq c(t)$. Hinweis: zuerst $\eta$, dann $\beta$, zuletzt $w$.

\item {\em kommutierende Interpolationsoperatoren}. Sei $I = [a,b]$,
  und
\begin{itemize}
\item
Elementraum $V_w = P^3$, Funktionale $\Psi_w = \{ v(a),
v^\prime(a), v(b), v^\prime(b) \}$, und
\item
Elementraum $V_\beta = P^2$, Funktionale $\Psi_\beta = \{ v(a), v(b),
\int_a^b v(s) ds \}$.
\end{itemize}
Zeigen Sie f\"ur die zugeh\"origen Interpolationsoperatoren 
$$
I_\beta w^\prime = (I_w w)^\prime
$$

\item Es wird nun $(w,\beta)$ mit FEM approximiert, wobei $w_h$ im $C^0$-stetigen finiten Elemente-Raum
  3. Ordnung, und $\beta_h$ im $C^0$-stetigen finite Elemente-Raum
  2. Ordnung liegt. Zeigen Sei folgende Fehlerabsch\"atzungen:
$$
\| \beta - \beta_h \|_{H^1} + \| w - w_h \|_{H^1}\leq c h^2 \| f \|_{H^1}
$$
mit $c \neq c(t)$.
Hinweise: Wenden Sie Cea's Lemma in der A-Norm an ($\|(w,\beta)\|_A^2 =
A(w,\beta; w, \beta)$). Zur Absch\"atzung
des Infimums w\"ahlen Sie die kommutierenden Interpolationsoperatoren
aus Bsp~2. 

\item Implementieren Sie ein High-Order Viereckselement in
  my-little-ngsolve. Gehen Sie analog zum High-Order Dreieckselement
  vor (files myHOElement.?pp, myHOFESpace.?pp).


\item Sei $u$ die L\"osung des Randwertproblems: Ges $u \in
  H^1(\Omega)$ mit $u = u_D$ auf $\Gamma_D$ sodass
$$
\int_\Omega \lambda(x) \nabla u \, \nabla v = \int_\Omega f v +
\int_{\Gamma_N} g v  \qquad \forall \, v \in H^1, v = 0 \text{ auf } \Gamma_D.
$$
Es sei $u_h$ eine Finite-Elemente N\"aherung mit $u_h = u$ auf
$\Gamma_D$. Weiters sei $\tau \in H(\diver)$ sodass $\diver \tau = - f$
und $\tau \cdot n = g$ auf $\Gamma_N$. Zeigen Sie:
$$
\int_\Omega \lambda | \nabla u - \nabla u_h |^2 \leq \int_\Omega
\lambda | \nabla u_h - \lambda^{-1} \tau |^2 
$$
Zusatzinfo: Kann so ein $\tau$ tats\"achlich berechnet werden, liefert dies eine
Fehlerabsch\"atzung ohne generische Konstante. \newline
Hinweis: Was ergibt $\int_\Omega (\nabla u - \nabla u_h) ( \lambda
\nabla u - \tau)$ ?
\end{enumerate}
\end{document}

