\documentclass[11pt,a4paper]{report}
\usepackage{exscale,times}
\usepackage{graphicx}
\usepackage{epstopdf}
\usepackage{amsmath,amsthm,amssymb}
\usepackage{epsfig}
\usepackage{latexsym}
\usepackage{amssymb}
\usepackage{german}
\setlength{\parindent}{0pt}
\setlength{\parskip}{5pt plus 2pt minus 1 pt}
\topmargin  -5mm
\evensidemargin 8mm
\oddsidemargin  2mm
\textwidth  158mm
\textheight 230mm
\frenchspacing
\sloppy
\newcommand{\xvec}{\bold{x}}
\newcommand{\vvec}{\bold{v}}
\newcommand{\wvec}{\bold{w}}
\newcommand{\evec}{\bold{e}}
\newcommand{\nuvec}{\bold{\nu}}
\newcommand{\R}[1]{\mathbb{R}^{#1}}
\newcommand{\p}{$p$}
\newcommand{\norm}[2][2]{\|#2 \|_#1}
\newcommand{\diver}{\text{div}}
\newcommand{\refer}[1]{(\ref{#1})}
\newcommand{\ndof}{\text{ndof}}
\newcommand{\gauss}[2]{e^{-\frac{(\vvec #2)^2}{#1}}}
\newcommand{\lagrange}[2]{l_i(\frac{\vvec #2}{\sqrt{#1}})}


\begin{document}
\begin{center}
\textbf{4. \"Ubung Numerik von partiellen Differentialgleichungen - station\"are Probleme} \newline 
\textbf{20. November 2015}
\end{center}
\begin{enumerate}

\item 
Zeigen Sie folgende Variante der Friedrichsungleichung:
$$
\int_0^\infty  \frac{1} {(1+x)^\alpha} u(x)^2 \, dx \leq c \, \int_0^\infty
u^\prime(x)^2 \, dx
$$
f\"ur $u \in C^1(\R{+}_0)$ mit $u^\prime \in L_2$ und $u(0) = 0$,  f\"ur $\alpha > 2$. Gilt
die Ungleichung auch f\"ur $\alpha = 2$ ?
Hinweis: HS der Integralrechung, Cauchy Schwarz, Fubini. 


\item Sei $I=(-1, 1)$ sowie $T$ das Dreieck mit Eckpunkten $(-1,0),\,(1,0),\,(0,1)$. Wir definieren den Fortsetzungsoperator $\mathcal{E}$:
$$\mathcal{E}: \left\{
							\begin{matrix}   L_2(I) \rightarrow L_2(T) \\ 
					 									u \mapsto \frac{1}{2y}\int\limits_{x-y}^{x+y}u(s) \,ds
					    \end{matrix}\right.$$
Zeigen Sie, dass f"ur $y\in (0,1)$ 
$$\int\limits_{-1+y}^{1-y} |\mathcal{E}u|^2 + |\nabla(\mathcal{E}u)|^2\,dx \leq c\, \min\{\|u\|^2_{H^1(I)}, \frac{1}{y^2}\|u\|^2_{L_2(I)}\}$$
gilt.

Zeigen Sie damit weiters
$$
\| \nabla \mathcal{E}u \|_{L_2([y-1,1-y] \times \{y\}) } \leq c \, \|u \|_{y^{-1}L_2 + H^1}.
$$
Rechts steht die Norm eines Summenraums, $s^{-1}L_2$ ist der Raum $L_2$
versehen mit der \newline Norm $s^{-1}\| \cdot \|_{L_2}$.




\item {\em Timoshenko Balken:} Gesucht sind $(w, \beta) \in V_0 \subset
  [H^1(0,1)]^2$ so dass
$$
\int_0^1 \beta^\prime \delta ^\prime + \frac{1}{t^2}\,\int_0^1
(w^\prime - \beta) (v^\prime - \delta) = \int_0^1 f v \qquad \forall \;
(v,\delta) \in V_0.
$$
Dabei ist $t \in (0,1)$ ein gegebener kleiner Parameter, und $f \in L_2$ gegeben. Untersuchen Sie L\"osbarkeit
mit Lax-Milgram (Stetigkeit, Elliptizit\"at \"uber Tartar, Kern der
Bilinearform ?). Wie
w\"achst $\frac{\alpha_2} {\alpha_1}$ mit $t \rightarrow 0$ ? Welche
Folgen hat dies f\"ur Cea's Lemma ?

Betrachten Sie f\"ur $V_0$ folgende Kombinationen von wesentlichen Randbedingungen:
\begin{enumerate}
\item $w(0) = \beta(0) = 0$
\item $w(0) = w(1) = 0$
\item $\beta(0) = \beta(1) = 0$
\end{enumerate}

Mechanische Interpretation (Zusatzinformation). $w$ beschreibt die
vertikale Deformation eines Balken unter vertikaler Last $f$, $\beta$
ist die linearisierte Verdrehung der Normalen. Die Energie im System
setzt sich aus der Biegeenergie $\frac{1}{2} \int_0^1 (\beta^\prime)^2$,
der Scherenergie $\frac{1}{t^2} \int_0^1 (w^\prime - \beta)^2$ welche die
Abweichung der Deformation der Normalen von der Normalen der
Deformation bestraft, und der Energie aufgrund der anliegenden
Belastung $-\int_0^1 f w$. Minimierung dieses Energiefunktionals f\"uhrt
auf obige Gleichung.

\item L\"osen Sie eine Reissner Mindlin Plattengleichung mit NGSolve -
  Python. Das ist das 2D Analogon zum Timoshenko-Balken.
Gesucht sind $w, \beta \in V_0 \subset [H^1]^3$ (mit $w$
  skalar und $\beta = (\beta_x, \beta_y)$ vektorwertig) sodass
$$
\int_\Omega \nabla \beta \cdot \nabla \delta + \frac{1}{t^2}
\int_\Omega (\nabla w - \beta) \cdot (\nabla v - \delta) = 
\int_\Omega f v  \qquad (v, \delta) \in V_0.
$$
Setzen Sie $\Omega = (0,1)^2$, $f = 1$ und $t = 0.1$ bzw. $t = 0.01$.
Verwenden Sie als Randbedingungen 
\begin{enumerate}
\item $w = \beta_x = \beta_y = 0$ auf $\partial \Omega$
\item $w = 0$  auf $\partial \Omega$
\item gemischte Randbedingungen Ihrer Wahl
\end{enumerate}
Visualisieren Sie $w$ und die Komponenten von $\beta$. Zeigen Sie
$\nabla w - \beta \rightarrow 0$ in $L_2$ f\"ur $t \rightarrow 0$. \newline
Welches Variationsproblem erwarten Sie f\"ur $w^0 = \lim_{t
    \rightarrow 0} w(t)$ ? (Konvergenz muss nicht bewiesen werden)

Die Definition von Produktr\"aumen und Verwendung der Komponenten
entnehmen Sie dem Beispiel {\it Navier Stokes Equation }.
\end{enumerate}
\end{document}

