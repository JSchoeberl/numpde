% \documentclass[12pt]{article}
% \usepackage{amsmath,amsthm,amssymb,a4wide}
% \usepackage[german,english]{babel}
% \usepackage{epsfig}
% \usepackage{latexsym}
% \usepackage{amssymb}
% % \usepackage{theorem}
% \usepackage{amsthm}
% % \usepackage{showkeys}

% \newcommand{\setR}{ {\mathbb R} }
% \newcommand{\setN}{ {\mathbb N} }
% \newcommand{\setZ}{ {\mathbb Z} }
% \newcommand{\eps}{\varepsilon}

% \newcommand{\beq}{\begin{equation}}
% \newcommand{\eeq}{\end{equation}}

% \newcommand{\opdiv}{\operatorname{div}}
% \newcommand{\opcurl}{\operatorname{curl}}
% \newcommand{\opdet}{\operatorname{det}}
% \newcommand{\optr}{\operatorname{tr}}
% \newcommand{\optrn}{\operatorname{tr}_n}
% \newcommand{\sfrac}[2]{ { \textstyle \frac{#1}{#2} } }

% \newcommand{\Zh}{\mathrm{Z}_h}
% \newcommand{\Ih}{\mathrm{I}_l}

% \newcommand{\leqc}{\preceq} 
% \newcommand{\geqc}{\succeq} 
% \newcommand{\eqc}{\simeq} 
% \newcommand{\ul}{\underline}

% \newtheorem{theorem}{Theorem}
% \newtheorem{definition}[theorem]{Definition}
% \newtheorem{lemma}[theorem]{Lemma}
% \newtheorem{remark}[theorem]{Remark}
% \newtheorem{example}[theorem]{Example}

% %
% %
% \setlength{\unitlength}{1cm}
% \sloppy 
% %

% \title{Proving LBB for the Stokes Equation}
% \author{Joachim Sch\"oberl}

% \begin{document}
% \maketitle
% \centerline{(supplement to lecture notes Chapter 8.1) } 

\subsection{Proving LBB for the Stokes Equation}
\subsubsection{Stability of the continuous equation}

We consider Stokes equation: find $u \in [H_0^1]^d$ and $p \in L_2^0$ such that
\begin{equation}
\begin{array}{ccccll}
\int \nabla u \cdot \nabla v & + & \int \opdiv v \, p & = & \int f v \quad & \forall \, v \in [H_0^1]^d \\[0.5em]
\int \opdiv u \, q && & = & 0 & \forall \, q \in L_2^0.
\end{array}
\end{equation}
Solvability follows from Brezzi's theorem. The only non-trivial part is the LBB condition:
$$
\sup_{v \in [H_0^1]^d} \frac { \int \opdiv v \, p}  { \| v \|_{H^1}} \geq \beta \| p \|_{L_2}
\qquad \forall \, p \in L_2^0
$$ 
We sketch two different proofs: 

{\em Proof 1: } The LBB condition becomes simple if we skip the Dirichlet conditions:
$$
\sup_{v \in [H^1]^d} \frac { \int \opdiv v \, p}  { \| v \|_{H^1}} \geq \beta \| p \|_{L_2}
\forall \, p \in L_2
$$
Take $p \in L_2(\Omega)$, extend it by $0$ to $L_2 (\setR^d)$. Now compute a right-inverse of $\opdiv$ via Fourier transform:
\begin{eqnarray*}
\hat p(\xi) & = & {\mathcal F} (p) \\
\hat u(\xi) & = & \frac{-i\xi}{|\xi|^2} \hat p(\xi) \\
u(x) & = & {\mathcal F}^{-1} (\hat u ) \\
\end{eqnarray*}
Since $\opdiv u = p$ translates to $i \xi \cdot \hat u = \hat p$, we
found a right-inverse to the divergence. Furthermore, $| u
|_{H^1(\Omega)} = \| i \xi \hat u\|_{L_2} \leqc \| \hat p \|_{L_2} =
\| p \|_{L_2}$. We restrict this $u$ to $\Omega$. 
The $L_2$-part of $\| u \|_{H^1}$ follows from the
Poincare inequality after subtracting the mean value. 

The technical part is to ensure Dirichlet - boundary conditions. One can build an extension operator ${\mathcal E}$ from $L_2(\setR^d \setminus \Omega)$ onto $\setR^d$, which commutes with the $\opdiv$-operator: $\opdiv {\mathcal E} u = {\mathcal E}^p \opdiv u$, and sets
$$
u_{final} := u - {\mathcal E} u
$$
This $u$ satisfies $u = 0$ on $\partial \Omega$. Since $\opdiv u = p = 0$ outside of $\Omega$, the correction did not change the divergence inside $\Omega$.

{\em Proof 2: } Directly construct a right-inverse for the $\opdiv$-operator via integration. We assume that $\Omega$ is star-shaped w.r.t. $\omega$, and $a\in \omega$. Extend $p$ by $0$ to $L_2(\setR^d)$:
$$
u_a(x) := -(x-a) \int_1^\infty  t^{d-1} p(a + t(x-a)) \, dt\qquad x
\neq a
$$
and $u_a(a) = 0$.
If $\int_\Omega p = 0$, then $\opdiv u_a = p$. Furthermore, $u = 0$ outside $\Omega$.
Next, we average over star-points in $\omega$:
$$
u := \frac{1}{|\omega|}  \int_\omega u_a \, da
$$
There is still $\opdiv u = p$. Now, one can show that $\| u \|_{H^1} \leqc \| p \|_{L_2}$

\subsection{Discrete LBB}

Now, we turn to the discrete system posed on $V_h \subset V$ and $Q_h \subset Q$. 
The discrete LBB condition follows from the continuous one by
construction of a Fortin operator (Lemma~99).

\subsubsection{Elements with discontinuous pressure}
The simplest pair is the non-conforming $P^1$ element, and constant pressure:
$$
V_h  = P^{1,nc}, \qquad Q_h = P^{0,dc}
$$
We have to extend the $V$-norm and forms by the sum over element-wise norms and forms.
The Fortin-operator $I_F : V \rightarrow V_h$ is defined via
$$
\int_E I_F u = \int_E u \quad \forall \, \text{edges } E
$$
It is continuous from $H^1$ to broken $H^1$ (via mapping), and satisfies
$$
\int_T \opdiv (I_F u) = \int_{\partial T} (I_F u) \cdot n = \int_{\partial T} u \cdot n = \int_T \opdiv u,
$$
and thus
$$
b_h(I_F u, q_h) = b(u, q_h) \qquad \forall \, q_h \in Q_h
$$
The error estimate follows similar as in the second Lemma by Strang:
\begin{eqnarray*}
\lefteqn{\| u - u_h\|_{H^1,nc} + \| p - p_h \|_{L_2}} \\
& \leqc &\inf_{v_h, q_h} \| u - v_h \|_{H^1,nc} + \| p - q_h\|_{L_2} 
+ \sup_{w_h} \frac{ \sum_T \int \nabla u \nabla w_h   + p \opdiv w_h - f w_h} {\| w_h \|_{H^1,nc} } \\
& \leqc & c h \, (  \| u \|_{H^2} + \| p \|_{H^1} )
\end{eqnarray*}
This convergence rate $O(h)$ is considered to be optimal for these elements.

Next we consider
$$
V_h = P^2 \qquad Q_h = P^{0,dc}
$$
We would like to define the Fortin operator similar as before:
$$
\begin{array}{rcll}
I_F u(V) & = & u(V)  \quad & \forall \, \text{vertices } V \\[0.5em]
\int_E I_F u & = & \int_E u \quad & \forall \, \text{edges } E
\end{array}
$$
But, the vertex evaluation is not allowed in $H^1$. We proceed now in two steps: First approximate $u$ in the finite element space via a Cl\'ement operator $\Pi_h$:
$$
u_h^1 := \Pi_h u,
$$
and modify this $u_h^1$ via a correction term:
$$
u_h := I_F u := u_h^1 + I_F^2 (u - u_h^1)
$$
The correction operator $I_F^2$ is defined as
$$
\begin{array}{rcll}
I_F^2 u(V) & = & 0  \quad & \forall \, \text{vertices } V, \\[0.5em]
\int_E I_F^2 u & = & \int_E u \quad & \forall \, \text{edges } E.
\end{array}
$$
It preserves edge-integrals and thus satisfies $b(u - I_F^2 u, q_h) =
0 \; \forall \, u \; \forall \, q_h$. Furthermore, it is continuous with respect to
$$
\| I_F^2 u \|_{H^1} \leqc \| u \|_{H^1} + h^{-1} \| u \|_{L_2}
$$
Thus, the combined operator $I_F$ is continuous:
\begin{eqnarray*}
\| I_F u \|_{H^1} & \leqc & \| \Pi_h u \|_{H^1} + \| I_F^2 (u-\Pi_h u) \|_{H1} \\
    & \leqc & \| u \|_{H^1} + \| u - \Pi_h u\|_{H^1} + h^{-1 } \| u - \Pi_h u \|_{L_2} \\
   & \leqc & \|  u \|_{H^1}
\end{eqnarray*}
It also satisfies the constraints:
$$
b(u - I_F u, q_h) = b(u - \Pi_h u - I_F^2 (u - \Pi_h u), q_h) = b( (Id - I_F^2)(u - \Pi_h u), q_h) = 0 
$$
Error estimates are
$$
\| u - u_h \|_{H^1} + \| p  - p_h \|_{L_2} \leqc \inf_{v_h, q_h} \| u - v_h \|_{H^1} + \| p - q_h \|_{L_2} 
= O(h)
$$
Although we approximate $u_h$ with $P^2$-elements, the bad
approximation of $p$ leads to first order convergence, only. This
element is considered to be sub-optimal.

A solution is the the pairing
$$
V_h = P^{2+} \qquad Q_h = P^{1, nc},
$$
where $P^{2+}$ is the second order space enriched with cubic bubbles:
$$
P^{2+}({\mathcal T}) = \{ v_h \in H^1 : v_{h|T} \in P^3(T), v_{h|E} \in P^2(E) \}
$$
It leads to second order convergence. Since the costs of a method depend mainly on the coupling dofs, the price for the additional bubble is low.

\subsubsection{Elements with continuous pressure}
Although the pressure $p$ is only in $L_2$, we may approximate it with
continuous elements. The so called mini-element is
$$
V_h = P^{1+} \qquad Q_h = P^{1,cont},
$$
where $P^{1+}$ is $P^1$ enriched by the cubic bubble. The continuous
pressure allows integration by parts:
$$
\int \opdiv u \, q_h = - \int u \nabla q_h
$$
The gradient of $q_h$ is element-wise constant. We thus construct a
Fortin-operator
preserving element-wise mean values. Again, we use the Cl\'ement operator
and a  correction operator:
$$
u_h := I_F u = \Pi_h u + I_F^2 (u - \Pi_h u)
$$
The correction is now defined as
$$
\begin{array}{rcll}
I_F^2 u & = & 0  \quad & \text{on} \cup E  \\
\int_T I_F^2 u & = & \int_T u \quad & \forall \, \text{elements } T.
\end{array}
$$
It satisfies $b(u - I_F^2 u, q_h) = 0 \qquad \forall \, u \, \forall
\, q_h$, and, as above:
$$
\| I_F^2 u \|_{H^1} \leqc \| u \|_{H^1} + h^{-1} \| u \|_{L_2}
$$
Thus, the combined operator is a Fortin operator.
This method is $O(h)$ convergent.
\medskip

Another (essentially) stable pair is $P^2 \times P^{1,cont}$ (the
popular Taylor Hood element). Its analysis is more involved. 
It requires the additional assumption that no two edge of one element are
on the domain boundary. Its convergence rate is $O(h^2)$.
% \end{document}