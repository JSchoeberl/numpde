\section{The weak formulation of the Poisson equation}

We are now able to give a precise definition of the weak formulation
of the Poisson problem as introduced in Section \ref{sec_intro}, and analyze the 
existence and uniqueness of a weak solution. 

Let $\Omega$ be a  bounded domain. Its boundary $\partial \Omega$ is 
decomposed as $\partial \Omega = \Gamma_D \cup \Gamma_N \cup \Gamma_R$ according to Dirichlet, Neumann and Robin boundary conditions. 

Let
\begin{itemize}
\item 
 $u_D \in H^{1/2}(\Gamma_D)$, 
\item
$f \in L_2(\Omega)$,
\item
$g \in L_2(\Gamma_N \cup \Gamma_R)$,
\item
$\alpha \in L_\infty(\Gamma_D), \alpha \geq 0$. 
\end{itemize}



Assume that there holds 
\begin{enumerate}
\item[(a)] The Dirichlet part has positive measure $| \Gamma_D | > 0$,
\item[(b)] or the Robin term has positive contribution $\int_{\Gamma_R} \alpha \, dx > 0$.
\end{enumerate}

Define the Hilbert space
$$
V := H^1(\Omega),
$$
the closed sub-space 
$$
V_0 = \{ v : \optr_{\Gamma_D} \, v = 0 \},
$$
and the linear manifold 
$$
V_D = \{ u \in V : \optr_{\Gamma_D} \, u = u_D \}.
$$

Define the bilinear form $A(.,.) : V \times V \rightarrow \setR$
$$
A(u,v) = \int_\Omega \nabla u \, \nabla v \, dx + \int_{\Gamma_R} \alpha u v \, ds 
$$
and the linear form
$$
f(v) = \int_\Omega f v \, dx + \int_{\Gamma_N \cup \Gamma_R} g v \, dx.
$$

\begin{theorem}
The weak formulation of the Poisson problem
\begin{quote}
Find $u \in V_D$ such that
\begin{equation} \label{equ_weakform}
A(u,v) = f(v) \qquad \forall \, v \in V_0
\end{equation}
\end{quote}
has a unique solution $u$.
\end{theorem}
{\em Proof:} 
The bilinear-form $A(.,.)$ and the linear-form $f(.)$ are continuous
on $V$. Tartar's theorem of equivalent norms proves that $A(.,.)$ is coercive
on $V_0$.

Since $u_D$ is in the closed range of
$\optr_{\Gamma_D}$, there exists an $\tilde u_D \in V_D$ such that
$$
\optr \, \tilde u_D = u_D \qquad \mbox{and} \qquad \| \tilde u_D \|_V \leqc \| u_D \|_{H^{1/2}(\Gamma_D)}
$$
Now, pose the problem: Find $z \in V_0$ such that
$$
A(z,v) = f(v) - A(\widetilde u_D, v) \qquad \forall \, v \in V_0.
$$

The right hand side is the evaluation of the continuous linear form
$f(.) - A(\widetilde u_D, .)$ on $V_0$. Due to Lax-Milgram, there
exists a unique solution $z$. Then, $u := \widetilde u_D + z$ solves 
(\ref{equ_weakform}). The choice of $\widetilde u_D$ is not unique, but,
the constructed $u$ is unique.
\hfill $\Box$


\subsection{Shift theorems}
%
Let us restrict to Dirichlet boundary conditions $u_D = 0$ on the whole
boundary. The variational problem: Find $u \in V_0$ such that
$$
A(u,v) = f(v) \qquad \forall \, v \in V_0
$$
is well defined for all $f \in V_0^\ast$, and, due to Lax-Milgram there holds
$$
\| u \|_{V_0} \leq c \| f \|_{V_0^\ast}.
$$
Vice versa, the bilinear-form defines the linear functional $A(u,.)$ with
norm 
$$
\| A(u,.) \|_{V_0^\ast}  \leq c \| u \|_{V_0}
$$


This dual space is called $H^{-1}$:
$$
H^{-1} := [H_0^1(\Omega)]^\ast
$$
Since $H_0^1 \subset L_2$, there is $L_2 \subset H^{-1}(\Omega)$. All 
negative spaces are defined as $H^{-s}(\Omega) := [H_0^s]^\ast(\Omega)$, 
for $s \in \setR^+$. There holds
$$
\ldots H_0^{2} \subset H_0^{1} \subset L_2 \subset H^{-1} \subset H^{-2} \ldots
$$

The solution operator of the weak formulation is smoothing twice. The
statements of shift theorem are that for $s > 0$, the solution operator 
maps also
$$
f \in H^{-1+s} \rightarrow u \in H^{1+s}
$$
with norm bounds
$$
\| u \|_{H^{1+s}} \leqc \| f \|_{H^{-1+s}}.
$$
In this case, we call the problem $H^{1+s}$ - regular.

\begin{theorem} [Shift theorem] ~
\begin{enumerate}
\item[(a)] Assume that $\Omega$ is convex. Then, the Dirichlet problem is $H^2$ regular.
\item[(b)] Let $s \geq 2$. Assume that $\partial \Omega \in C^s$. Then,
the Dirichlet problem is $H^s$-regular.
\end{enumerate}
\end{theorem}
We give a proof of (a) for the square $(0,\pi)^2$ by Fourier series. Let 
$$
V_N = \mbox{span} \{ \sin (k x) \sin (ly) : 1 \leq k,l \leq N \}
$$
For an $u = \sum_{k,l = 1}^N u_{kl} \sin (kx) \sin (ly) \in V_N$, there
holds
\begin{eqnarray*}
\| u \|_{H^2}^2 & = & \| u \|_{L_2}^2 + \| \partial_x u \|_{L_2}^2 + \| \partial_y u \|_{L_2}^2 + \| \partial _x^2 u \|_{L_2}^2 + \| \partial_x \partial_y u \|_{L_2}^2 + \| \partial_y^2 u \|^2 \\
        & \eqc & \sum_{k,l = 1}^N (1 + k^2 +l^2 + k^4 + k^2 l^2 + l^4) u_{kl}^2 \\
        & \eqc & \sum_{k,l = 1}^N (k^4 + l^4) u_{kl}^2,
\end{eqnarray*}
and, for $f = -\Delta u$,
$$
\| - \Delta u \|_{L_2}^2 = \sum_{k,l=1}^N  (k^2+l^2)^2 u_{kl}^2 
        \eqc  \sum_{k,l=1}^N  (k^4+l^4) u_{kl}^2.
$$
Thus we have $\| u \|_{H^2} \eqc \| \Delta u \|_{L_2} = \| f \|_{L_2}$ for
$u \in V_N$. The rest requires a closure argument: There is 
$\{ -\Delta v : v \in V_N \} = V_N$, and $V_N$ is dense in $L_2$. 
\hfill $\Box$

Indeed, on non-smooth non-convex domains, the $H^2$-regularity is not true.
Take the sector of the unit-disc
$$
\Omega = \{ (r \cos \phi, r \sin \phi) : 0 < r < 1, \; 0 < \phi < \omega \}
$$
with $\omega \in (\pi, 2 \pi)$. Set $\beta = \pi/\omega < 1$. The function
$$
u = (1-r^2) r^\beta sin (\phi \beta)
$$
is in $H_0^1$, and fulfills 
$\Delta u = -(4 \beta + 4) r^{\beta} sin (\phi \beta) \in L_2$.
Thus $u$ is the solution of a Dirichlet problem. But $u \not\in H^2$.


On non-convex domains one can specify the regularity in terms of weighted
Sobolev spaces. Let $\Omega$ be a polygonal domain containing $M$ vertices 
$V_i$. 
Let $\omega_i$ be the interior angle at $V_i$. If the vertex belongs to
a non-convex corner ($\omega_i > \pi$), then choose some 
$$
\beta_i \in (1 - \frac{\pi}{\omega}, 1)
$$
Define
$$
w(x) = \prod_{\mbox{\tiny non-convex} \atop \mbox{\tiny Vertices V}_i}  | x - V_i |^{\beta_i}
$$
\begin{theorem} If $f$ is such that $w f \in L_2$. Then $f \in H^{-1}$, and
the solution $u$ of the Dirichlet problem fulfills
$$
\| w D^2 u \|_{L_2} \leqc \| w f \|_{L_2}.
$$
\end{theorem}
