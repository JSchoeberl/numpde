% \documentclass[12pt]{article}
% \usepackage{amsmath,amsthm,amssymb,a4wide}
% \usepackage[german,english]{babel}
% \usepackage{epsfig}
% \usepackage{latexsym}
% \usepackage{amssymb}
% % \usepackage{theorem}
% \usepackage{amsthm}
% % \usepackage{showkeys}

% \newcommand{\setR}{ {\mathbb R} }
% \newcommand{\setN}{ {\mathbb N} }
% \newcommand{\setZ}{ {\mathbb Z} }
% \newcommand{\eps}{\varepsilon}

% \newcommand{\beq}{\begin{equation}}
% \newcommand{\eeq}{\end{equation}}

% \newcommand{\opdiv}{\operatorname{div}}
% \newcommand{\opcurl}{\operatorname{curl}}
% \newcommand{\opdet}{\operatorname{det}}
% \newcommand{\optr}{\operatorname{tr}}
% \newcommand{\optrn}{\operatorname{tr}_n}
% \newcommand{\sfrac}[2]{ { \textstyle \frac{#1}{#2} } }

% \newcommand{\Zh}{\mathrm{Z}_h}
% \newcommand{\Ih}{\mathrm{I}_l}

% \newcommand{\leqc}{\preceq} 
% \newcommand{\geqc}{\succeq} 
% \newcommand{\eqc}{\simeq} 
% \newcommand{\ul}{\underline}

% \newtheorem{theorem}{Theorem}
% \newtheorem{definition}[theorem]{Definition}
% \newtheorem{lemma}[theorem]{Lemma}
% \newtheorem{remark}[theorem]{Remark}
% \newtheorem{example}[theorem]{Example}

% %
% %
% \setlength{\unitlength}{1cm}
% \sloppy 
% %

% \title{Mixed methods for the flux : discrete norms, super-convergence and implementation techniques
% }
% \author{Joachim Sch\"oberl}

% \begin{document}
% \maketitle
% \centerline{(supplement to lecture notes Chapter 7) }

\section{Supplement on mixed methods for the flux : discrete norms, super-convergence and implementation techniques}
\subsection{Primal and dual mixed formulations}
\medskip
A mixed method for the flux can be posed either in the so called primal form: find $\sigma \in V = [L_2]^2$, $u \in H^1$ with $u = u_D$ on $\Gamma_D$
such that
$$
\begin{array}{ccccll}
\int_\Omega (a^{-1} \sigma) \cdot \tau \, dx & - & 
\int_\Omega \tau  \cdot \nabla u \, dx & = & 0 \qquad & \forall \, \tau, \\[0.5em]
-\int_\Omega \sigma \cdot \nabla v \, dx & & & = &  -\int f v \, dx  - \int_{\Gamma_N}  g v \, ds \qquad & \forall \, v, \; v = 0 \; \mbox{on} \; \Gamma_D, \\
\end{array}
$$
or in the so called dual mixed form: find $\sigma \in V = H(\opdiv)$,
$u \in L_2$ with $\sigma \cdot n = g$ on $\Gamma_N$
$$
\begin{array}{ccccll}
\int_\Omega (a^{-1} \sigma) \cdot \tau \, dx & + & 
\int_\Omega \opdiv \, \tau  \, u \, dx & = & \int_{\Gamma_D} u_D \tau _n \, ds \qquad & \forall \, \tau, \; \tau_n = 0 \; \mbox{on} \; \Gamma_N \\[0.5em]
\int_\Omega \opdiv \, \sigma \, v \, dx & & & = & - \int f v \, dx \qquad & \forall \, v.
\end{array}
$$
Both are formally equivalent: If the solutions are smooth enough for
integration by parts, both solutions are the same.
In both cases, the big-B bilinear-form is $inf-sup$ stable with respect to the corresponding norms.

\medskip

The natural discretization for the primal-mixed formulation uses
standard $H^1$-finite elements of order $k$ for $u$, and discontinuous
$L_2$ elements of order $k-1$ for $\sigma$. Here, the discrete Brezzi
conditions are trivial.  The dual one requires Raviart-Thomas (RT) or
Brezzi-Douglas-Marini (BDM) elements for $\sigma$, and $L_2$ elements
for $u$. 
This pairing delivers locally exact conservation ($\int_{\partial T}
\sigma_n = -\int_T f$). In particular this property makes the method
interesting by itself, but often this scheme is a part of a more
complex problem (e.g. Navier-Stokes equations).

\medskip

Our plan is as follows: We want to use the dual finite element method,
but analyze it in a primal - like setting. 
Since $Q_h$ is no sub-space of $H^1$, we have to use a discrete counterpart of the $H^1$-norm:
\begin{eqnarray*}
\| \tau \|_{V_h}^2 & := & \| \tau \|_{L_2}^2 \\
\| v \|_{Q_h}^2 & := & \| v \|_{H^1,h}^2 := \sum_T \| \nabla v \|_{L_2(T)}^2 + \sum_{E \subset \Omega} \tfrac{1}{h}\| [v] \|_{L_2(E)}^2 
+ \sum_{E \subset \Gamma_D} \tfrac{1}{h} \| v \|_{L_2(E)}^2
\end{eqnarray*}

The factor $\tfrac{1}{h}$ provides correct scaling: If we transform an
element patch to the reference patch, the jump term scales like the $H^1$-semi-norm. 
This norm is called discrete $H^1$-norm, or DG-norm (as it is essential for Discontinuous Galerkin methods discussed later).

There holds a discrete Friedrichs inequality
$$
\| v \|_{L_2} \leqc \| v \|_{H^1,h}.
$$

\medskip

\begin{theorem} \label{theo_mixeddiscretenorms}
The dual-mixed discrete problem satisfies Brezzi's conditons with respect to $L_2$ and discrete $H^1$-norms.
\end{theorem}
\begin{proof} The $a(.,.)$ bilinear-form is continuous and coercive on $(V_h, \| \cdot \|_{L_2})$. 
Now we show continuity of $b(.,.)$ on the finite element spaces. We integrate by parts on the elements, and rearrange boundary terms:
\begin{eqnarray*}
\lefteqn{b(\sigma_h, v_h) = \int_\Omega \opdiv \, \sigma_h \, v_h = \sum_T \int_T \opdiv \, \sigma_h \, v_h} \\
& = & \sum_T -\int_T \sigma_h \cdot \nabla v_h + \int_{\partial T} \sigma_h \cdot n \, v_h \\
& = & \sum_T - \int_T \sigma_h \cdot \nabla v_h + \sum_{E \subset \Omega} \int_E \sigma_h n_E [v] + \sum_{E \subset \Gamma_D} \int_E \sigma_h n_E v  + \sum_{E \subset \Gamma_N} \int_E \underbrace{\sigma_h n_E}_{=0} v 
\end{eqnarray*}
The jump term is defined as $[v](x) = \lim_{t \rightarrow 0^+} v(x+t
n_E) - v(x-t n_E)$. Thus, $\sigma_h n_E [v]$ is independent of
the direction of the normal vector.
Next we apply Cauchy-Schwarz, and use that $h \| \sigma_h \cdot n \|_{L_2(E)}^2 \leqc \| \sigma_h \|_{L_2(T)}$ for some $E \subset T$ (scaling and equivalence of norms on finite dimensional spaces):
\begin{eqnarray*}
b(\sigma_h, v_h) & \leq & \sum_T \| \sigma_h \|_{L_2(T)} \, \| \nabla v_h \|_{L_2(T)}  + \sum_{E \subset \Omega}  h^{1/2} \| \sigma \|_{L_2(E)} \, h^{-1/2} \| [v_h]\|_{L_2(E)}   + \sum_{E \subset \Gamma_D} ...  \\
& \leqc &  \| \sigma_h \|_{L_2(\Omega)}  \|  v_h \|_{H^1,h}
\end{eqnarray*}
The linear-forms are continous with norms $h^{-1/2} \| u_D \|_{L_2(\Gamma_D)}$ and $\| f \|_{L_2(\Omega)}$, respectively.

Finally, we show the LBB - condition: Given an $v_h \in Q_h$, we define $\sigma_h$ as follows:
\begin{eqnarray*}
\sigma_h \cdot n_E & =  & \tfrac{1}{h} [v_h] \qquad \text{on} \; E \subset \Omega \\
\sigma_h \cdot n_E & =  & \tfrac{1}{h} v_h \; \qquad \text{on} \; E \subset \Gamma_D \\
\sigma_h \cdot n_E & =  & 0 \qquad \qquad \text{on} \; E \subset \Gamma_N \\
\int_T \sigma_h \cdot q & = & -\int_T \nabla v_h \cdot q \quad \forall \, q \in [P^{k-1}]^2. 
\end{eqnarray*}
This definition mimics $\sigma = -\nabla v$. This construction is allowed by the definition of Raviart-Thomas finite elements.
Thus we get
\begin{eqnarray*}
b(\sigma_h, v_h) & = & \sum_T -\int \sigma_h \underbrace{\nabla
                       v_h}_{\in [P^{k-1}]^2} + \sum_{E \subset \Omega} \tfrac{1}{h} \| [v_h] \|_{L_2(E)}^2 + \sum_{E \subset \Gamma_D} \tfrac{1}{h} \| v_h \|_{L_2(E)}^2 \\
& = & \sum_T \int \nabla v_h \cdot  \nabla v_h + \sum_{E \subset \Omega} \tfrac{1}{h} \| [v_h] \|_{L_2(E)}^2 + \sum_{E \subset \Gamma_D} \tfrac{1}{h} \| v_h \|_{L_2(E)} \\
& = & \| v_h \|_{H^1,h}^2
\end{eqnarray*}
By scaling arguments we see that $\| \sigma_h \|_{L_2} \leqc \| v_h \|_{H^1,h}$. Thus we got $\sigma_h$ such that
$$
\frac{b(\sigma_h, v_h)} {\| \sigma_h \|_{L_2}} \geqc \| v_h \|_{H^1,h},
$$
and we have constructed the candidate for the LBB condition.
\end{proof}


\subsection{Super-convergence of the scalar}

Typically, the discretization error of mixed methods depend on best-approximation errors in all variables:
$$
\| \sigma - \sigma_h \|_{L_2} + \| u - u_h \|_{H^1,h} \leqc \inf_{\tau_h, v_h} \| \sigma - \tau_h \|_{L_2} + \| u - v_h \|_{H^1,h}
$$
By the usual Bramble-Hilbert and scaling arguments we see that (using the element-wise $L_2$-projection $P_h$):
$$
\| u - P_h u \|_{H^1,h} \leqc h^{k} \| u \|_{H^{1+k}} \qquad k \geq 0.
$$
In the lowest order case ($k = 0$) we don't get any convergence !!

But, we can show error estimates for $\sigma$ in terms of approximability for $\sigma$ only. Furthermore, we can perform a local postprocessing to
improve also the scalar variable.

\begin{theorem} There holds
$$
\| \sigma - \sigma_h \|_{L_2} + \| P_h u - u_h \|  \leqc \| \sigma - I_h \sigma \|_{L_2},
$$
where $I_h$ is the canonical RT interpolation operator satisfying the commuting diagram property $\opdiv I_h = P_h \opdiv$.
\end{theorem}
\begin{proof} As usual for a priori estimates, we apply stability of the discrete problem, and use the Galerkin orthogonality:
\begin{eqnarray}
\| I_h \sigma - \sigma_h \|_{L_2} + \| P_h u - u_h \|_{H^1,h} & \leqc & \sup_{\tau_h, v_h} \frac{B( (I_h \sigma - \sigma_h, P_h u - u_h), (\tau_h, v_h) )}  { \| \tau_h \|_{L_2} + \| v_h \|_{H^1,h}} \\
& = & 
\label{equ_invsubsuperproof}
\sup_{\tau_h, v_h} \frac{B( (I_h \sigma - \sigma, P_h u - u), (\tau_h, v_h) )}  { \| \tau_h \|_{L_2} + \| v_h \|_{H^1,h}} 
\end{eqnarray}
Now we elaborate on the terms of  $B( (I_h \sigma - \sigma, P_h u - u), (\tau_h, v_h) ) = \int a^{-1} (I_h \sigma - \sigma) \cdot \tau_h +
\int \opdiv (I_h \sigma - \sigma) v_h + \int \opdiv \tau_h (P_h u - u)$:
For the first one we use Cauchy-Schwarz, and bounds for the coefficient $a$:
$$
\int a^{-1} (I_h \sigma - \sigma) \cdot \tau_h \leqc \| \sigma - I_h \sigma \|_{L_2} \, \| \tau_h \|_{L_2}
$$
For the second one we use the commuting diagram, and orthogonality:
$$
\int \opdiv (I_h \sigma - \sigma) v_h = \int \underbrace{ (P_h - Id) \opdiv \sigma }_{\in Q_h^\bot} \underbrace{ v_h }_{\in Q_h} = 0
$$
For the third one we use that $\opdiv V_h \subset Q_h$:
$$
\int \underbrace{ \opdiv \tau_h}_{\in V_h}   \underbrace{ (P_h  u - u) }_{V_h^\bot}
$$
Thus, the right hand side of equation (\ref{equ_invsubsuperproof}) can be estimated by $\| \sigma - I_h \sigma \|_{L_2}$. Finally, an application of the
 triangle inequality proves the result.
\end{proof}

\begin{remark} This technique applies for BDM elements as well as for RT. Both satisfy $\opdiv V_h = Q_h$, and the commuting diagram.
For $RT_k$ elements, i.e. $[P^{k}]^2 \subset RT_k \subset [P^{k+1}]^2$ as well as $BDM_k$ elements, i.e. $BDM_k = [P^k]^d$
we get the error estimate
$$
\| \sigma - I_h \sigma \|_{L_2} \leqc h^{k+1} \| \sigma \|_{H^k}
$$ 
with $k \geq 0$ for RT and $k \geq 1$ for BDM.
\end{remark}

\begin{remark} The scalar variable shows super-convergence: A filtered
  error, i.e. $P_h u - u_h$ is of higer order than the error $u - u_h$
  itself: One order for RT and two orders for BDM
\end{remark}
We can apply a local post-processing to compute the scalar part with higher accuracy: We use the equation $\sigma = a \nabla u$, and the good 
error estimates for $\sigma$ and $P_h u$. We set $\widetilde Q_h := P^{k+1}$, and solve a local 
problem on every element:
$$
\min_{\tilde v_h \in \widetilde Q_h \atop \int_T v_h = \int_T \tilde v_h}  \| a \nabla
\tilde v_h - \sigma_h \|_{L_2(T),a^{-1}}^2 
$$

% \newpage
\subsection{Solution methods for the linear system}
%
The finite element discretization leads to the linear system for the
coefficient vectors (called $\sigma$ and $u$ again):
$$
\left( \begin{array}{cc}
  A & B^t \\ B & 0 
\end{array} \right) 
\left( \begin{array}{cc}
         \sigma \\ u 
\end{array} \right) 
=
\left( \begin{array}{cc}
         0 \\ -f
\end{array} \right) 
$$
This matrix is indefinite, it has $\dim V_h$ positive and $\dim Q_h$
negative eigenvalues. This causes difficulties for the linear equation
solver. 

The first possibility is a direct solver, which must (in
contrast to positive definite systems) apply Pivot strategies. 

A second possibility is block-elimination: eliminate $\sigma$ from the
first equation. The regularity of $A$ follows from $L_2$-coercivity of $a(.,.)$:
$$
\sigma = -A^{-1}B^t u 
$$
and insert it into the second equation:
$$
- B A^{-1} B^t u = -f
$$
Thanks to the LBB-condition, $B$ has full rank, and thus the Schur complement 
matrix is regular. Since $B$ is the discretization of the $\opdiv$-operator,
$B^t$ of the negative gradient, and $A$ of $a(x)^{-1} I$, the equation can be
interpreted as a discretization of 
$$
\opdiv a \nabla u = -f
$$
This approach is not feasible, since $A^{-1}$ is not a sparse matrix
anymore.

One can use extensions of the conjugate gradient (CG) method for
symmetric but indefinit matrices (e.g. MINRES). Here, preconditioners
are important. Typically, for block-systems one uses block-diagonal
preconditioners to rewrite the system as
$$
\left( \begin{array}{cc}
  \widetilde G_V^{-1} & 0 \\ 0 & \widetilde G_Q^{-1} 
\end{array} \right) 
\left( \begin{array}{cc}
  A & B^T \\ B & 0 
\end{array} \right) 
\left( \begin{array}{cc}
         \sigma \\ u 
\end{array} \right) 
=
\left( \begin{array}{cc}
  \widetilde G_V^{-1} & 0 \\ 0 & \widetilde G_Q^{-1} 
\end{array} \right) 
\left( \begin{array}{cc}
         0 \\ -f
\end{array} \right) 
$$
where $\widetilde G_V$ and $\widetilde G_Q$ are approximations to the
Gramien matrices in $V_h$ and $Q_h$:
$$
G_{V,ij} = (\varphi_i^\sigma, \varphi_j^\sigma)_V \quad \text{and}
\quad G_{Q,ij} = (\varphi_i^u, \varphi_j^u)_Q
$$
One can either choose the $H (\opdiv)$-$L_2$, or the $[L_2]^2$-$H^1$
setting, which leads to different kind of preconditioners. Here, the
later one is much simpler. This is a good motivation for considering the alternative framework.
\subsection{Hybridization}
Hybridization is a technique to derive a new variational formulation
which obtains the same solution, but its system matrix is positive
definit. For this, we break the normal-continuity of the flux
functions, and re-inforce it via extra equations. We obtain new
variables living on the element-edges (or faces in 3D).

We start from the first equation $a^{-1} \sigma - \nabla u = 0$,
multiply with element-wise discontinuous test-functions $\tau$, and
integrate by parts on the individual elements:
$$
\int_\Omega a^{-1} \sigma \tau + \sum_T \big\{  \int_T u \, \opdiv
\tau - \int_{\partial T} u \, \tau_n \big\} = 0
$$
We now introduce the new unknown variable $\hat u$ which is indeed 
the restriction of $u$ onto the mesh skeleton $u_{|\cup E}$. 

We set $V := \prod_T H(\opdiv, T)$ and $Q := L_2(\Omega) \times \prod_E
H^{1/2}(E)$, and pose the so called hybrid problem: find $\sigma \in V$ and $(u,\hat
u) \in Q$ such that
$$
\begin{array}{ccccccll}
  \int_\Omega (a^{-1} \sigma) \cdot \tau \, dx & + & 
\sum_T\int_T \opdiv \, \tau  \, u \, dx & + &
\sum_T\int_{\partial T} \hat u \, \tau_n   & = & 0 & \forall \, \tau \in V\\[0.5em]
\sum_T  \int_T \opdiv \, \sigma \, v \, dx & & & & & = & - \int f v \,
                                                        dx & \forall \, v \in L_2(\Omega) \\[0.5em]
  \sum_T\int_{\partial T} \hat v \, \sigma_n   &&&&& = & 0 
                                                         \qquad & \forall \, \hat v \in \prod_E H^{1/2}(E) .  
\end{array}
$$
The last equation can be rearranged edge by edge:
$$
\sum_{E \subset \Omega} \int_E [\sigma_n] \, \hat v +
\sum_{E \subset \partial \Omega} \int_E \sigma_n \, \hat v = 0 \qquad \forall \, \hat v \in \prod_E H^{1/2}(E) ,
$$
which implies normal-continuity of $\sigma$. Dirichlet/Neumann boundary conditions are posed now for the skeleton variable $\hat u$.

This system is discretized by discontinuous $RT/BDM$ elements for $\sigma$, piecewise polynomials on elements $T$ for $u$, 
and piecewise polynomials on edges for $\hat u$ such that the order matches with $\sigma \cdot n$. 
\begin{enumerate}
\item This discrete system is well-posed with respect to the norms $\|
  \sigma \|_V := \| \sigma \|_{L_2}$ and $\| u, \hat u \|^2 = \sum_T
  \| \nabla u \|_{L_2(T)}^2 + \tfrac{1}{h} \| u - \hat u \|_{\partial
    T}^2$. Similar proof as for Theorem~\ref{theo_mixeddiscretenorms}.
\item The components $\sigma_h$ and $u_h$ of the solution of the hybrid problem correspond to the solution of the mixed method.
\item Since the $\sigma_h$ is discontinuous across elements, the arising matrix $A$ is block-diagonal. Now it is cheap to form the Schur complement
$$
-B A^{-1} B^T \left( u \atop \hat u \right) = \left( f \atop 0 \right)
$$
This is now a system with a positive definite matrix for unknowns in the elements and on the edges. Since the matrix block for the element-variables is still block-diagonal, they can be locally eliminated, and only the skeleton variables are remaining. In the lowest order case, the matrix is the same as for 
the non-conforming $P^1$ element.
\end{enumerate}
% \end{document}