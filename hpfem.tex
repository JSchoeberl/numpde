% \documentclass[12pt]{article}
% \usepackage{amsmath,amsthm,amssymb,a4wide}
% \usepackage[german,english]{babel}
% \usepackage{epsfig}
% \usepackage{latexsym}
% \usepackage{amssymb}
% % \usepackage{theorem}
% \usepackage{amsthm}
% % \usepackage{showkeys}

% \newcommand{\setR}{ {\mathbb R} }
% \newcommand{\setN}{ {\mathbb N} }
% \newcommand{\setZ}{ {\mathbb Z} }
% \newcommand{\eps}{\varepsilon}

% \newcommand{\beq}{\begin{equation}}
% \newcommand{\eeq}{\end{equation}}

% \newcommand{\opdiv}{\operatorname{div}}
% \newcommand{\opcurl}{\operatorname{curl}}
% \newcommand{\opdet}{\operatorname{det}}
% \newcommand{\optr}{\operatorname{tr}}
% \newcommand{\optrn}{\operatorname{tr}_n}
% \newcommand{\sfrac}[2]{ { \textstyle \frac{#1}{#2} } }

% \newcommand{\Zh}{\mathrm{Z}_h}
% \newcommand{\Ih}{\mathrm{I}_l}

% \newcommand{\leqc}{\preceq} 
% \newcommand{\geqc}{\succeq} 
% \newcommand{\eqc}{\simeq} 
% \newcommand{\ul}{\underline}

% \newtheorem{theorem}{Theorem}
% \newtheorem{definition}[theorem]{Definition}
% \newtheorem{lemma}[theorem]{Lemma}
% \newtheorem{remark}[theorem]{Remark}
% \newtheorem{example}[theorem]{Example}

% %
% %
% \setlength{\unitlength}{1cm}
% \sloppy 
% %

% \title{hp - Finite Elements}
% \author{Joachim Sch\"oberl}

% \begin{document}
% % \selectlanguage{german}

% \maketitle

\section{hp - Finite Elements}

Let $V_{hp}$ be a $p$-th order finite element sub-space of $H^1$.
By scaling and Bramble-Hilbert technique one obtains the best-approxiamtion error estimate
$$
\inf_{v_{hp} \in V_{hp}} \| u - v_{hp} \|_{H^1}  \leq c h^{m-1} \| u \|_{H^m}
$$
for $m \leq p+1$. The constant $c$ depends on the order $p$. If $m$ is fixed, we do obtain reduction of the approximation error as we increase $p$. Next we develop methods to obtain so called $p$-version error estimates 
$$
\inf_{v_{hp} \in V_{hp}} \| u - v_{hp} \|_{H^1}  \leq c \left( \sfrac{h}{p} \right) ^{m-1} \| u \|_{H^m},
$$
where $c$ is independent of $h$ and $p$. This estimate proves also convergence of the $p$-version finite element method: One may fix the mesh, and increase the order $p$. 

A detailed analyis of local $H^m$ norms allows an optimal balance of
mesh-size $h$ and polynomial order $p$. This $hp$-version leads to
exponential convergence
$$
\inf_{v_{hp} \in V_{hp}} \| u - v_{hp} \|_{H^1}  \leq c e^{ -N^\alpha },
$$
where $N$ is the number of unknowns. 
 
We will prove the $p$-version estimate, but not the $hp$-result.

\subsection{Legendre Polynomials}
Orthogonal polynomials are important to construct stable basis 
functions for the $p$-FEM, as well as for error estimates.

Let $\Pi_n$ denote the space of polynomials up to order $n$. We write $\pi_n$ for 
a generic polynomial in $\Pi_n$, with a different value any time it appears.

Definition of Legendre polynomials via Rodrigues' formula:
$$
P_n(x) := \frac{1}{2^n n!} \frac{d^n}{dx^n} (x^2-1)^n.
$$
It is a polynomial of degree $n$. The first few Legendre polynomials are
\begin{eqnarray*}
P_0(x) & = & 1 \\
P_1(x) & = & x \\
P_2(x) & = & \sfrac{3}{2} x^2 - \sfrac{1}{2}
\end{eqnarray*}
$P_n$ is even if $n$ is even, and $P_n$ is odd if $n$ is odd.
Since $(x^2-1)^n = x^{2n} - n x^{2n-2} + \pi_{2n-4}$ (with proper
modification for small $n$) we have
\begin{equation}
\label{equ_leadingcoef}
P_n(x) = \frac{1}{2^n n!} \frac{ (2n)! }{ n! } x^n - \frac{n}{2^n n!} \frac{ (2n-2)!} {(n-2)!} x^{n-2} + \pi_{n-4}
\end{equation}
\begin{lemma} \label{lemma_ortho} There holds
\begin{equation} \label{equ_ortho}
\int_{-1}^1 P_n(x) P_m(x) \, dx = \frac{2}{2n+1} \delta_{n,m}.
\end{equation}
\end{lemma}
\begin{proof} W.l.o.g. let $n \leq m$. Multiple integration by parts gives
\begin{eqnarray*}
\lefteqn{ 2^{n+m} n! m! \int_{-1}^1 P_n(x) P_m(x) \, dx = 
\int_{-1}^1 \frac{d^n}{dx^n} (x^2-1)^n \frac{d^m}{dx^m} (x^2-1)^n  \, dx}
\\
& = & \int_{-1}^1 \frac{d^{n+1}}{dx^{n+1}} (x^2-1)^n \frac{d^{m-1}}{dx^{m-1}} (x^2-1)^m 
+ 
\left[ \frac{d^n}{dx^{n}} (x^2-1)^n \underbrace{ \frac{d^{m-1}}{dx^{m-1}} (x^2-1)^m}_{= 0 \text{ for } x \in \{ -1, 1\}} \right]_{-1}^1 \\
& = & \cdots \\
& = & \int_{-1}^1 \frac{d^{n+m}}{dx^{n+m}} (x^2-1)^n (x^2-1)^m \, dx
\end{eqnarray*} 
For $n < m$, the first factor of the integrand vanishes, and we have orthogonality. For $n = m$ this equals
\begin{eqnarray*}
2^{2n} (n!)^2 \| P_n \|_{L_2}^2 &= &\int_{-1}^1 (2n)! (x^2-1)^n \, dx = (2n)! \int_{-1}^1 (x-1)^n (x+1)^n \\
& = & -(2n)! \int_{-1}^1 \frac{n}{n+1} (x-1)^{n+1} (x+1)^{n-1} \\
& = & (2n)! \int_{-1}^1 \frac{ n (n-1) }{ (n+1) (n+2) } (x-1)^{n+2}(x+1)^{n-2} = ...  \\
& = & (2n)! \frac{n!} {2n (2n-1) \cdots (n+1)} \int_{-1}^1 (x-1)^{2 n} \, dx = (n!)^2 \frac{1}{2n+1} \, 2^{2n+1},
\end{eqnarray*}
which proves the scaling.
\end{proof}

Next we prove the 3-term recurrency, which can be used for efficient evaluation.
\begin{lemma} \label{lemma_threeterm} There holds
\begin{equation} \label{equ_threeterm}
(n+1) P_{n+1} (x) = (2n+1) x P_n(x) - n P_{n-1}(x).
\end{equation}
\end{lemma}
\begin{proof} Set $r(x) = (n+1) P_{n+1} (x) - (2n+1) x P_n(x) + n P_{n-1}(x)$.
Using (\ref{equ_leadingcoef}), we see that the leading coefficients cancel, 
and thus $r \in \Pi_{n-2}$. From Lemma~\ref{lemma_ortho} we get for 
any $q \in \Pi_{n-2}$
$$
\int_{-1}^1 r(x) q(x) \, dx = (n+1) \int_{-1}^1 P_{n+1} \, q - (2n+1) \int_{-1}^1 P_n \underbrace{x q}_{\in \Pi_{n-1}} + n \int_{-1}^1 P_{n-1} \, q = 0,
$$
and thus $r = 0$. 
\end{proof} 

\begin{lemma} \label{lemma_sturmliouville} Legendre polynomials satisfy the Sturm-Liouville differential equation
$$
\frac{d}{dx} \big[ (x^2-1) \frac{d}{dx} P_n(x) \big] = n (n+1) P_n(x)
$$
\end{lemma} 
\begin{proof} Both sides are in $\Pi_n$. We compare leading coefficients, for this set $P_n = a_n x^n + \pi_{n-2}$ (with $a_n =  \frac{1}{2^n n!} \frac{ (2n)! }{ n! }$).
\begin{eqnarray*}
lhs & = & \frac{d}{dx} \left[ (x^2-1) \frac{d}{dx} \left( a_n x^n + \pi_{n-2} \right) \right] \\
& = & \frac{d}{dx} \left[ (x^2-1) (a_n n x^{n-1} + \pi_{n-3}) \right] \\
& = & \frac{d}{dx} \left[ a_n n x^{n+1} + \pi_{n-1} \right] \\
& = & n (n+1) a_n x^n + \pi_{n-2}, 
\end{eqnarray*}
and we get the same leading coefficient for rhs. Furthermore, for $q \in \Pi_{n-1}$ there holds
\begin{eqnarray*}
\int_{-1}^1 lhs \, q & = &-\int_{-1}^1 (x^2-1) P_n^\prime q^\prime \, dx + \underbrace{ \left[ (x^2-1) P_n^\prime q \right]_{-1}^1 }_{= 0} \\
& = & \int_{-1}^1 P_n \underbrace{ \left( (x^2-1) q^\prime\right)^\prime}_{\in \Pi_{n-1}} \, dx - \left[ P_n (x^2-1) q^\prime \right]_{-1}^1 = 0,
\end{eqnarray*}
and the same for the rhs. Thus the identity is proven.
\end{proof}

Lemma~\ref{lemma_sturmliouville} implies that the Legendre polynomials are also orthogonal w.r.t. $(u^\prime, v^\prime)_{L_2, 1-x^2}$, i.e.
$$
\int_{-1}^1 (1-x^2) P_n^\prime P_m^\prime = n(n+1) \| P_n \|_{L_2}^2 \delta_{n,m}
$$


\subsection{Error estimate of the $L_2$ projection}
Since polynomials are dense in $L_2(-1,1)$, we get
$$
u = \sum_{n=0}^\infty a_n P_n
$$
with the generalized Fourier coefficients
$$
a_n = \frac{ (u, P_n)_{L_2} } { \| P_n \|^2_{L_2} },
$$
and
$$
\| u \|_{L_2}^2 = \sum_{n=0}^\infty a_n^2 \| P_n \|^2.
$$
Let $P_{L_2}^{\Pi_p}$ denote the $L_2$-projection onto $\Pi_p$. There holds
$$
P_{L_2}^{\Pi_p} u = \sum_{n=0}^p a_n P_n
$$
The projection error is
$$
\| u - P_{L_2}^{\Pi_p} u \|_{L_2}^2 = \sum_{n=p+1}^\infty a_n^2 \| P_n \|^2
$$
\begin{lemma} \label{lemma_l2est}  The $L_2$-projection error satisfies
\begin{equation}
\| u - P_{L_2}^{\Pi_p} u \|_{L_2(-1,1)} \leq \frac{1}{\sqrt{(p+1)(p+2)}} \, | u |_{H^1(-1,1)}
\end{equation}
\end{lemma}
\begin{proof} Since $P_n$ are orthogonal also w.r.t. $(u^\prime, v^\prime)_{L_2, 1-x^2}$,
there holds
$$
\| u^\prime \|_{L_2, 1-x^2}^2 = \sum_{n \in \setN} a_n^2 \, \| P_n^\prime \|_{1-x^2}^2,
$$
provided that $u$ is in $H^1$.
The projection error satisfies
\begin{eqnarray*}
\| u - P_{L_2}^{\Pi_p} u \|^2_{L_2} & = & \sum_{n > p} a_n^2 \| P_n \|^2_{L_2} 
 = \sum_{n > p} a_n^2 \frac{1}{n (n+1)} \| P_n^\prime \|^2_{1-x^2} \\
& \leq & \frac{1}{(p+1)(p+2)} \sum_{n > p} a_n^2 \|P_n^\prime \|^2_{1-x^2} 
\leq  \frac{1}{(p+1)(p+2)} \sum_{n \in \setN} a_n^2 \|P_n^\prime \|^2_{1-x^2} \\
& = & \frac{1}{(p+1)(p+2)} \| u^\prime \|_{1-x^2}^2 \\
\end{eqnarray*}
Finally, the result follows from 
$$
\int_{-1}^1 (1-x^2) (u^\prime)^2 \, dx \leq \int_{-1}^1 (u^\prime)^2 \, dx.
$$
\end{proof}
\newpage
Similar as in Lemma~\ref{lemma_sturmliouville} on shows also
$$
\frac{d^m}{dx^m} \big[ (x^2-1)^m \frac{d^m}{dx^m} P_n(x) \big] = (n+m)(n+m-1) \ldots (n-m+1) P_n(x)
$$
for $m \leq n$, and, as in Lemma~\ref{lemma_l2est}
$$
\| u - P_{L_2}^{\Pi_p} u \|_{L_2} \leq \sqrt{ \frac{ (p-m+1)! }{ (p+m+1)! } } | u |_{H^m}.
$$


\subsection{Orthogonal polynomials on triangles}
%
Orthogonal polynomials on tensor product elements are simply
constructed by tensorization. Orthogonal polynomials on simplicial
elements are more advanced. They are based on Jacobi polynomials:

For $\alpha, \beta > -1$, Jacobi polynomials are defined by 
$$
P_n^{(\alpha, \beta)}(x) := \frac{(-1)^n}{2^n n!} \frac{1}{w(x)} \frac{d^n}{dx^n} \big( w(x) (1-x^2)^n \big)
$$
with the weight function 
$$
w(x) = (1-x)^\alpha (1+x)^\beta.
$$
Jacobi polynomials are orthogonal w.r.t. the weighted inner product
$$
\int_{-1}^1 w(x) P_n^{(\alpha,\beta)}(x) P_m^{(\alpha, \beta)}(x) dx = \delta_{n,m} \frac{2^{\alpha+\beta+1}}{2n+\alpha+\beta+1} \frac{ \Gamma(n+\alpha+1) \, \Gamma (n+\beta+1) } { n! \Gamma (n+\alpha+\beta+1)}.
$$
Note that $P^{(0,0)}_n = P_n$.

Define the unit-triangle $T$ with vertices $(-1,0)$,  $(1,0)$ and $(0,1)$. 
\begin{lemma} [Dubiner basis] The functions
$$
\varphi_{i,j} (x,y) := P_i \big( \sfrac{x}{1-y} \big) (1-y)^i P_j^{(2i+1,0)} (2y-1)
\qquad i+j \leq p
$$ 
form an $L_2(T)$-orthogonal basis for $\Pi_p(T)$.
\end{lemma}
\begin{proof}
Note that $\varphi_{i,j} \in \Pi_{i+j}$. Substitution $\xi = \frac{x}{1-y}$ leads to
\begin{eqnarray*}
\lefteqn{ \int_T \varphi_{ij}(x,y) \varphi_{kl} (x,y) \, d(x,y) = } \\
& = & \int_0^1 \int_{-1+y}^{1-y} 
 P_i \big( \sfrac{x}{1-y} \big) (1-y)^i P_j^{(2i+1,0)} (2y-1)
 P_k \big( \sfrac{x}{1-y} \big) (1-y)^k P_l^{(2k+1,0)} (2y-1) \, dx dy \\
& = & \int_0^1 \int_{-1}^1 P_i(\xi) P_k(\xi) (1-y)^{i+k+1} P_j^{(2i+1,0)} (2y-1) P_l^{(2k+1,0)}(2y-1) \, d\xi dy \\
& = & \delta_{i,k} \| P_i \|_{L_2}^2 \int_0^1 (1-y)^{2i+1} P_j^{(2i+1,0)}(2y-1) P_l^{(2i+1,0)}(2y-1) \, dy \\
& = & C_{ij} \delta_{i,k} \delta_{j,l} 
\end{eqnarray*}
\end{proof}



\subsection{Projection based interpolation}

By means of the orthogonal polynomials one shows approximation error estimates
of the form
$$
\inf_{q \in \Pi_p(T)} \| u - q \|_{H^k(T)} \leq c p^{k-m} | u |_{H^m(T)} \qquad m \geq k,
$$
with $c \neq c(p)$, easily in 1D and tensor product elements, and also on $n$-dimensional simplices [Braess+Schwab: Approximation on simplices with respect to weighted Sobolev norms, J. Approximation Theory 103, 329-337 (2000)].
 
But, an interpolation operator to an $H^1$-conforming finite element space has to satisfy continuity constraints across element boundaries. We show that we get the same rate of convergence under these constraints.

\subsubsection{The 1D case}
Let $I=(-1,1)$. We define the operator $I_p : H^1(I) \rightarrow \Pi_p$ such that
\begin{eqnarray}
I_p u(x) & = & u(x) \qquad x \in \{ -1, 1 \}   \label{equ_projbased1} \\
\int_I (I_p u)^\prime q^\prime & = &  \int_I u^\prime q^\prime \qquad \forall \, q \in \Pi_{p,0}(I),   \label{equ_projbased2} 
\end{eqnarray}
where $\Pi_{p,0}(D) := \{ q \in \Pi_p(D) : q = 0 \text{ on } \partial D\}$.
This procedure is exactly a $p$-version Galerkin-method for the
Dirichlet problem.  Since boundary values are preserved, the
interpolation operator produces a globally continuous function.  The
operator $I_p$ is a kind of mixture of interpolation and projection,
thus the term {\em projection based interpolation} introduced by
Demkowicz has been established.

\begin{lemma}[Commuting diagram] There holds
$$
\Pi_{L_2}^{\Pi_{p-1}} u^\prime = (I_p u)^\prime 
$$
\end{lemma}
\begin{proof} The range of both sides is $\Pi_{p-1}$. We have to show that 
$(I_p u)^\prime$ is indeed the $L_2$-projection of $u^\prime$, i.e.
$$ 
\int_I (I_p u)^\prime q = \int_I u^\prime q \qquad \forall \, q \in \Pi_{p-1}.
$$
This holds since $\{ q^\prime : q \in \Pi_{p,0} \} = \{ q \in \Pi_{p-1} : \int q = 0 \}$ and (\ref{equ_projbased2}), and 
$$
\int_{-1}^1 (I_p u)^\prime \, 1 = (I_p u)(1)  - (I_p u)(-1) = u(1) - u(-1) = \int_{-1}^1 u^\prime \, 1.
$$
\end{proof}

The $H^1$-error estimate follows directly from the commuting diagram property:
$$
| u - I_p u |_{H^1(I)} = \| u^\prime - (I_p u)^\prime \|_{L_2} = 
\| u^\prime - P_{L_2}^{\Pi_{p-1}} u^\prime \|_{L_2} \leq \frac{c}{p^{m-1}} | u^\prime |_{H^{m-1}}.
$$
By the Aubin-Nitsche technique one obtains an extra $p$ for the $L_2$-error:
$$
\| u - I_p u \|_{L_2(I)} \leqc \frac{1}{p} | u - I_p u |_{H^1(I)} \leq \frac{1}{p^m} | u |_{H^m(I)}
$$
One also gets for $q \in \Pi_p$
$$
| u - I_p u |_{H^1} = | u-q - I_p (u-q) |_1 \leq || Id - I_p ||_{H^1\rightarrow H^1} \, | u - q |_{H^1}
$$

\subsubsection{Projection based interpolation on triangles}

We define the operator $I_p : H^2(T) \rightarrow \Pi_p(T)$ as follows:
\begin{eqnarray}
I_p u(x) & = & u(x) \qquad \forall \text{ vertices } x \\
\int_E \partial_\tau (I_p u) \partial_\tau q & = &  
\int_E \partial_\tau u  \partial_\tau q \qquad \forall \text{ edges } E, \; \forall \, q \in \Pi_{p,0}(E) \\
\int_T \nabla (I_p u) \nabla q & = & \int_T \nabla u \nabla q \qquad \forall \, q \in \Pi_{p,0}(T)
\end{eqnarray}
Note that $I_p u$ on the edge $E$ depends  only on $u|_E$, and thus the interpolant is
continuous across neighbouring elements. 



\begin{lemma}  \label{lemma_polext}  Let $v \in C(\partial T)$ such that $v|_E \in \Pi_p(E)$. Then there exists
an extension $\tilde v \in \Pi_p(T)$ such that $\tilde v|_{\partial T} = v$ and
$$
| \tilde v |_{H^1} \leq c \, | v |_{H^{1/2}(\partial T)},
$$
where $c$ is independent of $p$.
\end{lemma}
Major steps have been shown in exercises 5.2 and 6.6.
Note that the minimal-norm extension $\tilde v$ is the solution of the Dirichlet problem, i.e.
$$
\int_T \nabla \tilde v \nabla w = 0 \qquad \forall \, w \in \Pi_{p,0}
$$

\begin{theorem} [error estimate] There holds 
$$
| u - I_p u |_{H^1} \leqc
\inf_{q \in \Pi_p} | u - q |_{H^1(T)} + 
\sum_{E \subset \partial T} \frac{1}{\sqrt{p}} \inf_{q \in \Pi_p(E)} | u - q |_{H^1(E)}
\leqc 
\frac{1}{p^{m-1}} | u |_{H^m} 
$$
for $u \in H^m, m \geq 2$.  
\end{theorem}
\begin{proof} Let $u_p$ be the $|\cdot |_{H^1}$ best approximation to $u$, i.e.
$$
\int_T \nabla u_p \nabla v = \int_T \nabla u \nabla v \qquad \forall \, v \in \Pi_p,
$$
and, for uniqueness, mean values are preserved: $\int_T u_p = \int_T u$. There holds
$$
| u - u_p |_{H^1} \leq  \frac{c}{p^{m-1}} | u |_{H^m}.
$$

We apply the triangle inequality:
$$
| u - I_p u |_{H^1} \leq | u - u_p |_{H^1} + | u_p - I_p u |_{H^1}
$$
Since 
$$
\int_T \nabla u_p \nabla v = \int_T \nabla u \nabla v = \int_T \nabla I_p u  \nabla v
\qquad \forall \, v \in \Pi_{p,0}(T),
$$
we have that
$$
u_p - I_p u \; \bot_{H^1} \; \Pi_{p,0},
$$
i.e. $u_p - I_p u$ is solution of the Dirichlet problem with boundary values
$ (u_p - I_p u)|_{\partial T}$. Lemma~\ref{lemma_polext} implies that
$$
| u_p - I_p u |_{H^1(T)} \leqc | u_p - I_p u |_{H^{1/2}(\partial T)}
$$

We insert an $u$ on the boundary to obtain
\begin{eqnarray*}
| u - I_p u |_{H^1} & \leqc & | u - u_p |_{H^1(T)} + | u_p - I_p u |_{H^{1/2}(\partial T)} \\
& \leq & | u - u_p |_{H^1(T)} + | u_p - u |_{H^{1/2}(\partial T)} + | u - I_p u |_{H^{1/2}(\partial T)} \\
& \leq & | u - u_p |_{H^1(T)} + 
\| u - I_p u \|_{L_2(\partial T)}^{1/2} 
| u - I_p u |_{H^1(\partial T)}^{1/2}.
\end{eqnarray*}
In the last step we used that $H^{1/2}(\partial T) = [L_2, H^1]_{1/2}$ (i.e. the interpolation space).
Next, we observe that $I_p$ restricted to one edge $E$ is exactly the 1D operator. Using Aubin-Nitsche we get
\begin{eqnarray*}
| u - I_p u |_{H^1} & \leqc & | u - u_p |_{H^1(T)} + p^{-1/2} \| u - I_p u\|_{H^1(\partial T)} \\
& \leqc & | u  - u_p |_{H^1(T)} + \sum_E p^{1-m} \, | u |_{H^{m-1/2}(E)} \\
& \leqc & p^{1-m} | u |_{H^m(T)}
\end{eqnarray*}
In the last step we used the trace theorem.
\end{proof}


% \end{document}      