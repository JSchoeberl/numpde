% \documentclass[12pt]{article}
% \usepackage{amsmath,amsthm,amssymb,a4wide}
% \usepackage[german,english]{babel}
% \usepackage{epsfig}
% \usepackage{latexsym}
% \usepackage{amssymb}
% % \usepackage{theorem}
% \usepackage{amsthm}
% % \usepackage{showkeys}

% \newcommand{\setR}{ {\mathbb R} }
% \newcommand{\setN}{ {\mathbb N} }
% \newcommand{\setZ}{ {\mathbb Z} }
% \newcommand{\eps}{\varepsilon}

% \newcommand{\beq}{\begin{equation}}
% \newcommand{\eeq}{\end{equation}}

% \newcommand{\opdiv}{\operatorname{div}}
% \newcommand{\opcurl}{\operatorname{curl}}
% \newcommand{\opdet}{\operatorname{det}}
% \newcommand{\optr}{\operatorname{tr}}
% \newcommand{\optrn}{\operatorname{tr}_n}
% \newcommand{\sfrac}[2]{ { \textstyle \frac{#1}{#2} } }

% \newcommand{\Zh}{\mathrm{Z}_h}
% \newcommand{\Ih}{\mathrm{I}_l}

% \newcommand{\leqc}{\preceq} 
% \newcommand{\geqc}{\succeq} 
% \newcommand{\eqc}{\simeq} 
% \newcommand{\ul}{\underline}

% \newtheorem{theorem}{Theorem}
% \newtheorem{definition}[theorem]{Definition}
% \newtheorem{lemma}[theorem]{Lemma}
% \newtheorem{remark}[theorem]{Remark}
% \newtheorem{example}[theorem]{Example}

% %
% %
% \setlength{\unitlength}{1cm}
% \sloppy 
% %

% \title{Hyperbolic Conservation Laws}
% \author{Joachim Sch\"oberl}

% \begin{document}
% \maketitle

\chapter{Hyperbolic Conservation Laws}


We consider the equation
$$
\frac{\partial u}{\partial t} + \opdiv f(u) = 0
$$
in space dimension $n$, with the state $u \in \setR^m$, and the 
flux $f : \setR^m \rightarrow \setR^{m \times n}$.  
We need initial conditions $u(x,0) = u_0(x)$, and proper boundary conditions.
\medskip 

\noindent
Examples:
\begin{itemize}
\item Transport equation $m = 1$, $n \in \{ 1, 2, 3, \ldots \}$. 
$$
f(u) = b^T u
$$
with $b \in \setR^n$ the given wind.

\item Burgers' equation $m = 1$, $n = 1$
$$
f(u) = \tfrac{1}{2} u^2
$$
Burgers' equation is a typical model problem to study effects of
non-linear conservation laws.

\item Wave equation in $\setR^n$:  $u = (p, v_1, \ldots v_n)$, here $m
  = n+1$:
$$
f(p,u) = \left( \begin{array}{ccc}
   u_1 & \cdots & u_n \\
  p & & \\
  & \ddots & \\
  & & p
  \end{array} \right)
$$

\item Euler equations (model for compressible flows) in $\setR^n$: $m
  = n+2$, state $u = (\rho, m_1, \ldots, m_n, E)$,
  with density $\rho$, momentum $m = \rho v$, and energy. The flux
  function is 
$$
f = \left( \begin{array}{c} \rho v \\  
   \rho v \otimes v + p I \\
   (E + p) v   \end{array} \right)
$$
with the internal energy $e = E/\rho - \tfrac{1}{2} |v|^2$
(proportional to temperature), and state equation $p = p(\rho,
e)$. Equations are conservation of mass, momentum and energy.
\end{itemize}


\section{A little theory}

Set $n = 1$, $m = 1$. We assume that $f$ is convex,
i.e. $f^\prime$ is strictly monotone increasing. For linear fluxes $f = b u$, the solution
is the traveling wave
$$
u(x,t) = u_0(x-bt).
$$
It is constant along the characteristic lines $x(t) = x_0 + bt$.

\medskip

For smooth fluxes $f$, the solution is constant along characteristic lines
$x(t) = x_0 + f^\prime (u_0(x_0)) t$:
$$
u(x(t), t) = u_0(x_0)
$$
proof:
$$
0 = \frac{d}{dt} u(x(t), t) = \frac{\partial u}{\partial t} +
        \frac{d x}{dt} \frac{\partial u}{\partial x} \\
 =  \frac{\partial u}{\partial t} + f^\prime(u) \frac{\partial
       u}{\partial x} 
= f_t + (f(u))_x
$$
The smooth solution exists as long as characteristic lines don't
intersect.

Example: Burgers equation. The velocity of the characteristic is
$f^\prime(u) = (\tfrac{1}{2} u^2)^\prime = u$, i.e. the solution itself.

\subsection{Weak solutions and the Rankine-Hugoniot relation}
%
If characteristic lines intersect, the solution forms a shock. the
Rankine-Hugoniot relation is a equation for the speed of the shock. 

We assume the solution is piecewise smooth.
To have meaningful discontinuous solutions, we have to consider weak
solutions in space-time:
$$
\int_{\Omega \times (0,T)} u \varphi_t + f(u) \nabla \varphi = 0 \qquad
\forall \varphi \in C_0^\infty 
$$
(initial condition is skipped here, easily covered by non-vanishing
test-functions for $t=0$).

The weak form states that for $F = (f, u)$ there holds 
$$
\opdiv_{x,t} F = 0 
$$
in weak senses. Thus, $F \in H(\opdiv)$.  This requires that $F \cdot
n$ is continuous across discontinuities. Let $s(t)$ the position
of the shock. The normal vector satisfies
$$
n \sim (1, -s^\prime)
$$
Thus, $[F \cdot n] = 0$ reads as
$$
f(u_l) - u_l s^\prime = f(u_r) - u_r s^\prime
$$
and we get the Rankine-Hugoniot relation
$$
s^\prime = \frac{f(u_l) - f(u_r)}{u_l - u_r} 
$$

Example Burgers: The speed of the shocks is
$$
s^\prime = \frac{ \tfrac{1}{2} u_l^2 - \tfrac{1}{2} u_r^2}  {u_l -
  u_r} = \frac{u_l + u_r}{2}
$$

\subsection{Expansion fans}
Assume $u(x+) > u(x-)$, then, due to convexity of the flux there is
also $f^\prime(u(x+)) > f^\prime(u(x-))$.
The speed on the right is higher than on the left. Here, all
monotone increasing functions (between $u(x-)$ and $u(x+)$), 
constant along lines $x + f^\prime(u) t$ are weak solutions.

Another conditions is necessary to pick the meaningful physical
solution. Two choices are

{\em Viscosity solutions}. Consider the regularized equation
$$
u_t^{\eps} + f(u^{\eps})_x - \eps u_{xx}^{\eps}= 0
$$
The limit (if existent)  $\lim_{\eps \rightarrow 0} u^\eps$ is called
viscosity solution.

{\em Entropy solutions}. We define some quantity $E(u)$ called
entropy, where, for physical reasons, the total amount should not
increase:
$$
\frac{d}{dt} \int_\Omega E(u) \leq 0 
$$
To localize it, we define the entropy flux $F$ such that
$$
F^\prime = E^\prime f^\prime
$$
If the solution is smooth, then
$$
E(u)_t + F(u)_x = E^\prime u_t + F^\prime u_x = E^\prime ( u_t +
f^\prime u_x) = 0
$$
Thus
$$
\frac{d}{dt} \int_{\Omega} E(u) = \int_\Omega E(u)_t = -\int_\Omega
F(u)_x = -\int_{\partial \Omega} F(u) \cdot n
$$
No entropy changes for smooth solutions with isolated boundary.  But,
this is not true for discontinuous solutions. 

We pose the entropy decrease $E(u)_t + F(u)_x \leq 0$ in weak sense:
$$
- \int_{\Omega \times (0,T)}  E(u) \varphi_t + F(u) \nabla \varphi \leq 0
\qquad \forall \, \varphi \in C_0^\infty, \varphi \geq 0
$$
Similar to the Rankine-Hugoniot relation we integrate back on smooth
regions in space-time
$$
\sum_{( \Omega\times (0,T))_i}  \int \big( \underbrace{E(u)_t + \opdiv
  F(u)}_{= 0} \big) \varphi - 
\int_\gamma \big(  [E(u)] s^\prime - [F(u)]  \big) \varphi \leq 0
\qquad \forall \, \varphi \geq 0
$$
Example Burgers:
Choose the entropy $E(u) = u^2$. Then 
$$
F(u) = \int E^\prime f^\prime = \int 2 u u = \frac{2}{3} u^3
$$
Calculating 
\begin{eqnarray*}
[E] s^\prime - [F] & = & (u_r^2 - u_l^2) \frac{u_r + u_l}{2} +
                         \frac{2}{3} (u_r^3 - u _l^3) = \ldots \\ 
& = & -(u_r - u_l) \frac{(u_r-u_l)^2}{6}
\end{eqnarray*}
Now, posing the non-negative condition for $[E] s^\prime - [F]$ we
allow jumps only for $u_r < u_l$. 

\section{Numerical Methods}

The natural methods for conservation laws are finite volume /
discontinuous Galerkin methods:
$$
\int_T \frac{\partial u}{\partial t}  v - f(u) \nabla v +
\int_{\partial T} g (u_l, u_r) v = 0 \qquad \forall \, T \; \forall \,
v \in P^k(T)
$$
Here, $g$ is the numerical flux on the element boundary, calculated
from left- and right sided states. For continuous $u = u_l = u_r$, it satisfies
$$
g(u, u) = f(u) n 
$$
Otherwise, up-wind like fluxes (many different choices !) are
used.

Finite volume methods can be designed such that entropy is
non-increasing, often the calculations are technical. They are also
used to prove the existence of solutions. 

Higher order methods do not satisfy the maximum principle, which may
lead to problems for non-linear equations (Euler: divide by
$\rho$). Here, {\em limiters} are used: If the solution produces
oscillations, it is smoothed (somehow). E.g., switch back to a finite
volume method.

\medskip

Recent development (by Guermond+Pasquetti+Popov) is the so-called
{\em entropy viscosity method}: If the entropy relation is violated,
artificial viscosity is switched on. Ideally, this happens only close
to shocks.

\medskip

Space-time methods include special mesh-generation related to the
finite speed of propagation (front tracking methods,  tent-pitching
methods). [Gopalakrishnan, Sch\"oberl, Wintersteiger 2016, master thesis Wintersteiger].

% \end{document}