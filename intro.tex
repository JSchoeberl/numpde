\chapter{Introduction}
%
{\em Differential equations} are equations for an unknown function
involving differential operators. An {\em ordinary} differential
equation (ODE) requires differentiation with respect to one variable. A {\em
partial} differential equation (PDE) involves partial differentiation with
respect to two or more variables.

\section{Literatur}

Some general textbooks for this class are

\begin{itemize}
\item {\bf Dietrich Braess} {\it Finite Elemente: Theorie, schnelle L\"oser und Anwendungen}, Springer 2013 (English and Chinese versions available)
\item {\bf Susanne C. Brenner, L. Ridgway Scott} {\it The mathematical Theory of Finite Element Methods}, Sringer 2008
\item {\bf Daniele Boffi, Franco Brezzi, Michel Fortin} {\it Mixed Finite Element Methods and Applications}, Springer 2013
\item {\bf Alexandre Ern, Jean-Luc Guermond} {\it Finite Elements I-III}, Springer 2021
\end{itemize}


\section{Classification of PDEs}
%
The general form of a linear PDE of second order is: find $u : \Omega \subset \setR^d \rightarrow \setR$ such that
\beq \label{general_pde}
\sum_{i,j = 1}^d 
-\frac{\partial}{\partial x_i} \big( a_{i,j}(x) \frac{\partial u(x)}{\partial x_j}  \big) + 
\sum_{i = 1}^d 
b_{i}(x) \frac{\partial u(x)}{\partial x_i}  +  c(x) u(x) = f(x).
\eeq
%
The coefficients $a_{i,j}(x), b_i(x), c(x)$ and the right hand side $f(x)$ are
given functions. In addition, certain type of boundary conditions are
required. The behavior of the PDE depends on the type of the
differential operator
$$
L := 
\sum_{i,j = 1}^d 
\frac{\partial}{\partial x_i} a_{i,j} \frac{\partial}{\partial x_j}   + 
\sum_{i = 1}^d 
b_{i} \frac{\partial}{\partial x_i}   +  c.
$$
Replace $\frac{\partial}{\partial x_i}$ by $s_i$. Then
$$
\sum_{i,j = 1}^d s_i a_{i,j} s_j +  \sum_{i = 1}^d  b_{i} s_i  +  c = 0
$$
%
describes a quartic shape in $\setR^d$. We treat the following cases:
\begin{enumerate}
\item
In the case of a (positive or negative) definite matrix $a = (a_{i,j})$ this is an
ellipse, and the corresponding PDE is called elliptic. A simple example is
$a = I$, $b = 0$, and $c = 0$, i..e.
$$
- \sum_i \frac{\partial^2 u}{\partial x_i^2} = f.
$$
Elliptic PDEs require boundary conditions.
\item
If the matrix $a$ is semi-definite, has the one-dimensional kernel $\operatorname{span} \{v \}$, and $b \cdot v \neq 0$, then the shape is a parabola. Thus, the PDE is called parabolic. A simple example is
$$
- \sum_{i=1}^{d-1} \frac{\partial^2 u}{\partial x_i^2} + \frac{\partial u}{\partial x_d} = f.
$$
Often, the distinguished direction corresponds to time. This type of equation requires boundary conditiosn on the $d-1$-dimensional boundary, and initial conditions in the different direction.
\item
If the matrix $a$ has $d-1$ positive, and one negative (or vise versa) eigenvalues, then the shape is a hyperbola. The PDE is called hyperbolic. The simplest one is
$$
- \sum_{i=1}^{d-1} \frac{\partial^2 u}{\partial x_i^2} + \frac{\partial^2 u}{\partial x_d^2} = f.
$$
Again, the distinguished direction often corresponds to time. Now, two initial conditions are needed.
\item
If the matrix $a$ is zero, then the PDE degenerates to the first order PDE
$$
b_{i} \frac{\partial u}{\partial x_i} +  c u = f.
$$
Boundary conditions are needed at a part of the boundary.
\end{enumerate}
%
These cases behave very differently. We will establish theories for
the individual cases.  A more general classicfication, for more
positive or negative eigenvalues, and systems of PDEs is possible. The
type of the PDE may also change for different points $x$.






\section{Weak formulation of the Poisson Equation}
\label{sec_intro}
The most elementary and thus most popular PDE is the Poisson equation
\begin{equation}
\label{equ_laplace}
-\Delta u = f \qquad \mbox{ in } \Omega,
\end{equation}
with the boundary conditions
\begin{equation}
\label{equ_laplace_bc}
\begin{array}{rcll}
u & = & u_D \qquad & \mbox{on } \Gamma_D, \\
\frac{\partial u}{\partial n} & = & g \qquad & \mbox{on } \Gamma_N, \\
\frac{\partial u}{\partial n} + \alpha u & = & g \qquad & \mbox{on } \Gamma_R.
\end{array}
\end{equation}
%
The domain $\Omega$ is an open and bounded subset of $\setR^d$, where
the problem dimension $d$ is usually 1, 2 or 3. For $d = 1$, the
equation is not a PDE, but an ODE. The boundary $\Gamma := \partial
\Omega$ consists of the three non-overlapping parts $\Gamma_D$,
$\Gamma_N$, and $\Gamma_R$. The outer
unit normal vector is called $n$. The Laplace differential operator is
$\Delta := \sum_{i=1}^d
\frac{\partial^2}{\partial x_i^2}$, the normal derivative at the
boundary is $\frac{\partial}{\partial n} := \sum_{i=1}^d n_i
\frac{\partial}{\partial x_i}$. Given are the functions $f$, $u_D$ and
$g$ in proper function spaces (e.g., $f \in L_2(\Omega)$). We search for
the unknown function $u$, again, in a proper function space defined later. 

The boundary conditions are called
\begin{itemize}
\item 
Dirichlet boundary condition on $\Gamma_D$. The function value is prescribed,
\item
Neumann boundary condition on $\Gamma_N$. The normal derivative is prescribed,
\item
Robin boundary condition on $\Gamma_R$. An affine linear relation between
the function value and the normal derivative is prescribed.
\end{itemize}

Exactly one boundary condition must be specified on each part of the
boundary.

We transform equation (\ref{equ_laplace}) together with the boundary
conditions (\ref{equ_laplace_bc}) into its weak
form. For this, we multiply (\ref{equ_laplace}) by smooth functions 
(called test functions) and integrate over the domain:
\begin{equation}
\label{equ_multiplied}
- \int_\Omega \Delta u v \, dx  = \int_\Omega f v \, dx 
\end{equation}
We do so for sufficiently many test functions $v$ in a proper function space.
Next, we apply Gauss' theorem 
$
\int_\Omega \opdiv p \, dx = \int_\Gamma p \cdot n \, ds
$ 
to the function $p := \nabla u \, v$ to obtain
$$
\int_\Omega \opdiv (\nabla u \, v) \, dx = \int_\Gamma \nabla u \cdot n \, v \, ds
$$
From the product rule there follows $\opdiv (\nabla u v) = \Delta u v + \nabla u \cdot \nabla v$. Together we obtain
$$
\int_\Omega \nabla u \cdot \nabla v \, dx - \int_\Gamma \frac{\partial u}{\partial n} v \, ds = \int_\Omega f v \, dx.
$$


Up to now, we only used the differential equation in the domain. Next, we
incorporate the boundary conditions. The Neumann and Robin b.c. are very
natural (and thus are called natural boundary conditions). We simply replace
$\frac{\partial u}{\partial n}$ by $g$ and $-\alpha u + g$ on $\Gamma_N$ and
$\Gamma_R$, respectively. Putting unknown terms to the left, and known terms
to the right hand side, we obtain
$$
\int_\Omega \nabla u \cdot \nabla v \, dx + \int_{\Gamma_R} \alpha u v \, ds 
-\int_{\Gamma_D} \frac{\partial u}{\partial n} v \, ds = 
\int f v \, dx + \int_{\Gamma_N+\Gamma_R} g v \, ds.
$$
Finally, we use the information of the Dirichlet boundary condition. We
work brute force and simple keep the Dirichlet condition in strong sense.
At the same time, we only allow test functions $v$ fulfilling $v = 0$
on $\Gamma_D$. We obtain the 
\begin{quote}
{\bf Weak form of the Poisson equation:} \newline
Find $u$ such that $u = u_D$ on $\Gamma_D$ and
\begin{equation}
\label{equ_weak_form}
\int_\Omega \nabla u \cdot \nabla v \, dx + \int_{\Gamma_R} \alpha u v \, ds = 
\int_\Omega f v \, dx + \int_{\Gamma_N+\Gamma_R} g v \, ds \quad
\end{equation}
\hfill $\forall \, v \mbox{ such that } v = 0 \mbox{ on } \Gamma_D.$
\end{quote}
%
We still did not define the function space in which we search for the
solution $u$. A proper choice is $$ V := \{ v \in L_2(\Omega) : \nabla
u \in [L_2(\Omega)]^d \mbox{ and } u|_\Gamma \in L_2(\partial \Omega)
\}.  $$ It is a complete space, and, together with the inner product
$$ (u,v)_V := (u,v)_{L_2(\Omega)} + (\nabla u, \nabla v)_{L_2(\Omega)}
+ (u, v)_{L_2(\Gamma)} $$ it is a Hilbert space.  Now, we see that $f
\in L_2(\Omega)$ and $g \in L_2(\Gamma)$ is useful. The Dirichlet
b.c. $u_D$ must be chosen such that there exists an $u \in V$ with $u
= u_D$ on $\Gamma_D$.  By definition of the space, all terms are well
defined. We will see later, that the problem indeed has a unique
solution in $V$. 


\section{The Finite Element Method}
%
Now, we are developing a numerical method for approximating the weak
form~(\ref{equ_weak_form}). For this, we decompose the domain $\Omega$
into triangles $T$. We call the set ${\cal T} = \{ T \}$
triangulation. The set ${\cal N} = \{ x_j \}$ is the set of nodes.
By means of this triangulation, we define the finite element space,
$V_h$:
%
$$
V_h := \{ v \in C(\Omega) : v|_T \mbox{ is affine linear } \; \forall \, T \in {\cal T} \}
$$
%
This is a sub-space of $V$. The derivatives (in weak sense, see below) are
piecewise constant, and thus, belong to $[L_2(\Omega)]^2$.
The function $v_h \in V_h$ is uniquely defined by its values $v(x_j)$ in the 
nodes $x_j \in {\cal N}$. We decompose the set of nodes as
$$
{\cal N} = {\cal N}_D \cup {\cal N}_f,
$$
%
where ${\cal N_D}$ are the nodes on the Dirichlet boundary, and
${\cal N}_f$ are all others ($f$ as free).  The finite element approximation 
is defined as
%
\begin{quote}
Find $u_h$ such that $u_h(x) = u_D(x) \; \; \forall \, x \in {\cal N}_D$ and
\begin{equation}
\label{equ_fem}
\int_\Omega \nabla u_h \cdot \nabla v_h \, dx + \int_{\Gamma_R} \alpha u_h v_h \, ds =
\int f v_h \, dx + \int_{\Gamma_N+\Gamma_R} g v_h \, ds \quad
\end{equation}
\hfill
$\forall \, v_h \in V_h \mbox{ such that } v_h(x) = 0 \; \; \forall \, x \in {\cal N}_D$
\end{quote}

Now it is time to choose a basis for $V_h$. The most convenient one is 
the nodal basis $\{ \varphi_i \}$ characterized as
%
\begin{equation}
\label{equ_nodal_prop}
\varphi_i(x_j) = \delta_{i,j}.
\end{equation}
%
The Kronecker-$\delta$ is defined to be 1 for $i=j$, and 0 else. These are the popular
hat functions.
We represent the finite element solution with respect to this basis:
\begin{equation}
\label{equ_basisexpansion}
u_h(x) = \sum u_i \varphi_i(x)
\end{equation}
By the nodal-basis property (\ref{equ_nodal_prop}) there holds 
$u_h(x_j) = \sum_i u_i \varphi_i(x_j) = u_j$.
We have to determine the coefficients $u_i \in \setR^N$, with 
$N = | {\cal N} |$. The $N_D := | {\cal N}_D | $ values according to nodes on $\Gamma_D$ are
given explicitely:
$$
u_j = u_h (x_j) = u_D(x_j) \qquad \forall \, x_j \in \Gamma_D
$$
The others have to be determined from the variational equation (\ref{equ_fem}).
It is equivalent to fulfill (\ref{equ_fem}) for the whole space $\{ v_h \in V_h : v_h(x_j) = 0 \; \forall \; x_j \in {\cal N}_D \}$, or just for its basis 
$\{ \varphi_i : x_i \in {\cal N}_f\}$ associated to the free nodes:
\begin{quote}
\begin{equation}
\sum_i \Big\{ \int_\Omega \nabla \varphi_i \cdot \nabla \varphi_j \, dx +
 \int_{\Gamma_R} \alpha \varphi_i \varphi_j \, ds \Big\} u_i   = 
\int f \varphi_j \, dx + \int_{\Gamma_N+\Gamma_R} g \varphi_j \, ds \quad
\end{equation}
\hfill
$\forall \, \varphi_j \; \mbox{ such that } x_j \in {\cal N}_f$
\end{quote}
We have inserted the basis expansion (\ref{equ_basisexpansion}).
We define the matrix $A = (A_{ji}) \in \setR^{N\times N}$ and the vector $f = (f_j) \in \setR^N$ as
\begin{eqnarray*}
A_{ji} & := & \int_\Omega \nabla \varphi_i \cdot \nabla \varphi_j \, dx +
 \int_{\Gamma_R} \alpha \varphi_i \varphi_j \, ds, \\ 
f_j & := & \int f \varphi_j \, dx + \int_{\Gamma_N+\Gamma_R} g \varphi_j \, ds.
\end{eqnarray*}
According to Dirichlet- and free nodes they are splitted as
$$
A = \left( \begin{array}{cc}
        A_{DD} & A_{Df} \\
        A_{fD} & A_{ff} 
        \end{array} \right)
\qquad \text{and} \qquad
f = \left( \begin{array}{c}
        f_{D} \\
        f_{f}
        \end{array} \right).
$$
Now, we obtain the system of linear equations for $u = (u_i) \in \setR^N$, $u = (u_D, u_f)$:
\begin{equation}
\label{equ_linear_system}
\left( \begin{array}{cc}
        I & 0 \\
        A_{fD} & A_{ff} 
        \end{array} \right)
 \left( \begin{array}{c}
        u_{D} \\
        u_{f}
        \end{array} \right)
=
\left( \begin{array}{c}
        u_{D} \\
        f_{f}
        \end{array} \right).
\end{equation}
At all, we have $N$ coefficients $u_i$. $N_D$ are given explicitely from the
Dirichlet values. These are $N_f$ equations to determine the remaining ones.
Using the known $u_D$, we can reformulate it as symmetric system of equations for $u_f \in \setR^{N_f}$:
$$
A_{ff} u_f = f_f - A_{fD} u_D
$$
