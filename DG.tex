% \documentclass[12pt]{article}
% \usepackage{amsmath,amsthm,amssymb,a4wide}
% \usepackage[german,english]{babel}
% \usepackage{epsfig}
% \usepackage{latexsym}
% \usepackage{amssymb}
% % \usepackage{theorem}
% \usepackage{amsthm}
% % \usepackage{showkeys}

% \newcommand{\setR}{ {\mathbb R} }
% \newcommand{\setN}{ {\mathbb N} }
% \newcommand{\setZ}{ {\mathbb Z} }
% \newcommand{\eps}{\varepsilon}

% \newcommand{\beq}{\begin{equation}}
% \newcommand{\eeq}{\end{equation}}

% \newcommand{\opdiv}{\operatorname{div}}
% \newcommand{\opcurl}{\operatorname{curl}}
% \newcommand{\opdet}{\operatorname{det}}
% \newcommand{\optr}{\operatorname{tr}}
% \newcommand{\optrn}{\operatorname{tr}_n}
% \newcommand{\sfrac}[2]{ { \textstyle \frac{#1}{#2} } }

% \newcommand{\Zh}{\mathrm{Z}_h}
% \newcommand{\Ih}{\mathrm{I}_l}

% \newcommand{\leqc}{\preceq} 
% \newcommand{\geqc}{\succeq} 
% \newcommand{\eqc}{\simeq} 
% \newcommand{\ul}{\underline}

% \newtheorem{theorem}{Theorem}
% \newtheorem{definition}[theorem]{Definition}
% \newtheorem{lemma}[theorem]{Lemma}
% \newtheorem{remark}[theorem]{Remark}
% \newtheorem{example}[theorem]{Example}

% %
% %
% \setlength{\unitlength}{1cm}
% \sloppy 
% %

% \title{Discontinuous Galerkin Methods}
% \author{Joachim Sch\"oberl}

% \begin{document}
% \maketitle

\chapter{Discontinuous Galerkin Methods}
Discontinuous Galerkin (DG) methods approximate the solution with
piecewise functions (polynomials), which are discontinuous across
element interfaces. Advantages are
\begin{itemize}
\item block-diagonal mass matrices which allow cheap explicit
  time-stepping
\item upwind techniques for dominant convection
\item coupling of non-matching meshes
\item more flexibility for stable mixed methods
\end{itemize}
DG methods require more unknowns, and also have a denser stiffness
matrix. The last disadvantage can be overcome by hybrid DG methods (HDG).


\section{Transport equation}
We consider the first order equation
$$
\opdiv (bu) = f  \qquad \text{on}  \; \Omega,
$$
where $b$ is the given wind, and $f$ is the given source.
Boundary conditions are specified
$$
u = u_D \qquad \text{on} \; \Gamma_{in},
$$
where the inflow boundary is 
$$
\Gamma_{in} = \{ x \in \partial \Omega : b \cdot n < 0 \},
$$
and the outflow boundary $\Gamma_{out} = \partial \Omega \setminus
\Gamma_{in}$.

The instationary transport equation
$$
\frac{\partial u}{\partial t}  + \opdiv (bu) = f  \qquad \text{on}  \;
\Omega \times (0,T)
$$
with initial conditions $u = u_0$ for $t = 0$ can be considered as
stationary transport equation in space-time:
$$
\opdiv_{x,t} (\tilde b u) = f,
$$
where $\tilde b = (b, 1)$. The inflow boundary consists now of the
lateral boundary $\Gamma_{in} \times (0,T)$ and the bottom
 boundary  $\Omega \times \{0 \}$, which is an inflow boundary according
 to $(b, 1) \cdot (0,-1) < 0$.


The equation in conservative form leads to a conservation
principle. Let $V$ be an arbitrary control volume. From the Gau\ss{}
theorem we get
$$
\int_{\partial V} b\cdot n u = \int_V f
$$
The total outflow is in balance with the production inside $V$.


For stability, we assume $\opdiv b = 0$.  This is a realistic
assumption, since the wind is often the solution of the
incompressible Navier Stokes equation.

A variational formulation is
$$
\int \opdiv (bu) v = \int f v \qquad \forall v 
$$
If we set $v = u$, and use
$$
\opdiv (bu) u =  \tfrac{1}{2} \opdiv (b u^2)
$$
We obtain
$$
\int \opdiv (b u) v = \tfrac{1}{2} \int_{\partial \Omega} b_n u^2 =
\int f u
$$
For $f = 0$ we obtain 
$$
\int_{\Gamma_{out}} |b_n| u^2 = \int_{\Gamma_{in}} | b_n | u^2.
$$
This inflow-outflow isometrie is a stability argument. For time dependent problems
(with $b_n = 0$ on $\partial \Omega$), it ensures the conservation of
$L_2$-norm in time.

\subsection{Solvability}

We assume $b \in L_\infty$ with $\opdiv b = 0$. We consider the problem: find $u \in V$,
$u = u_D$ on $\Gamma_{in}$ and 
$$
B(u,v) = f(v) \qquad \forall \, v \in W
$$
with 
$$
B(u,v) = \int \opdiv (bu) v \qquad \text{and} \qquad f(v) = \int f v 
$$
The space $V$ shall be defined by the (semi)-norm
$$
\| u \|_V = \| b \nabla u \|_{L_2}
$$
Depending on $b$, it is a norm for $\{ u = 0 \; \text{on} \; \Gamma_{in}
\}$. Roughly speaking, if every point in $\Omega$ can be reached by a
finite trajectory along $b$, then $\| u \|_{L_2} \leqc \| u \|_V$, by
Friedrichs inequality. It does not hold if $b$ has vortices. Then
$\| \cdot \|_V$ is only a semi-norm. Note, for space-time problems
$\tilde b$ cannot have vortices. For theory, we will assume that
$$
\| u \|_{L_2} \leqc \| u \|_V \qquad \forall \, u = 0 \; \text{on} \; \Gamma_{in}
$$

The test space is
$$
W  = L_2
$$
The forms $B(.,.)$ and $f(.)$ are continuous. The $\inf-\sup$
conditions is trivial:
$$
\sup_v \frac{B(u,v)} { \| v \|_{L_2}} \underbrace{ \geq }_{v := b
 \nabla u} \frac { \int (b \nabla u)^2 } { \| b \nabla u
  \| } = \| b \nabla u\|_{L_2}
$$


\section{Discontinuous Galerkin Discretization}
A DG method is a combination of finite volume methods and
finite element methods.
We start with a triangulation $\{ T \}$. On every element we multiply
the equation by a test-function:
$$
\int_T b \nabla u v = \int_T f v  \qquad \forall \, v
$$
We integrate by parts:
$$
-\int_T b u \nabla v + \int_{\partial T} b_n u v = \int_T f v  
$$ 
On the element-boundary we replace $b_n u$ by its up-wind limit $b_n
u^{up}$. On the element inflow boundary $\partial T_{in} = \{ x \in \partial T
: b n_T < 0 \}$, the upwind value is the value from the neighbour
element, while on the element outflow boundary it is the value from
the current element $T$. For elements on the domain inflow boundary,
the upwind value is taken as the boundary value $u_D$.
For the continuous solution there holds
$$
-\int_T b u \nabla v + \int_{\partial T} b_n u^{up} v = \int_T f v 
$$ 
Now we integrate back:
$$
\int_T b \nabla u v + \int_{\partial T} b_n (u^{up}-u) v = \int_T f v 
$$ 
On the outflow boundary, the boundary integral cancels out, on the
inflow boundary we can write it as a jump term $[u] = u^{up} - u$:
$$
\int_T b \nabla u v + \int_{\partial T_{in}} b_n [u] v = \int_T f v 
$$
We define the DG bilinear-form
$$
B^{DG} (u,v) = \sum_T \left\{ \int_T b \nabla u v + \int_{\partial
  T_{in}} b_n 
[u] v \right\}.
$$
The true solution is consistent with 
$$
B^{DG} (u,v) = f(v)  \qquad \forall \, v \; \text{piece-wise continuous}.
$$

We define DG finite element spaces:
$$
V_h := W_h := \{ v \in L_2 : v |_T \in P^k \} 
$$
The DG formulation is: find $u_h \in V_h$ such that
$$
B^{DG} (u_h, v_h) = f(v_h)  \qquad \forall \, v_h \in W_h
$$
For the discontinuous space, the jump-term is important. If we use 
continous spaces, the jump-term disappears.
The discrete norms are defined as 
\begin{eqnarray*}
\| u_h \|_{V_h}^2 & := & \sum_T \| b \nabla u \|_{L_2(T)}^2 +
  \sum_E \tfrac{1}{h} \| b_n [u] \|_{L_2(E)}^2  \\
\| v_h \|_{W_h} & = & \| v_h \|_{L_2}
\end{eqnarray*}
The part with the jump-term mimics the derivative as kind of finite
difference term across edges.

We prove solvability of the discrete problem by showing a discrete
$\inf-\sup$ condition.  But, in general, one order in $h$ is lost due to a
mesh-dependent $\inf-\sup$ constant. This factor shows up in the
general error estimate by consistency and stability. It can be avoided
in 1D, and on special meshes. 

\begin{theorem} There holds the discrete $\inf-\sup$ condition
$$
\sup_{v_h} \frac{B(u_h, v_h)}{ \| v_h \|_{W_h}}  \geqc h \, \| u_h \|_{V_h}
$$
\end{theorem}
\begin{proof}
We take two different test-functions: $v_1 = u_h$ and $v_2 := b \cdot
\nabla_T u$, and combine them properly. The second test-function would not be possible in the
standard $C^0$ finite element space.

There holds (dropping sub-scripts $h$):
\begin{eqnarray*}
B(u_h, v_1) & = & B(u_h, u_h) = \sum_T \int_T b \nabla u \, u + \int_{\partial T_{in}}
  b_n [u] u \\
& = & \sum_T \frac{1}{2} \int_{\partial T} b_n u^2 + \int_{\partial T_{in}}
  b_n [u] u 
\end{eqnarray*}

We reorder the terms edge-by-edge. On the edge $E$ we get
contributions from two elements: On the inflow-boundary of the
down-wind element we get
\begin{eqnarray*}
\tfrac{1}{2} \int_E b_n (u^d)^2 + \int_E b_n (u^u - u^d) u^d = \int_E
  |b_n| \left( \tfrac{1}{2} (u^d)^2 - u^u u^d \right)
\end{eqnarray*}
We used that $b\cdot n$ is negative on the inflow boundary. From the
up-wind element we get on its outflow-boundary:
$$
\int_E \tfrac{1}{2} |b_n| \, (u^u)^2 
$$
Summing up, we have the square
$$
\int_E \tfrac{1}{2} | b_n |  (u^u  - u^d)^2,
$$
and summing over elements we get the non-negative term
$$
B(u_h, u_h) = \tfrac{1}{2} \sum_E \int_E |b_n| [u]^2.
$$
An extra treatment of edges on the whole domain boundary gives that
the jump must be  replaced by the function values on $\partial u$.

We plug in the second test function $v_2$:
\begin{eqnarray*}
B(u_h, v_2) & = & \sum_T \int_T (b \nabla u)^2 +
                  \int_{\partial {T_{in}}} b_n [u] \, b \nabla u 
\end{eqnarray*}
We use Young's inequality to bound the second term from below:
$$
B(u_h, v_2) \geq \sum_T \int_T (b \nabla u)^2 - \frac{1}{2
  \gamma} \| b_n  [u] \|^2_{\partial L_2(T_{in})} - \frac{\gamma}{2}  \| b
\nabla u\|_{L_2(\partial T_{in})}^2
$$
By the choice $\gamma \eqc h$ and a inverse trace inequality (which
needs smoothness assumptions onto $b$) we can bound the last term by the first
one on the right hand. Thus
$$
B(u_h, v_2) \geqc \sum_T \int_T (b \nabla u)^2 - \sum_E \frac{1}{h}  \| [u] \|^2_{L_2(E),b_n} 
$$
Finally, we set
$$
v_h = \frac{1}{h}  v_1 + v_2
$$
to obtain 
$$
B(u_h, v_h)  \geqc   \sum_T \int_T (b \nabla u)^2 + \sum_E \frac{1}{h}
\| [u] \|^2_{L_2(E),b_n}    \eqc  \| u_h \|^2_{V_h}.
$$
But, for this choice we get
$$
\| v_h \|_{W_h} \leqc h^{-1} \| u_h \|_{V_h},
$$
and thus the $h$-dependent $\inf-\sup$-{\it constant}.
\end{proof}

\newpage


\section{Nitsche's method for Dirichlet boundary conditions}

We build in Dirichlet b.c. in a weak sense. In constrast to a mixed
method, we obtain a positive definit matrix.

We consider the equation 
$$
-\Delta u = f \qquad \text{and} \qquad u = u_D \; \text{on}
\; \partial \Omega.
$$
A diffusion coefficient, or mixed boundary conditions are possible as
well. We multiply with testfunctions, integrate and integrate by
parts:
$$
\int_\Omega \nabla u  \nabla v -\int  \partial_n u v = \int f v \qquad \forall \, v  
$$ 
We do not restrict test functions to $v = 0$. To obtain a symmetric
bilinear-form, we add a consistent term
$$
\int_\Omega \nabla u  \nabla v -\int_{\partial \Omega}   \partial_n u v - \int_{\partial \Omega} \partial_n v
u = \int f v  - \int_{\partial \Omega}  \partial_n v u_D \qquad \forall \, v  
$$
Finally, to obtain stability (as proven below), we add the so called
stabilization term
$$
\int_\Omega \nabla u  \nabla v -\int_{\partial \Omega}   \partial_n u v - \int_{\partial \Omega} \partial_n v 
u + \frac{\alpha}{h} \int u v = \int f v  - \int_{\partial 
  \Omega}  \partial_n v u_D   + \frac{\alpha}{h} \int u_D v    \qquad \forall \, v  
$$
These are the forms of Nitsche's method:
\begin{eqnarray*}
A(u,v) & = & \int_\Omega \nabla u  \nabla v -\int_{\partial \Omega}   \partial_n u v - \int_{\partial \Omega} \partial_n v 
u + \frac{\alpha}{h} \int_{\partial \Omega}  u v \\
f(v) & = & \int_\Omega f v  - \int_{\partial 
  \Omega}  \partial_n v u_D   + \frac{\alpha}{h} \int_{\partial \Omega}  u_D v
\end{eqnarray*}
$A(.,.)$ is not defined for $u,v \in H^1$, but it requires also well
defined normal derivatives. This is satisfied for the  flux $\nabla u \in
H(\opdiv)$ of the solution, and finite element test functions $v$.

We define the Nitsche norm:
$$
\| u \|^2_{1,h} := \| \nabla u \|_{L_2}^2 + \tfrac{1}{h} \| u
\|_{L_2(\partial \Omega)}^2
$$
\begin{lemma} If $\alpha = O(p^2)$ is chosen sufficiently large, then
  $A(.,.)$ is elliptic on the finite element space:
$$
A(u_h, u_h) \geqc \| u_h \|^2_{1,h}  \quad \forall \, u_h \in V_h
$$
\end{lemma}
\begin{proof} On one element there holds the inverse trace inequality
$$
\| u_h \|_{L_2(\partial T)}^2 \leq  c \, \frac{p^2}{h} \| u_h \|_{L_2}^2
\qquad \forall \, u_h \in P^p(T)
$$
The $h$-factor is shown by transformation to the reference element,
the $p$-factor (polynomial order) is proven by expansion in terms of orthogonal
polynomials. Using the element-wise estimate for all edges on the
domain boundary, we obtain
\begin{equation}
\| u_h \|_{L_2(\partial \Omega)}^2 \leq c \, \frac{p^2}{h} \| u \|_{L_2(\Omega)}^2
\end{equation}
Evaluating the bilinear-form, and applying Young's inequality for the
mixed term we get
\begin{eqnarray*}
A(u_h, u_h) & = & \| \nabla u_h \|^2_{L_2} - 2 \int_{\partial
                  \Omega}  \partial_n u u + \frac{\alpha}{h}
                  \| u \|_{L_2(\partial \Omega)}^2 \\
& \geq & \| \nabla u_h \|^2_{L_2} 
         - \tfrac{1}{\gamma} \| n\cdot \nabla u_h \|_{L_2(\partial
         \Omega)} ^2 
      - \gamma \| u \|_{L_2(\partial \Omega)}^2 + \frac{\alpha}{h}
                  \| u \|_{L_2(\partial \Omega)}^2 
\end{eqnarray*}
The inverse trace inequality applied to $\nabla u_h$ gives
$$
\| n \cdot \nabla u_h \|_{\partial \Omega} \leq \| \nabla u_h
\|_{\partial \Omega}
 \leq c \frac{p^2}{h} \| \nabla u_h \|_\Omega
$$
By choosing 
$$
\gamma > c \frac{p^2}{h} \qquad \text{and} \qquad \gamma \leq
\frac{\alpha}{h}
$$
we can absorb the negative terms into the positive ones. Therefore it
is necessary to choose 
$$
\alpha > c p^2
$$
\end{proof}

For the error analysis we apply the discrete stability and
consistency:
\begin{eqnarray*}
\| u_h - I_h u\|_{1,h} & \leqc & \sup_{v_h}  \frac{ A(u_h - I_h u,
                                 v_h)  }{ \| v_h \|_{1,h}} \\
& = & \sup_{v_h}  \frac{ A(u - I_h u,
                                 v_h)  }{ \| v_h \|_{1,h}} 
\end{eqnarray*}
We cannot argue with continuity of $A(.,.)$ on $H^1$ (which is not
true), but we can estimate the interpolation error $u - I_hu$ for all four
terms of $A(u_h - I_h u, v_h)$.

\subsection{Nitsche's method for interface conditions}
We give now a variational formulation for interface conditions $u_1 =
u_2$, $\partial_{n_1} u_1 + \partial_{n_2} u_2 = 0$ on the interface
$\gamma$ separating $\Omega_1$ and $\Omega_2$. Boundary conditions on
 the outer boundary are treated as usual. Integration by parts on the
 sub-domains leads to
$$
\int_{\Omega_1} \nabla u \nabla v - \int_\gamma \partial_{n_1} u_1 v_1  + 
\int_{\Omega_2} \nabla u \nabla v - \int_\gamma \partial_{n_2} u_2 v_2 =
\int f v
$$
We define the mean value
$$
\{ \partial_{n_1} u \} = \tfrac{1}{2} ( \partial_{n_1} u_1
+ \partial_{n_2} u_2 )
$$
and jump 
$$
[v] = v_1 - v_2.
$$
Using continuity of the normal flux (taking the orientation into
account) we get
$$
\sum_i \int_{\Omega_i} \nabla u \nabla v - \int_{\gamma}
\{ \partial_{n_1} u \}  [v] = 
\int f v.  
$$ 
Note that both terms, mean of normal derivative and jump, change sign
if we exchange the enumeration of sub-domains.

We procede as before and add consistent symmetry and stabilization
terms:

$$
\sum_i \int_{\Omega_i} \nabla u \nabla v - 
\int_{\gamma} \{ \partial_{n_1} u \}  [v] -
\int_{\gamma} \{ \partial_{n_1} v \}  [u] +
\frac{\alpha}{h} \int_{\gamma} [u ] [v] =
\int f v. 
$$
The variational formulation  is consistent on the solution, and
elliptic on  $V_h$, which is proven as before. This approach is an alternative to the
mixed method (mortar method), since it leads to positive definite
matrices (called also gluing method). 

\section{DG for second order equations}
Nitsche's method for interface conditions can be applied
element-by-element. This is the (independently developed)
Discontinuous Galerkin (DG) method. Precisely, it's called SIP-DG
(symmetric interior penalty)  DG:
$$
A(u,v) = \sum_T \left\{ \int_T \nabla u \nabla v
  - \tfrac{1}{2} \int_{\partial_T} \partial_n u [v]  
- \tfrac{1}{2} \int_{\partial_T} \partial_n v [u] 
  +  \tfrac{\alpha}{h} \int_{\partial_T} [u] [v] \right\}
$$
(and proper treatment of integrals on the domain boundary). The
factor $\tfrac{1}{2}$ is coming from splitting the consistent terms to
the two elements on the edge.

Convergence analysis similar to Nitsche's method.


Beside the SIP-DG, also different version are in use: The NIP-DG
(non-symmetric interior penalty) DG:
$$
A(u,v) = \sum_T \left\{ \int_T \nabla u \nabla v
  - \tfrac{1}{2} \int_{\partial_T} \partial_n u [v]  
+ \tfrac{1}{2} \int_{\partial_T} \partial_n v [u] 
  +  \tfrac{\alpha}{h} \int_{\partial_T} [u] [v] \right\}
$$
The term formerly responsible for symmetry is added with a different
sign. The variational problem is still consistent on the true
solution. The advantage of the NIP-DG is that
$$
A(u,u) = \sum_T \| \nabla u \|^2_{L_2(T)} + \tfrac{\alpha}{h} \| [u]
\|_{L_2(\partial T)}^2,
$$
i.e. $A(.,.)$ is elliptic in any case $\alpha > 0$. The disadvantage
is that $A(.,.)$ is not consistent for the dual problem, i.e. the
Aubin-Nitsche trick cannot be applied. It is popular for
convection-diffusion problems, where the bi-form is non-symmetric
anyway. The IIP-DG (incomplete) skips the third term
completely. Advantages are not known to the author.

\subsection{Hybrid DG}
One disadvantage of DG - methods is that the number of degrees of
freedom is much higher than a continuous Galerkin method on the same
mesh. Even worse, the number of non-zero entries per row in the system
matrix is higher. The second disadvantage can be overcome by hybrid DG
methods: One adds additional variables $\hat u$, $\hat v$ on the inter-element facets (edges in 2D,
faces in 3D). The derivation is very similar:
$$
\sum_T \int_T \nabla u \nabla v - \int_{\partial T} \partial_n u v = \sum_T f  
v  \qquad \forall \, v \in P^k(T), \forall \, T  
$$ 
Using continuity of the normal flux, we may add $\sum_T \int_{\partial
  T} \partial_n u \hat v$ with a single-valued test-function on the
facets:
$$
\sum_T \int_T \nabla u \nabla v - \int_{\partial T} \partial_n u
(v-\hat v) = \sum_T f 
v  \qquad \forall \, v \in P^k(T), \forall \, T 
$$
Again, we smuggle in consistent terms for symmetry and coercivity:
$$
\sum_T \int_T \nabla u \nabla v
 - \int_{\partial T} \partial_n u (v-\hat v) 
 - \int_{\partial T} \partial_n v (u-\hat u) 
 + \frac{\alpha}{h} \int_{\partial T}  (u-\hat u) (v-\hat v) 
= \sum_T f 
v  \qquad \forall \, v \in P^k(T), \forall \, T 
$$
The jump between neighbouring elements is now replaced by the
difference of element-values and facet values. The natural norm is
$$
\| u, \hat u\|^2 = \sum_T \| \nabla u \|^2 + \tfrac{1}{h}  \| u - \hat
u \|_{\partial T}
$$
The HDG methods allows for static condensation of internal variables
which results in a global system for the edge-unknowns, only.

The lowest order method uses $P^1(T)$ and $P^1(E)$, and we get $O(h)$
convergence. When comparing with the non-conforming $P^1$-method, HDG
has more unknowns on the edges, but the same order of convergence. The
Lehrenfeld-trick is to smuggle in a projector:
$$
\sum_T \int_T \nabla u \nabla v
 - \int_{\partial T} \partial_n u (v-\hat v) 
 - \int_{\partial T} \partial_n v (u-\hat u) 
 + \frac{\alpha}{h} \int_{\partial T}  P^{k-1}(u-\hat u) (v-\hat v) 
= \sum_T f 
v  \qquad \forall \, v \in P^k(T), \forall \, T 
$$
This allows to reduce the order on edges by one, while maintaining the
order of convergence.

\subsection{Bassi-Rebay DG}
%
One disadvantage of IP-DG is the necessary penalty term with $\alpha$
sufficiently large. For well-shaped meshes $\alpha=5 p^2$ is usually
enough. But, for real problems the element deformation may become large,
and then a fixed $\alpha$ is not feasible. Setting $\alpha$ too large
has a negative effect for iterative solvers. 

An alternative is to replace the penalty term by
$$
\| [u] \|_{BR}^2 := \sup_{\sigma_h \in [P^{k-1}]^d } \frac{ ([u],
  n\cdot \sigma_h)_{L_2(\partial T)}^2 }{ \| \sigma_h \|_{L_2(T)}^2}
$$
It can be implemented by solving a local problem with
$L_2$-bilinear-form.

In the coercivity proof, the {\it bad } term is now estimated as
$$
\int_{\partial T}  n \cdot \nabla u_h  \, [u_h]  \leq \sup_{\sigma_h }
\frac{ \int_{\partial T}  n \cdot \sigma_h  \, [u_h]  } { \| \sigma_h \|
  }  \, \| \nabla u_h \| \leq \| \nabla u_h \| \, \| [u] \|_{BR}
$$
The BR-norm scales like the IP - norm (in $h$ and $p$), but the
(typically unknown) constant in the inverse trace inequality can be avoided.

\subsection{Matching integration rules}
Another method to avoid  guessing the sufficiently large $\alpha$ is
to use integration rules, such that the integration points for the
boundary integral are a sub-set of the integration points of the
volume term. Now, Young's inequality can be applied for the numerical
integrals. The $\tfrac{\alpha}{h}$ factor is now replaced by the
largest relative scaling of weights for the boundary integrals and
volume integrals. The pro is the simplicity, the con is the need of
numerical integration rules which need more points.

\subsection{(Hybrid) DG for Stokes and Navier-Stokes}
%
DG or HDG methods allow the construction of numercal methods for
incompressible flows, which
obtain exactly divergence free discrete velocities. We discretize
Stokes's equation as follows:
$$
V_h = BDM^k \qquad Q_h = P^{k-1,dc},
$$
and the bilinear-forms
$$
a(u_h, v_h) = a^{DG} (u_h, v_h)
$$
and
$$
b(u_h, q_h) = \int \opdiv u_h \, q_h.
$$
Since the space $V_h$ is not conforming for $H^1$, the DG - technique
is applied. The $b(.,.)$ bilinear-form is well defined for the
$H(\opdiv)$-conforming finite element space $V_h$. There holds
$$
\opdiv V_h = Q_h,
$$
and thus the discrete divergence free condition
$$
\int \opdiv u_h q_h = 0 \; \quad \forall q_h
$$
implies
$$
\| \opdiv u_h \|_{L_2}^2 = \int \opdiv u_h \underbrace {\opdiv
  u_h}_{\in Q_h} = 0.
$$
Hybridizing the method leads to facet variables for the tangential
components, only. This method can be applied to the Navier Stokes
equations. Here, the exact divergence-free discrete solution leads to
a stable method for the nonlinear transport term (References: Master
thesis Christoph Lehrenfeld: HDG for Navier Stokes, h-version LBB, 
Master thesis Philip Lederer: p-robust LBB for triangular elements).

