% \documentclass[12pt]{article}
% \usepackage{amsmath,amsthm,amssymb,a4wide}
% \usepackage[german,english]{babel}
% \usepackage{epsfig}
% \usepackage{latexsym}
% \usepackage{amssymb}
% % \usepackage{theorem}
% \usepackage{amsthm}
% % \usepackage{showkeys}

% \newcommand{\setR}{ {\mathbb R} }
% \newcommand{\setN}{ {\mathbb N} }
% \newcommand{\setZ}{ {\mathbb Z} }
% \newcommand{\eps}{\varepsilon}

% \newcommand{\beq}{\begin{equation}}
% \newcommand{\eeq}{\end{equation}}

% \newcommand{\opdiv}{\operatorname{div}}
% \newcommand{\opcurl}{\operatorname{curl}}
% \newcommand{\opdet}{\operatorname{det}}
% \newcommand{\optr}{\operatorname{tr}}
% \newcommand{\optrn}{\operatorname{tr}_n}
% \newcommand{\sfrac}[2]{ { \textstyle \frac{#1}{#2} } }

% \newcommand{\Zh}{\mathrm{Z}_h}
% \newcommand{\Ih}{\mathrm{I}_l}

% \newcommand{\leqc}{\preceq} 
% \newcommand{\geqc}{\succeq} 
% \newcommand{\eqc}{\simeq} 
% \newcommand{\ul}{\underline}

% \newtheorem{theorem}{Theorem}
% \newtheorem{definition}[theorem]{Definition}
% \newtheorem{lemma}[theorem]{Lemma}
% \newtheorem{remark}[theorem]{Remark}
% \newtheorem{example}[theorem]{Example}

% %
% %
% \setlength{\unitlength}{1cm}
% \sloppy 
% %

% \title{Space-time formulation of Parabolic Equations}
% \author{Joachim Sch\"oberl}

% \begin{document}
% \maketitle
% \centerline{(supplement to lecture notes Chapter 9, draft version) } 

\section{Space-time formulation of Parabolic Equations}

In the previous section we have discretized in space to obtain an ordinary differential equation, which 
is solved by some time-stepping method. This approach is known as method of lines.
Now we formulate a space-time variational problem. This is discretized in time and space
by a (discontinuous) Galerkin method. We obtain time-slabs which are solved one after another.
This approach is more flexible, since it allows to use different meshes in space on different time-slabs.

\subsection{Solvability of the continuous problem}
%
Let $V \subset H$ be Hilbert spaces, typically $H = L_2(\Omega)$ and $V = H^1(\Omega)$. Duality is defined with respect to $H$.
For $t \in (0,T)$ we define the familiy $A(t) : V \rightarrow V^\ast$ of uniformely continuous and elliptic operators:
\begin{enumerate}
\item[(a)] $\left< A(t) u, u \right> \geq \alpha_1 \| u \|_V^2 $
\item[(b)] $\left< A(t) u, v \right> \leq \alpha_2 \| u \|_V  \, \| v \|_V$ 
\end{enumerate}
We assume that $\left< A(t) u, v \right>$ is integrable with respect to time. We do not assume that $A(t)$ is symmetric.
We consider the parabolic equation: Find $u : [0,T] \rightarrow V$ such that
\begin{eqnarray*}
u^\prime + A u & = & f  \quad \forall \, t \in (0,T) \\
u(0) & = & u_0
\end{eqnarray*}
We define $X = \{ v \in L_2(V) : v^\prime \in L_2(V^\ast) \} $ and $Y = L_2(V)$, with its dual $Y^\ast = L_2(V^\ast)$.
A variational formulation of is: Find $u \in X$ such that
\begin{eqnarray} 
\label{equ_parabolic1}
\int_0^T \left< u^\prime + A u, v \right> & = & \int_0^T \left< f, v \right>   \qquad \forall \, v \in Y \\
\label{equ_parabolic2}
(u(0), v_0)_H & = & (u_0, v_0) \qquad \forall v_0 \in H
\end{eqnarray}

Adding up both equations leads to the variational problem $B(u,v) = f(v)$ with the bilinear-form
$B(.,.) : X \times (Y \times H) \rightarrow \setR$:
$$
B(u, (v,v_0)) = \int_0^T \left< u^\prime + A u, v \right> + (u(0), v_0)_H
$$
and the linear-form $f : Y \times H \rightarrow \setR$:
$$
f(v,v_0) = \int_0^T \left< f, v \right> + (u_0, v_0)_H
$$ 
We assume that $f \in Y^\ast$ and $u_0 \in H$

\begin{theorem}[Lions] \label{theo_continuous} Problem~(\ref{equ_parabolic1})-(\ref{equ_parabolic2}) is uniquely solvable.
\end{theorem}
\begin{proof}
We apply the theorem by Babu\v{s}ka-Aziz. We observe that all forms are continuous (trace-theorem). 
We have to verify both inf-sup conditions.

First, we show
\begin{equation}
\inf_{u \in X} \sup_{ (v,v_0) \in Y \times H} \frac{B(u,v)} {\| u \|_X  \| (v,v_0) \|_{Y \times H}}  \geq \beta > 0
\end{equation}
We fix some $u \in X$ and set (with $A^{-T}$ the inverse of the adjoint operator)
\begin{eqnarray*}
v &  := & A^{-T} u^\prime + u \\
v_0 & := & u(0)
\end{eqnarray*}
and obtain
\begin{eqnarray*}
B(u,v) & = & \int \left< u^\prime + A u, A^{-T} u^\prime + u \right> \, dt \; + (u(0), u(0))_H \\
& = & \int \left<A^{-T} u^\prime, u^\prime \right> + \left< A u, u \right> + \left< u^\prime, u \right> + \left< u, u^\prime \right> \, dt \;  + \| u_0 \|_H^2 \\
& = & \int \left<A^{-T} u^\prime, u^\prime \right> + \left< A u, u \right> + \frac{d}{dt} \| u \|_H^2  + \| u_0 \|_H^2 \\
& \geq & \alpha_2^{-1}  \| u^\prime \|^2_{L_2(V^\ast)} + \alpha_1 \| u \|^2_{L_2(V)} + \| u(T) \|_H^2 \\
& \geqc &  \| u \|_X^2
\end{eqnarray*}
Since $\| (v,v_0) \|_Y \leqc \| u \|_X$ the first $\inf-\sup$-condition is proven.
For the other one, we show
\begin{equation}
\forall \, 0 \neq (v,v_0) \in Y \times H \; \; \exists \, u \in X  \;  : \;  B(u,v) > 0
\end{equation}
We fix some $v, v_0$. We define $u$ by solving the parabolic equation
$$
u^\prime + \gamma L u = A^T v, \qquad u(0) = v_0,
$$
where $L$ is a symmetric, constant-in-time, continuous and elliptic operator on $V$. 
The parameter~$\gamma > 0, \gamma = O(1)$ will be fixed later. The equation has a unique solution, which can be constructed by spectral theory.
If $(v,v_0) \neq 0$, then also $u \neq 0$.

\begin{eqnarray*}
B(u,v) & = & \int \left< u^\prime + A u, A^{-T} (u^\prime + \gamma L u) \right> + \| v_0\|_H^2 \\
& = & \int \left< u^\prime, A^{-T} u^\prime\right> + \left< u, u^ \prime\right> + \left< u^\prime, A^{-T } \gamma L u \right> + \gamma \left< u, L u \right> \, dt +  \| v_0 \|_H^2 \\
& \geq & \int \tfrac{1}{\alpha_2} \| u^\prime \|_{V^\ast}^2 + \tfrac{1}{2} \tfrac{d}{dt} \| u \|_H^2 - \| u^\prime \|_{V^\ast} \| A^{-T} \gamma L u \|_V + \gamma \left< u, L u \right> + \| v_0 \|_{H}^2
\end{eqnarray*}
The second term is integrated in time, and we apply Young's inequality for the negative term:
\begin{eqnarray*}
B(u,v) & \geq & \int \tfrac{1}{\alpha_2} \| u^\prime \|_{V^\ast}^2  - \tfrac{1}{2\alpha_2}\| u^\prime \|_{V^\ast}^2 - \tfrac{\alpha_2}{2} \| A^{-T} \gamma L u \|_V^2 + \gamma \left< u, L u \right> + \tfrac{1}{2} \| v_0 \|_{H}^2 + \tfrac{1}{2}  \| v(T) \|_H^2 \\
& \geq &  \int \tfrac{1}{2 \alpha_2} \| u^\prime \|_{V^\ast}^2  - \tfrac{\alpha_2 \gamma^2}{2} \| A^{-T} L \|_{V \rightarrow V}^2  \| u \|_V^2 + \gamma \left< u, L u \right> + \tfrac{1}{2} \| u(0) \|_{H}^2 
\end{eqnarray*}
We fix now $\gamma$ sufficiently small such that $\tfrac{\alpha_2 \gamma^2}{2} \| A^{-T} \gamma L \|_{V \rightarrow V}^2  \| u \|_V^2 \leq \gamma \left< u, L u \right>$ to obtain
$$
B(u,v) \geqc \| u^\prime \|_{L_2(V^\ast)}^2 + \| u_0 \|_H^2 > 0.
$$
\end{proof}
A similar proof of Lions's theorem is found in Ern + Guermond.
\subsection{A first time-discretization method}
We discretize in time, but keep the spatial function space infinit dimensional. 
A first reasonable attempt is to use $X_h = P^1(V)$, and $Y_h = P^{0,disc}(V)$. 
Evaluation of $B(.,.)$ leads to
\begin{eqnarray*}
B(u_h, v_h) & = & \sum_{j=1}^n \int_{t_{j-1}}^{t_j} \left< u_h^\prime + A u_h , v_h \right> + (u_0, v_h)  \\
& = & \sum_{j=1}^n \left< u_j - u_{j-1}, v_j \right> + \tfrac{\tau_j}{2} \left<A (u_{j-1} + u_j), v_j \right> + (u_0, v_0)
\end{eqnarray*}
Here, the time derivative evaluates to finite differences of point values in $t_j$. Since $u_j \in V$, the duality pairs
coincide with inner products in $H$. Thus, for every time-step we get the equation
$$
u_j - u_{j-1} + \frac{\tau}{2} A (u_j + u_{j-1}) = \tau f_j
$$
This is the trapezoidal method (Crank-Nicolson). From numerics for odes we remember it is A-stable, but not L-stable.
We cannot prove a discrete $\inf-\sup$ condition.

\subsection{Discontinuous Galerkin method}

We give an alternative, formally equivalent variational formulation for the parabolic equation by integration by parts in time
$$
\int - \left< u, v^\prime \right> + \left< A u, v \right> + (u(T),v(T))_H - (u(0), v(0))_H = \int  \left< f, v \right> 
$$ 
Now, we plug in the given initial condition $u(0) = u_0$:
$$
\int - \left< u, v^\prime \right> + \left< A u, v \right> + (u(T),v(T))_H = \int  \left< f, v \right> + (u_0, v(0))_H
$$
The higher $H^1$-regularity is now put onto the test-space, which validates point-evaluation at $t=0$ and $t=T$. The trial-space is now only $L_2$, which gives no meaning for $u(T)$. There are two possible remedies, either to introduce a new variable for $u(T)$, or, to restrict the test space:
\begin{enumerate}
\item Find $u \in L_2(V)$, $u_T \in H$ such that
\begin{equation}
\int - \left< u, v^\prime\right> + \left< A u, v\right>  + (u_T, v(T))_H = \int \left< f, v \right> + (u_0, v(0))_H \quad \forall \, v \in L_2(V) , v^\prime \in L_2(V^\ast)
\end{equation}
\item Find $u \in L_2(V)$ such that
\begin{equation}
\int - \left< u, v^\prime\right> + \left< A u, v\right>  = \int \left< f, v \right> + (u_0, v(0))_H \quad \forall \, v \in L_2(V) , v^\prime \in L_2(V^\ast), v(T) = 0
\end{equation}
\end{enumerate}
Both problems are well posed (continuity and $\inf-\sup$ conditions, exercise). Now, the initial condition was converted from an essential to a natural boundary condition. 

Next, we integrate back, but we do not substitute the initial condition back:
$$
\int \left< u^\prime + A u, v \right> + (u(0), v(0))_H = \int \left< f, v \right> + (u_0, v(0))
$$
The initial condition is again a part of the variational formulation. Note that this formulation is fulfilled for $u \in H^1$, and smooth enough test functions providing the trace $v(0)$.

This technique to formulate initial conditions is used in the Discontinuous Galerkin (DG) method. For every time-slab $(t_{j-1}, t_j)$ we define a parabolic equation, where the initial value is the end value of the previous time-slab.

Here, we first define a mesh $\mathcal T = \{t_0, t_1, \ldots t_n\}$, and then the mesh-dependent formulation:
$$
\int_{t_{j-1}}^{t_j} \left< u^\prime + A u , v \right>  + (u(t_{j-1}^+), v(t_{j-1}^+))_H = \int_{t_{j-1}}^{t_j}  \left< f, v \right> +  (u(t_{j-1}^-), v(t_{j-1}^+))_H  \qquad \forall \, j \in \{ 1, \ldots n\} 
$$
(with the notation $u(t_0^-) := u_0$).
By using left and right sided limits, we get the $u$ from the current time-slab, and the end-value from the previous time-slab, respectively.
The variational formulation is valid for the solution $u \in H^1$, and piece-wise regular test-functions on the time-intervals.

The bilinear-form is defined as
$$
B(u,v) = \sum_{j=1}^n \int_{t_{j-1}}^{t_j} \left< u^\prime + A u , v \right>  + ([u]_{t_{j-1}}, v(t_{j-1}^+))_H 
$$
where the jump is defined as $[u]_{t_j} = u(t_j^+)-u(t_j^-)$, and the special case $[u]_{t_0} = u(t_0^+)$. The solution satisfies
$$
B(u,v) =  \int \left< f, v \right> + (u_0, v(0)) \qquad \forall \, \text{p.w. smooth} \, v
$$

The bilinear-form is defined for discontinuous trial and discontinuous test functions. It allows to define 
$$
X_h = Y_h = P^{k,dc} (V)
$$
Let us elaborate the case of piece-wise constants in time:
$$
\tau \left< A u_j, v_j \right> + ( u_j - u_{j-1}, v_j) = \tau \left<f_j, v_j \right>
$$
which leads to the implicit Euler method
$$
\frac{u_j - u_{j-1}}{\tau} + A u_j = f_j
$$
The imlicit Euler method is $A$ and $L$-stable.

We define the mesh-dependent norms
\begin{eqnarray*}
\| u \|_{X_h}^2 &= &\sum_j \| u \|_{L_2(t_{j-1}, t_j, V)}^2 + \| u^\prime \|_{L_2(t_{j-1}, t_j, V^\ast)}^2 + \tfrac{1}{t_j - t_{j-1}} \| [u] _{t_{j-1}} \|_{V^*}^2 \\
\| v \|_{Y_h}^2 &= &\sum_j \| v \|_{L_2(t_{j-1}, t_j, V)}^2 
\end{eqnarray*}
Since $v|_{[t_{j-1}, t_j]}$ is a polynomial, we can bound
$$
\| v(t_{j-1}^+) \|_V^2 \leq \frac{c}{t_{j} - t_{j-1}} \| v \|^2_{L_2(t_{j-1}, t_j, V)}, 
$$
where the constant $c$ deteriors with the polynomial degree. Thus, the bilinear-form is well defined and continuous on $X_h \times Y_h$.

\begin{theorem}
The discrete problem is $\inf-\sup$ stable on $X_h \times Y_h$ 
\end{theorem}
\begin{proof} We mimic the first $\inf-\sup$ condition in Theorem~\ref{theo_continuous}, where we have set $v = u + A^{-T} u^\prime$. We give the proof for the lowest order case (k=0)
$$
v_h := u_h + \tfrac{\gamma}{t_j - t_{j-1}} A(t_{j-1})^{-1}  [u_h]_{j-1},
$$
with $\gamma = O(1)$ to be fixed later. Thanks to the discontinuous test-space, this is a valid test-function. 

In the following we skip the subsripts $h$, and set $\tau = t_{j} - t_{j-1}$. 
There holds
$$
\| v \|_{Y_h} \leqc \| u \|_{X_h}.
$$
\begin{eqnarray*}
B(u_h, v_h) & = & \sum_{j=1}^n \int_{t_{j-1}}^{t_j} \left< A u, u + \tfrac{\gamma}{\tau}  A_{j-1}^{-T} [u]_{j-1} \right> + 
\sum_j ([u]_{j-1}, u + \tfrac{\gamma}{\tau} A_{j-1}^{-1} [u]_{j-1})_H \\
& = & \int \left< A u, u \right> + \sum_j \int \left< A u, \tfrac{\gamma}{\tau} A^{-T} [u_j]  \right> +
 \sum_j ( u_j - u_{j-1}, u_j)_H + \sum_j \tfrac{\gamma}{\tau} \| [u]_{t_{j-1}}\|_{A^{-1}} 
\end{eqnarray*}
The second term is split by Young's inequality:
$$
\int_{t_{j-1}}{t_j} \left< A u, \tfrac{\gamma}{\tau}  A^{1}_{j-1} [u]_{j-1} \right> \leq \int \tfrac{1}{2} \left<A u, u \right> + \tfrac{\gamma^2}{2 \tau^2}  \int \| A^{-1} [u] \|_A 
$$
Thus, for sufficiently small $\gamma$ it can be absorbed into the first and last term.

We reorder the summation of the third term:
\begin{eqnarray*}
& & (u_1, u_1)^2 - (u_1, u_2) + (u_2, u_2)^2 - (u_2, u_3) + \ldots \\
& = &\tfrac{1}{2} \| u_1 \|_H^2 + \tfrac{1}{2}  \| u_1 - u_2 \|_H^2 + \ldots
\end{eqnarray*} 

Thus, we got (for piecewise  constants in time):
$$
B(u_h, v_h) \geq \| u \|_{L_2,V}^2 + \sum \| [u]_{t_j} \|_H^2 + \frac{1}{\tau} \| [u] \|_{V^\ast}^2 \geqc \| u_h \|_{X_h}^2 
$$
\end{proof}

By stability, we get for the discrete error
\begin{eqnarray*} 
\| I_h u - u_h \|_{X_h}  & \leqc & \sup_{v_h}  \frac{B_h(I_h u - u_h, v_h)} {\| v_h \|_{Y_h} )} \\
 & = & \sup_{v_h}  \frac{B_h(I_h u - u, v_h)} {\| v_h \|_{Y_h} )} \\
 & = & \sup_{v_h}  \frac{ \sum_j \int \left< u^\prime + A u - (I_h u)^\prime + A I_h u, v_h \right> + \sum_j ( [u]  - [I_h u] , v_h)_H }
{ \| v_h \|_{L_2(V)} }  \\
& \leqc & \ldots
\end{eqnarray*} 
where the convergence rate depends as usual on the regulariy of the exact solution 


